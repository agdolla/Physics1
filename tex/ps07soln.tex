\documentclass[12pt]{article}
\usepackage{amssymb}
\usepackage{amsfonts}
\usepackage{epsfig,latexsym}
\voffset -.5cm
\hoffset -1.5cm
\textheight 21cm
\textwidth 16cm
\def\dspace{\baselineskip = .30in}

\def\beq{\begin{equation}}
\def\eeq{\end{equation}}
\def\be{\begin{eqnarray}}
\def\ee{\end{eqnarray}}

\begin{document}
\begin{center}
{\bf\large Homework 7.}
\end{center}

{\bf Problem 1.}

(a)  Due to conservation of momentum
$$mv=(m+M)v',\;\;\;\; v'=\frac{m}{m+M}v$$

(b) $$E_i=\frac{mv^2}{2},\;\;\;E_f=\frac{(m+M)v'^2}{2}=\frac{m}{m+M}\frac{m v^2}{2},\;\;\;\Delta K=E_i-E_f=\frac{M}{m+M}\frac{m v^2}{2}$$

(c) After the collision energy is conserved
$$mg\Delta h= E_f=\frac{m}{m+M}\frac{m v^2}{2},\;\;\;\Delta h= \frac{m}{m+M}\frac{ v^2}{2 g}$$

(d) In part (a) we have inelastic collision kinetic energy is not conserved, in part (c) we consider what
happened after collision (kinetic energy is already lost in collision), thus kinetic energy again is conserved.

(e) Speed of the block is $u\le v'$. Let's find deflection angle $\alpha$ after collision. $\alpha l=u \Delta t\le v' \Delta t$, where $\Delta t$ is sticking time, $l$ length of the string. Velocity of bullet
with respect to the block is $v_b$ and size of the block is  $a$, then $\Delta t=a/v_b$ and we get
$\alpha\le\frac{v'}{v_b}\frac{a}{l}$. From here one can see if $l$ is big enough, $\alpha$ is negligible (moreover $h\sim\alpha^2$).
\\

{\bf Problem 2.}

$$\frac{m<v^2>}{2}=\frac{3}{2}kT$$
Mass of $N_2$ is 28 atomic units or in grams $m=28/N_A$ where $N_A$ is Avagadro's number.
$v_{av}=\sqrt{3kT/m}\approx515\;m/s$ where we used $k=1.38\times 10^{-23} J/K$ and for room temperature $T\approx 300\;K$. 

Speed of sound $v_s\approx 330\;m/s$ the result we got is about $1.5$ times bigger.

The typical magnitude  of momentum is $\Delta p= 2 p=2mv_{av}\approx 5\times 10^{-23}kg\;m/s$ ($2p$ is because molecules bounce of a hand).

The pressure is $P=15\;lb/in^2\approx 8\;N/in^2\approx 1.3\; N/cm^2\approx 1.3\times 10^{-4}Pa$.
One molecule bounced of a hand produces force $F_1=\Delta p/1\;s$, for total force we should
multiply this by number of molecules hitting hand per second $F=N \Delta p$. Pressure is force
per unit area. Area of a hand is about $A=100\;cm^2$, thus for pressure we get $P=N\Delta p/A=1.3\;N/cm^2$. From here for number of molecules we have $N=P A/\Delta p\approx 2.5\times 10^{24}\;molecules/s$.\\
\\


{\bf Problem 3.}

(a) Momentum which sail gets after collision with one molecule is $\Delta p\approx 2 p \cos\theta$,
because the tangent  to the sail  component of momentum $p\sin\theta$ doesn't change, only 
normal one get flipped. For $\theta=0$,  $\Delta p= 2 p $ and for  $\theta=\pi/2$,  $\Delta p= 0$
(tangent wind).

(b) If we'll look at one dimensional problem than for elastic collision
$$m_1 v_1 +m_2 v_2=m_1 u_1 +m_2 u_2,\;\;\;\mbox{-- conservation of momentum}$$
$$
\frac{m_1 v_1^2}{2} +\frac{m_2 v_2^2}{2}=\frac{m_1 u_1^2}{2} +\frac{m_2 u_2^2}{2},\;\;\;\mbox{-- conservation of energy.}$$
Resolving the system and making approximation $m_2\gg m_1$ (sail is much more heavy than a molecule)  we get
\be
u_1=\frac{m_1-m_2}{m_1+m_2}v_1 +\frac{2 m_2}{m_1+m_2}v_2\approx -v_1 +2v_2,\;\;\;\;
u_2=\frac{2 m_1}{m_1+m_2}v_1 +\frac{ m_2-m_1}{m_2+m_1}v_2\approx v_2.\nonumber
\ee
Thus, perpendicular to the sail component of velocity just flips and gives $2p=2m_1v_1$ for change of momentum of the resting sail ($v_2=0$). Nothing happens to the tangent part of velocity (there are no collisions). 

(c) Sailboat sails at $90^o$ to the wind. It gets momentum from a molecule
$\Delta p=2 p\cos\theta$. Total momentum is proportional to the number of molecules
and it is proportional to the effective (projected) area of the sail $p_{total}\sim\cos\theta\cos\theta$.
This momentum should be projected on direction of the boat. Thus the momentum
driving boat at angle $\beta=90^o$ to the wind is $p_d\sim  \cos^2\theta\cos(\beta-\theta)$.
For $90^o$ we get  $p_d\sim  (1-\sin^2\theta)\sin\theta$ which maximizes at $\sin\theta=1/\sqrt{3},\;\;\theta\approx 35^o$.
\\

{\bf Problem 4.}

Momentum is conserved. Let's write it for small times
$-dm u=m dv$, where $u$ is speed of fuel expelling $dm$ is mass of expelled fuel in very small
time, $dv$ is speed gained by rocket due to this. Integrating this we get 
$$dv=-u\frac{dm}{m},\;\;\;\;\int_{v_0}^{v}dv=-u\int_{m_0}^{m_p}\frac{dm}{m},\;\;\;\;v=v_0 -u\log\frac{m_p}{m_0}=v_0+u\log\frac{m_0}{m_p},$$
where $m_0=m_p+m_f$ is initial mass (package plus fuel), $m_p$ is final mass (probably just package itself),
$v_0=0$ is initial velocity. To put package on the orbit it should get speed $v=\sqrt{g R}$ where
$g$ is acceleration of gravity, $R$ is radius of the Earth.
$$\sqrt{g R}=u\log\frac{m_p+m_f}{m_p},\;\;\;\;\;\; m_f=m_p\left(\exp\frac{\sqrt{gR}}{u}-1\right).$$

Speed of sound  is relevant to speed of motion of molecules (it should be smaller), expel speed of propellant is due to motion of propellant molecules (they are about 3 times heavier than nitrogen) and they should produce pressure bigger then pressure of air. At higher temperatures, speed of molecules is higher so is for speed of waves in a gas.



\end{document}