\documentclass[12pt]{article}
\usepackage{graphics}
\begin{document}
\newcounter{problem}
\thispagestyle{empty}

\section*{NYU Physics I---Problem set 3}

Due Thursday 2016 September 29 at the beginning of lecture.

\paragraph{Problem~\theproblem:}\refstepcounter{problem}%
In a 200-m race, the winner crosses the halfway mark (ie, 100~m) at
$t_{1/2}=10.12$~s and the finish line at $t_{\rm f}=19.32$~s.  If you
make a kinematic model of the runner's behavior as (i)~constant
acceleration $a$ from rest for the period $0<t<t_a$ followed by
(ii)~constant speed $v=a\,t_a$ during the period $t_a<t<t_{\rm f}$,
what are $a$ and $t_a$?

\emph{Hint: Draw a graph of $v(t)$ before you start writing equations.}

\emph{Hint: Once you have a graph, consider the question: is $t_a$
going to be before or after $t_{1/2}$?  Make a quantitative argument.}

\emph{Hint: You can solve this problem with geometry and the graph; no
  need to do calculus!}

\paragraph{Problem~\theproblem:}\refstepcounter{problem}%
\noindent~\hfill\includegraphics{../mp/pulley_table.pdf}\hfill~

In the system shown, the strings are massless and inextensible and the
pulley is massless and frictionless.  There is a small frictional
force $F_\mathrm{f}=\mu\,N$ between block $m_1$ and the table, but
when released from rest, block $m_2$ falls.

\textsl{(a)} Draw complete free-body diagrams for both masses, and for
the pulley.

\textsl{(b)} What is the acceleration $\vec{a}$ (magnitude and
direction) of mass $m_1$?  Give your answer in terms of quantities
shown in the diagram.

\textsl{(c)} Why did we treat the strings as massless? How did that
help us?

\paragraph{Problem~\theproblem:}\refstepcounter{problem}%
another newton problem

\end{document}
