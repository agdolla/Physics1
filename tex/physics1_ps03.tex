\documentclass[12pt]{article}
\usepackage{url, graphicx}

% page layout
\setlength{\topmargin}{-0.25in}
\setlength{\textheight}{9.5in}
\setlength{\headheight}{0in}
\setlength{\headsep}{0in}

% problem formatting
\newcommand{\problemname}{Problem}
\newcounter{problem}

% math
\newcommand{\dd}{\mathrm{d}}

% primary units
\newcommand{\rad}{\mathrm{rad}}
\newcommand{\kg}{\mathrm{kg}}
\newcommand{\m}{\mathrm{m}}
\newcommand{\s}{\mathrm{s}}

% secondary units
\renewcommand{\deg}{\mathrm{deg}}
\newcommand{\km}{\mathrm{km}}
\newcommand{\mi}{\mathrm{mi}}
\newcommand{\h}{\mathrm{h}}
\newcommand{\ns}{\mathrm{ns}}
\newcommand{\J}{\mathrm{J}}
\newcommand{\eV}{\mathrm{eV}}
\newcommand{\W}{\mathrm{W}}

% derived units
\newcommand{\mps}{\m\,\s^{-1}}
\newcommand{\mph}{\mi\,\h^{-1}}
\newcommand{\mpss}{\m\,\s^{-2}}

% random stuff
\sloppy\sloppypar\raggedbottom\frenchspacing\thispagestyle{empty}

\begin{document}

\section*{NYU Physics I---Problem Set 3}

Due Thursday 2017 September 28 at the beginning of lecture.

\paragraph{Problem~\theproblem:}\refstepcounter{problem}%
Re-do the ``block on a plane'' problem we did in class, but now using
vector components. Make the $x$ direction point down the plane, and
make the $y$ direction point perpendicular to the plane, and solve for
the $x$ and $y$ components of gravity, the normal force, and the
block's acceleration (that is, three vectors, and two components per
vector).

\paragraph{Problem~\theproblem:}\refstepcounter{problem}%
In a 200-m race, the winner crosses the halfway mark (ie, 100~m) at
$t_{1/2}=10.12$~s and the finish line at $t_{\rm f}=19.32$~s.  If you
make a kinematic model of the runner's behavior as (i)~constant
acceleration $a$ from rest for the period $0<t<t_a$ followed by
(ii)~constant speed $v=a\,t_a$ during the period $t_a<t<t_{\rm f}$,
what are $a$ and $t_a$?

\emph{Hint: Draw a graph of $v(t)$ before you start writing equations.}

\emph{Hint: Once you have a graph, consider the question: is $t_a$
going to be before or after $t_{1/2}$?  Make a quantitative argument.}

\emph{Hint: You can solve this problem with geometry and the graph; no
  need to do calculus!}

\paragraph{Problem~\theproblem:}\refstepcounter{problem}%

In the system shown below, the strings are massless and inextensible
and the pulley is massless and frictionless.  There is a small
frictional force $F_\mathrm{f}=\mu\,N$ between block $m_1$ and the
table, but when released from rest, block $m_2$ falls. Ignore air
resistance!

\noindent~\hfill\includegraphics{../mp/pulley_table.pdf}\hfill~

\textsl{(a)} Draw complete free-body diagrams for both masses, and for
the pulley.

\textsl{(b)} What is the acceleration $\vec{a}$ (magnitude and
direction) of mass $m_1$?  Give your answer in terms of quantities
shown in the diagram.

\textsl{(c)} Why did we treat the strings as massless? How did that
help us?

\paragraph{Problem~\theproblem:}\refstepcounter{problem}%
Imagine a package of mass $M$ sitting on the floor of an elevator that
is accelerating upwards at acceleration $a$. The acceleration due to
gravity is $g$.

\textsl{(a)} Draw the free-body diagram for the package.

\textsl{(b)} What is the magnitude $N$ of the normal force on the
package?

\textsl{(c)} Answer those same two questions again, but with the
elevator accelerating \emph{downwards} at acceleration $a$.

\textsl{(d)} Answer those same two questions again, but with the
elevator accelerating downwards at acceleration $g$ (that is, at
the gravitational acceleration).

\textsl{(e)} In lecture Prof Hogg did a stunt with a coffee cup and a
quarter. Draw the free-body diagram for the quarter when it was at the
top of the arc (that is, when it was directly overhead).

\paragraph{Extra Problem (will not be graded for credit):}%
In what sense were the astronauts in the Space Station weightless? Is
there no gravity acting? Look up the altitude at which the Space
Station orbited and ask yourself: How much weaker is the gravitational
force at that altitude? (The relevant distance is the distance from
the Space Station to the center of the Earth.)

Also, why didn't the quarter fall out of the cup?

\end{document}
