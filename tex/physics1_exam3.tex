\documentclass[12pt]{article} \usepackage{url, graphicx}

% page layout
\setlength{\topmargin}{-0.25in}
\setlength{\textheight}{9.5in}
\setlength{\headheight}{0in}
\setlength{\headsep}{0in}

% problem formatting
\newcommand{\problemname}{Problem}
\newcounter{problem}

% math
\newcommand{\dd}{\mathrm{d}}

% primary units
\newcommand{\rad}{\mathrm{rad}}
\newcommand{\kg}{\mathrm{kg}}
\newcommand{\m}{\mathrm{m}}
\newcommand{\s}{\mathrm{s}}

% secondary units
\renewcommand{\deg}{\mathrm{deg}}
\newcommand{\km}{\mathrm{km}}
\newcommand{\mi}{\mathrm{mi}}
\newcommand{\h}{\mathrm{h}}
\newcommand{\ns}{\mathrm{ns}}
\newcommand{\J}{\mathrm{J}}
\newcommand{\eV}{\mathrm{eV}}
\newcommand{\W}{\mathrm{W}}

% derived units
\newcommand{\mps}{\m\,\s^{-1}}
\newcommand{\mph}{\mi\,\h^{-1}}
\newcommand{\mpss}{\m\,\s^{-2}}

% random stuff
\sloppy\sloppypar\raggedbottom\frenchspacing\thispagestyle{empty}

\begin{document}

\noindent
Name: \rule[-1ex]{0.55\textwidth}{0.1pt}
NetID: \rule[-1ex]{0.2\textwidth}{0.1pt}

\section*{NYU Physics I---Term Exam 3}

\paragraph{\problemname~\theproblem:}\refstepcounter{problem}%
What is the potential energy gain by a typical NYU student who has
climbed a single flight of stairs in a typical building? What is the
typical amount of kinetic energy the student has during the climb?
(From Problem Set 5, \problemname~1.)

\vfill

\paragraph{\problemname~\theproblem:}\refstepcounter{problem}%
A block of mass $m$ is subject to gravitational acceleration $g$ and
rests on a horizontal surface. It is pulled horizontally a distance
$h$ by a horizontally oriented string. There is a coefficient of
sliding friction $\mu$ between the block and the plane. How much heat
is generated?  (From Problem Set 6, \problemname~2.)

\vfill

\paragraph{\problemname~\theproblem:}\refstepcounter{problem}%
In this hanging sign problem, what is the direction of the force on
the beam from the wall at the pivot? Circle your best option (From Problem set 6,
\problemname~3.)\\
(a) straight up\\
(b) up and to the left\\
(c) straight left\\
(d) down and to the left\\
(e) straight down\\
(f) down and to the right\\
(g) straight right\\
(h) up and to the right\\
(i) there is no force\marginpar{\includegraphics[width=1in]{../mp/hanging_sign.pdf}}

\clearpage
\paragraph{\problemname~\theproblem:}\refstepcounter{problem}%
A very massive ball is traveling upwards at speed $v$. A far, far less
massive ball is traveling downwards at speed $v$. They bounce off each
other elastically, and precisely aligned in the vertical
direction. After the collision, how fast is the less massive ball
traveling? As you did in recitation, assume that the less massive ball
is negligible in mass relative to the more massive ball, and that the
collision is perfect in elasticity and alignment. (From the recitation
worksheet on bouncing.)

\vfill

\paragraph{\problemname~\theproblem:}\refstepcounter{problem}%
A pool ball of mass $0.2\,\kg$ moving at speed $1\,\mps$ hits a
concrete floor. It bounces. Roughly what is the magnitude of the mean
force of the floor on the ball during the collision? Estimate (as we
did in Lecture) that the collision lasts for $0.001\,\s$. If you need
to make any additional assumptions, state them. (From Lecture on
2016-10-11.)

\vfill

\paragraph{\problemname~\theproblem:}\refstepcounter{problem}%
What is the second derivative of $x(t) = A\,\sin(\omega\,t)$ with respect to
$t$? (From Lecture on 2016-10-18.)

\vfill
~
\end{document}
