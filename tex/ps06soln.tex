\documentclass[12pt]{article}
\usepackage{amssymb}
\usepackage{amsfonts}
\usepackage{epsfig,latexsym}
\voffset 0cm
\hoffset -1.5cm
\textheight 20cm
\textwidth 16cm
\def\dspace{\baselineskip = .30in}

\def\beq{\begin{equation}}
\def\eeq{\end{equation}}
\def\be{\begin{eqnarray}}
\def\ee{\end{eqnarray}}

\begin{document}
\begin{center}
{\bf\large Homework 6.}
\end{center}

{\bf Problem 1.}
\begin{itemize}
\item {} Bullet fired from a rifle: speed $v\approx 1000\;m/s$,  mass $m\approx 10 \;g=.01\;kg$, momentum $p=mv\approx 10\;kg\;m/s$, energy $E=mv^2/2\approx 5000\;J$.
\item {} A soccer ball kicked hard: speed $v\approx 30\;m/s$ (soccer field size is 100m, it takes
about 3s for a ball to get a bit farther then middle of the field after goalkeeper kicks it),  mass $m\approx 500 \;g=.5\;kg$, momentum $p=mv\approx 15\;kg\;m/s$, energy $E=mv^2/2\approx 250\;J$.
\item {} A three year old on tricycle: speed $v\approx .5\;m/s$,  mass $m\approx 20\;kg$, momentum $p=mv\approx 10\;kg\;m/s$, energy $E=mv^2/2\approx 2.5\;J$.
\end{itemize}


{\bf Problem 2.}

Masses of the blocks $m_1=5\;kg$ and $m_2=2\;kg$, corresponding initial velocities
$v_1=7\;m/s$ and $v_2=-2\;m/s$. The collision is {\bf elastic} that means the {\bf energy is conserved}. Let's velocities after collision be $u_1$ and $u_2$ then
$$m_1 v_1 +m_2 v_2=m_1 u_1 +m_2 u_2,\;\;\;\mbox{-- conservation of momentum}$$
$$
\frac{m_1 v_1^2}{2} +\frac{m_2 v_2^2}{2}=\frac{m_1 u_1^2}{2} +\frac{m_2 u_2^2}{2},\;\;\;\mbox{-- conservation of energy.}$$
Let's rewrite above formulas by taking terms with $m_1$ to the left and with $m_2$ to the right:
\be
m_1( v_1 - u_1) =m_2( u_2-v_2),\;\;\; \frac{m_1}{2}(v_1^2-u_1^2) = \frac{m_2}{2}(u_2^2-v_2^2)\nonumber
\ee
Dividing one equation by the other we get following system
\be
m_1( v_1 - u_1) =m_2( u_2-v_2),\;\;\; v_1+u_1 = u_2+v_2.\nonumber
\ee
It is now trivial to solve
\be
u_1=\frac{m_1-m_2}{m_1+m_2}v_1 +\frac{2 m_2}{m_1+m_2}v_2=1\frac{6}{7}m/s,\;\;\;\;
u_2=\frac{2 m_1}{m_1+m_2}v_1 +\frac{ m_2-m_1}{m_2+m_1}v_2=10\frac{6}{7}m/s\nonumber
\ee
\\
\\

{\bf Problem 3.}

Let's see what forces are acting on a car: force of gravity down $mg$, reaction force normal to the  to the road $N$ and friction force tangent to the road $F_f$ and we know
that $|F_f|\le \mu N$ where $\mu$ is friction coefficient. Road have been optimize, that means reducing force of friction to zero. Thus centripetal force, necessary to make turn, is produced
just by the projection of force of reaction to the center of curvature $F_c=mv_0^2/R=N\sin\theta$. At the same time, reaction force should compensate gravity $mg=N\cos \theta$. From here 
we find slope $\theta$:  $\tan\theta = \frac{ v_0^2}{Rg}$ or $\theta\approx 51^o$, where $v_0=55\;mile/h$ is optimization speed.

Now we know $\theta$, let's calculate the maximal safe speed. Centripetal force is produced
by projections of force of reaction and force of friction $F_c=mv^2/R=N\sin\theta +F_f\cos\theta$. Since
we are interested in the maximal speed we'll use maximal force of friction $F_f=\mu N$.
Thus we get $mv^2/R=N\sin\theta +\mu N\cos\theta$.  Taking into account that now
$mg=N \cos\theta-F_f \sin\theta $
we get  for maximal safe speed 
$$v^2=gR\frac{v_0^2 +\mu g R}{gR-\mu v_0^2}\approx 46.4\;m/s\approx 104.5\;mile/h.$$
 









\end{document}