\documentclass[12pt]{article}
\usepackage{url, graphicx}

% page layout
\setlength{\topmargin}{-0.25in}
\setlength{\textheight}{9.5in}
\setlength{\headheight}{0in}
\setlength{\headsep}{0in}

% problem formatting
\newcommand{\problemname}{Problem}
\newcounter{problem}

% math
\newcommand{\dd}{\mathrm{d}}

% primary units
\newcommand{\rad}{\mathrm{rad}}
\newcommand{\kg}{\mathrm{kg}}
\newcommand{\m}{\mathrm{m}}
\newcommand{\s}{\mathrm{s}}

% secondary units
\renewcommand{\deg}{\mathrm{deg}}
\newcommand{\km}{\mathrm{km}}
\newcommand{\mi}{\mathrm{mi}}
\newcommand{\h}{\mathrm{h}}
\newcommand{\ns}{\mathrm{ns}}
\newcommand{\J}{\mathrm{J}}
\newcommand{\eV}{\mathrm{eV}}
\newcommand{\W}{\mathrm{W}}

% derived units
\newcommand{\mps}{\m\,\s^{-1}}
\newcommand{\mph}{\mi\,\h^{-1}}
\newcommand{\mpss}{\m\,\s^{-2}}

% random stuff
\sloppy\sloppypar\raggedbottom\frenchspacing\thispagestyle{empty}

\begin{document}

\section*{NYU Physics I---Problem Set 1}

Due Tuesday 2018 September 11 at the beginning of lecture.

\paragraph{Problem~\theproblem:}\refstepcounter{problem}%
If a car gets 27 miles to the gallon, how many liters of fuel does it
take to go $100\,\km$? (In Europe, fuel efficiencies are given in the
latter terms.)

\paragraph{Problem~\theproblem:}\refstepcounter{problem}%
\textsl{(a)}~What is the maximum amount of cash you can obtain by
successfully robbing an armored truck?  Assume that it is packed with
twenties; that is, estimate an answer by considering the volume of the
truck, and (harder to estimate) the volume of a 20-dollar bill.  State
your assumptions and explain your work, but please don't attempt an
experiment.  Be sure to explain exactly how you estimated the volume
of a 20-dollar bill.  \emph{Hint: think of a stack of bills to
  estimate the volume.  Feel free to \emph{check} any part of your
  answer on the internet, but make sure you actually make a justified,
  quantitative estimate independently.}

\textsl{(b)}~Do you think that many of the armored trucks in Manhattan
are fully packed with 20-dollar bills?

\textsl{(c)}~Would a similar truck weigh more, less, or about the same
if it contained the same amount of money but in the form of gold bars
instead of 20-dollar bills?

\paragraph{Problem~\theproblem:}\refstepcounter{problem}%
Imagine you have a mass $M$, a length $h$, a velocity $v$, and an
acceleration $g$. What combinations of these can you make that will
have units of
\textsl{(a)}~time,
\textsl{(b)}~force, and
\textsl{(c)}~energy?
Don't try to be exhaustive; just try to get two different expressions
for each!

\paragraph{Problem~\theproblem:}\refstepcounter{problem}%
A bit on air resistance and terminal velocities. If you want more
discussion of these issues, see \url{http://arxiv.org/abs/0709.0107}
(click the PDF link at the right of the page).

\textsl{(a)}~Show that the ram-pressure-force formula $\rho\,A\,v^2$ has the
correct units to be a force, when $\rho$ is a mass density, $A$ is a
cross-sectional area, and $v$ is a speed. Look up air drag on
Wikipedia and see what this formula is missing.

\textsl{(b)}~At what downward falling speed $v$ does an object with
mass $M$ and cross-sectional area $A$ find that ram pressure balances
the gravitational force? Derive an expression. This is the formula
(ish) for the terminal velocity!

\textsl{(c)}~Two pennies are dropped (carefully) from a tall building,
one so that it falls precisely edge-on, and one so that it falls
precisely face-on (difficult but not impossible in practice).  What
(roughly) are the two terminal velocities and what (roughly) is
their ratio $v_{\mathrm{edge}}/v_{\mathrm{face}}$?

\textsl{(d)}~Roughly how far does a typical American car have to drive
to ``sweep up'' or drive through a column of air that is comparable in
weight to the car itself?  You will have to estimate the
cross-sectional area and weight of a typical car (or look both things
up on the web; if you look them up, give the make and model).

\paragraph{Extra Problem (will not be graded for credit):}%
Describe in words the \emph{environmental significance} of the
distance you calculated in part \textsl{(d)} of the previous problem.

\end{document}
