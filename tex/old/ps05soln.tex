\documentclass[12pt]{article}
\usepackage{amssymb}
\usepackage{amsfonts}
\usepackage{epsfig,latexsym}
\voffset 0cm
\hoffset -1.5cm
\textheight 20cm
\textwidth 16cm
\def\dspace{\baselineskip = .30in}

\def\beq{\begin{equation}}
\def\eeq{\end{equation}}
\def\be{\begin{eqnarray}}
\def\ee{\end{eqnarray}}

\begin{document}
\begin{center}
{\bf\large Homework 5.}
\end{center}

{\bf Problem 1.}

Since the string is continuous and the pulleys are frictionless, the tension is the same
all over the string.  If we'll assume that acceleration of mass $m_2$ is $a$ down than
acceleration of $m_1$ is $a/2$ up (move $m_2$ by $\Delta x_2$ down, then $m_1$ will
move by $\Delta x_1=-\Delta x_2/2$, minus stays for up). Thus we can write equation of motion
for both masses
$$
m_1 a/2=2 T -m_1 g,\;\;\;\;\;\; m_2 a= m_2 g - T.
$$
We get for accelerations and tension:
$$
a_2=a=\frac{2m_2 -m_1}{2m_2+m_1/2}g,\;\;\;a_1=-\frac{a}{2},\;\;\;T=\frac{3}{2}\frac{m_1m_2}{2m_2+m_1/2}g.$$
(Check the limits for consistency $m_1\rightarrow 0,\;\;m_2\rightarrow 0,\;\; m_1\rightarrow m_2$.)
\\

{\bf Problem 2.}

$1\;gal\approx3.5\;liters$, since gas is a bit less dense than water, the mass of $1\;gal$ of gas is
about $3\;kg$. Atomic weight of gas is about $100$ it means that Avogadro's number
$N_A\approx 6\times 10^{23}$ of
gas molecules make $100\;gram$. Thus, there are $N=30 N_A$ molecules in $3\;kg$ of gas. 
If the burning energy of a single molecule is $5\times 10^{-19}\;J$ than $3\;kg$ of gas produce
energy $E\approx 30\times6\times 10^{23}\times 5\times 10^{-19}\;J\approx 10^{7}\;J$, which is enough to make $25\; miles\approx 4\times 10^4\;km$ at speed 55 miles/h. 

The work done by friction forces equals to useful work done by the engine $W=F_f l=.25 E$,
where $l$ is the path of the car. Thus, the total "friction" force  acting on the car at $55\;miles/h$
is about $F_f\approx 50\;N$.

Power is work done in a unit of time $P=.25 E/\Delta t$, where $\Delta t$ is time in which car makes
$25\;miles$. $P\approx .25 E/25\;min\approx 1.7\times 10^3\; J/s=1.7\;kWatt\approx2.5\;hp$ it is about a percent of power for typical car, which is pretty reasonable.
\\

{\bf Problem 3.}

If $m_2$ falls by a distance $\Delta h$,  $m_1$ would rise by $\Delta h/2$.
Consequently, the change of potential energy is $\Delta U=m_1g\Delta h/2 -m_2g\Delta h=(m_1/2-m_2)g\Delta h$. If $m_1/m_2=2$ than $\Delta U=0$, so kinetic energy should not change as well.
The acceleration for this ratio is zero.











\end{document}