\documentclass[12pt]{article}
\begin{document}
\newcounter{problem}
\thispagestyle{empty}

\section*{NYU Physics 1---Problem set 0}

Due Tuesday 2009 September 15 at the beginning of lecture.

\paragraph{Problem~\theproblem:}\refstepcounter{problem}%
\textsl{(a)}~What is the maximum amount of cash you can obtain by
successfully robbing an armored truck?  Assume that it is packed with
twenties; that is, estimate an answer by considering the volume of the
truck, and (harder to estimate) the volume of a 20-dollar bill.  State
your assumptions and explain your work, but please don't attempt an
experiment.  Be sure to explain exactly how you estimated the volume
of a 20-dollar bill.  \emph{Hint: think of a stack of bills to
  estimate the volume.  Feel free to \emph{check} any part of your
  answer on the internet, but make sure you actually make a justified,
  quantitative estimate independently.}

\textsl{(b)}~Do you think that many of the armored trucks in Manhattan
are fully packed with 20-dollar bills?

\textsl{(c)}~Would a similar truck weigh more, less, or about the same
if it contained the same amount of money but in the form of gold bars
instead of 20-dollar bills?

\paragraph{Problem~\theproblem:}\refstepcounter{problem}%
After having read the dropped bucket write-up, answer the following
approximate questions about air resistance:

\textsl{(a)}~Two pennies are dropped (carefully) from a tall building,
one so that it falls precisely edge-on, and one so that it falls
precisely face-on (difficult but not impossible in practice).  What is
the ratio of terminal velocities
$v_{\mathrm{edge}}/v_{\mathrm{face}}$?

\textsl{(b)}~If the building is 50 stories tall, what is the ratio
$v_{\mathrm{ff}}/v_{\mathrm{edge}}$ of the free-fall velocity
$v_{\mathrm{ff}}$ you would naively calculate if you ignored air
resistance to the terminal velocity of the penny dropped edge-on?

\textsl{(c)}~Roughly how far does a typical American car have to drive
to ``sweep up'' or drive through a column of air that is comparable in
weight to the car itself?  You will have to estimate the
cross-sectional area and weight of a typical car (or look both things
up on the web; if you look them up, give the make and model).

\paragraph{Extra Problem (will not be graded for credit):}%
Describe in words the \emph{environmental significance} of the
distance you calculated in the second part of the previous problem.

\end{document}
