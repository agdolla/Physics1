\documentclass[12pt]{article}
\usepackage{graphics}
\begin{document}

\section*{NYU Physics 1---In-class Exam 3}

\vfill

\paragraph{Name:} ~

\paragraph{email:} ~

\paragraph{recitation:} ~

\vfill

This exam consists of two problems.  Write only in this booklet.  Be
sure to show your work.

\vfill ~

\clearpage

\section*{Problem 1}

\noindent~\hfill\includegraphics{../mp/rollinghill.eps}\hfill~

A thin hoop of mass $M$ and radius $R$ (with moment of inertia
$I=M\,R^2$) rolls without slipping along the hilly landscape shown.
If it was released from rest at point A, what is its speed $v$ at
point B?  Give your answer in terms of $h$, $g$, $M$, and $R$.  Recall
that the kinetic energy $K$ of a translating, rotating object is
$K=(1/2)\,M\,v^2+(1/2)\,I\,\omega^2$.

\clearpage

\section*{Problem 2}

A figure skater spins in place on frictionless ice at angular speed
$\omega_i$ with her hands outstretched.  She has a total moment of
inertia $I_i$.  As the skater draws her hands into her body, her
moment of inertia decreases to $I_f=I_i/2$.  Does her kinetic energy
$K$ increase, decrease, or stay the same?  If it increases, where does
the energy come from?  If it decreases, where does the energy go to?
\emph{Explain all your answers concisely but clearly.}

\clearpage

[This page intentionally left blank for calculations or other work.]

\end{document}
