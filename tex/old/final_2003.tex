\documentclass[12pt]{article}
\usepackage{graphics}
\begin{document}

\section*{NYU Physics 1---Final Exam}

\vfill

\paragraph{Name:} ~

\paragraph{email:} ~

\vfill

This exam consists of six problems.  Write only in this booklet.  Be
sure to show your work.

\vfill

\emph{You must ensure that the proctor checks off your name on the
attendance sheet when you hand in your exam.  Your exam will
\textbf{not} be graded if your name is not checked off.}

\clearpage

\section*{Problem 1}

(a) A block of mass $M$ slides down a plane inclined at an angle
$\theta$ to the horizontal.  There is a coefficient of sliding
friction $\mu$ between block and plane.  What is the magnitude of the
acceleration $a$ of the block down the plane?  Be sure to draw a
free-body diagram.

\vfill

(b) A \emph{different} block of mass $M$ sits on the same plane.  In
this case, the frictional force is so great that the block does not
slide down the plane.  What is the magnitude $f$ and direction of the
frictional force?  Try to give your answer in terms of $M$, $g$ and
$\theta$ only.

\vfill ~

\clearpage

\section*{Problem 2}

A major-league baseball pitcher can throw a baseball at about
$100~\mathrm{mi\,hr^{-1}}$.  Estimate the impulse $\Delta p$ given to
the ball by the pitcher.  Make an estimate of the work done $\Delta W$
by the pitcher on the ball during the throw.  Now figure out some way
to compute the average force $F$ applied by the pitcher to the ball.
Does your answer seem reasonable?  \emph{Clearly state your
assumptions, and justify any that are non-trivial.}

\clearpage

\section*{Problem 3}

Draw free-body diagrams for all the masses and pulleys in this
mechanism.  Find the tensions in all three strings and the
accelerations of the two blocks.
\\ \rule{0.35\textwidth}{0pt}
\resizebox{0.3\textwidth}{!}{\includegraphics{../mp/tackle_blocks.eps}}
\\
Be careful with your kinematic constraints, and assume that all the
strings and pulleys are massless and frictionless.

\clearpage

\section*{Problem 4}

A figure skater spins in place on frictionless ice at angular speed
$\omega_i$ with her hands outstretched.  She has a total moment of
inertia $I_i$.  As the skater draws her hands into her body, her
moment of inertia decreases to $I_f=I_i/2$.  Does her kinetic energy
$K$ increase, decrease, or stay the same?  \emph{Explain your answer
concisely but clearly with either one sentence or a short
calculation.}

\clearpage

\section*{Problem 5}

A ``seconds'' pendulum has a period of exactly $2.000~\mathrm{s}$.

\textsl{(a)}~If $g=9.797~\mathrm{m\,s^{-2}}$ in Austin Texas, how long
should a Texan make her or his seconds pendulum?

\vfill

\textsl{(b)}~If $g$ is 0.1~percent (a factor of 1.001) larger in
Paris, France, than in Austin, how different, in percent, should the
length of a Parisian's seconds pendulum be than a Texan's?  Should it
be longer or shorter?  \emph{Note:} You can answer this question
without getting part (a) correct.

\vfill ~

\clearpage

\section*{Problem 6}

Consider the following attractive radial force law, which is
\emph{very different} from Newton's law of gravity:
\begin{equation}
F = \frac{k\,M\,m}{r^4}
\end{equation}
Where $k$ is a constant.

Consider a body of mass $m$ orbiting in a circular orbit of radius $R$
around another body of mass $M$, with $m\ll M$, according to the above
force law.

\textsl{(a)}~One of Kepler's laws is that for gravity, orbital period
$T$ is related to orbital radius $r$ by $T\propto r^{3/2}$.  Use
either a dimensional argument or a direct calculation to get the
equivalent relation for this \emph{very different} radial force law.

\vfill

\textsl{(b)}~What is the potential energy of the mass $m$ at radius
$r=R$, relative to its potential energy at infinite $r$?  \emph{Note:}
You can answer this question without getting part (a) correct.

\vfill ~

\clearpage

[This page intentionally left blank for calculations or other work.]

\end{document}
