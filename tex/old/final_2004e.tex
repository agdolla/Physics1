\documentclass[12pt]{article}
\usepackage{graphics}
\begin{document}
\setcounter{page}{0}
\thispagestyle{empty}

\section*{NYU Engineering Physics 1---Final Exam}

\vfill

\paragraph{Name:} ~

\paragraph{email:} ~

\vfill

This exam consists of seven problems.  Write only in this booklet.  Be
sure to show your work.

\vfill

\emph{You must ensure that the proctor checks off your name on the
attendance sheet when you hand in your exam.  Your exam will
\textbf{not} be graded if your name is not checked off.}

\clearpage

\section*{Problem 1}

(a) A block of mass $M$ slides down a plane inclined at an angle
$\theta$ to the horizontal.  There is a coefficient of sliding
friction $\mu$ between block and plane.  What is the magnitude of the
acceleration $a$ of the block down the plane?  \emph{Be sure to draw a
free-body diagram.}

\vfill

(b) A \emph{different} block of mass $M$ sits on the same plane.  In
this case, the frictional force is so great that the block does not
slide down the plane.  What is the magnitude $f$ and direction of the
frictional force?  Try to give your answer in terms of $M$, $g$ and
$\theta$ only.

\vfill ~

\clearpage

\section*{Problem 2}

A major-league baseball pitcher can throw a baseball at about
$90~\mathrm{mi\,hr^{-1}}$.  Estimate the impulse $\Delta p$ given to
the ball by the pitcher.  Make an estimate of the work done $\Delta W$
by the pitcher on the ball during the throw.  Use the work to figure
out the average force $F$ applied by the pitcher to the ball in SI
units (Newtons).  Does your answer seem reasonable?  \emph{Clearly
state your assumptions, and justify any that are non-trivial.}

\clearpage

\section*{Problem 3}

Draw free-body diagrams for all the masses and pulleys in this
mechanism.  Find the tensions in all three strings and the
accelerations of the two blocks.
\\ \rule{0.35\textwidth}{0pt}
\resizebox{0.3\textwidth}{!}{\includegraphics{../mp/tackle_blocks.eps}}
\\
Be careful with your kinematic constraints (\textit{ie,} the
relationship between the two accelerations), and assume that all the
strings and pulleys are massless and frictionless.

\clearpage

\section*{Problem 4}

~\hfill\includegraphics{../mp/stick_n_force.eps}\hfill~\\ A stick of
mass $m$ and length $\ell$, initially at rest, experiences a force $F$
at one end, as shown, for a \emph{very short} time $\Delta t$.  This
impulse causes the stick to move (translate) and to rotate.  At what
angular speed $\omega$ does the stick end up rotating around its
center of mass?  Its moment of inertia is $(1/12)\,m\,\ell^2$.
\emph{Assume this is happening in outer space; there is no gravity or
friction.}  Use the center of mass as the reference point or axis for
your calculation.

\clearpage

\section*{Problem 5}

A ``seconds'' pendulum has a period of exactly $2.000~\mathrm{s}$.

\textsl{(a)}~If $g=9.797~\mathrm{m\,s^{-2}}$ in Austin Texas, how long
should a Texan make her or his seconds pendulum?

\vfill

\textsl{(b)}~If $g$ is 0.1~percent (a factor of 1.001) larger in
Paris, France, than in Austin, how different, in percent, should the
length of a Parisian's seconds pendulum be than a Texan's?  Should it
be longer or shorter?  \emph{Note:} You can answer this question
without getting part (a) correct.

\vfill ~

\clearpage

\section*{Problem 6}

At what radius $R$ from the center of the Earth (mass $M$) does a
geostationary (one orbit every $T=24$~h) satellite (mass $m$) orbit on
a uniform circular orbit?  Solve this problem in three \emph{easy}
steps.

\emph{Hint:} Every step in your calculation should have correct
dimensions!

\textsl{(a)}~Write a symbolic expression for the acceleration $a$ of
   an object undergoing uniform circular motion with radius $R$ and
   period $T$.

\vfill

\textsl{(b)}~Use Newton's law of gravity ($F=G\,M\,m/r^2$) to write a
   symbolic expression for the acceleration due to gravity of an
   object with mass $m$ at radius $R$.

\vfill

\textsl{(c)}~Use the fact that $g=10~\mathrm{m\,s^{-2}}$ at the
   surface of the Earth ($R_E=6000$~km) to eliminate the quantity
   $G\,M$ and solve for the radius of the orbit $R$.  Give your answer
   in km.

\vfill ~

\clearpage

\section*{Problem 7}

Compute---approximately---the terminal velocity of a penny dropped
from the Empire State Building as follows.

\textsl{(a)}~The magnitude of the drag force $F_d$ depends on the
density of air $\rho$, speed of the penny $v$, and cross-sectional
area of the penny $A$.  What is the \emph{only} combination of these
quantities that has dimensions of force?  Give a symbolic answer.

\vfill

\textsl{(b)}~Estimate the mass of the penny (it's made of metal), the
area of the penny (assuming it falls face-on), the density of air,
etc., and find the (approximate) ``terminal'' falling speed
$v_\mathrm{term}$ at which the drag force cancels the gravitational
force.  Give a numerical answer in m\,s$^{-1}$.

\vfill

\textsl{(c)}~How much faster is the terminal speed when the penny
falls edge-on rather than face-on?

\vfill ~

\clearpage

\resizebox{\textwidth}{!}{\includegraphics{formulae.eps}}

\clearpage

[This page intentionally left blank for calculations or other work.]

\end{document}
