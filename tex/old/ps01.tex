\documentclass[12pt]{article}
\newcommand{\s}{\mathrm{s}}
\begin{document}
\newcounter{problem}
\thispagestyle{empty}

\section*{NYU Physics 1---Problem set 1}

Due Tuesday 2009 September 22 at the beginning of lecture.

\paragraph{Problem~\theproblem:}\refstepcounter{problem}%
Chabay \& Sherwood, problem 1.HW.101

\paragraph{Problem~\theproblem:}\refstepcounter{problem}%
Re-do the numerical integration worksheet problem from recitation this
week (download it from the course web site if you have forgotten), but
this time do it with a computer spreadsheet and use a time interval of
0.01 sec.  No need to hand in the whole spreadsheet, or the answers to
all the questions, but hand in a graph (plotted by your spreadsheet
program) of the position as a function of time for the interval
$0<t<2\,\s$.  Make sure your axes are clearly labeled and
``calibrated'' in units of m and s.

\paragraph{Problem~\theproblem:}\refstepcounter{problem}%
Look up the value and units of Newton's constant $G$.
\textsl{(a)}~Find some combination of $G$, the speed of light $c$, and
some (arbitrary) mass $M$, that has units of \emph{length}.
\textsl{(b)}~Evaluate your expression for the mass of the Sun (you
will have to look it up); that is, find the characteristic length
associated with that mass.  Give your answer in units of km.  What is
the physical meaning of this length (approximately)?

\end{document}
