\documentclass[12pt]{article}
\usepackage{amssymb}
\usepackage{amsfonts}
\voffset -2cm
\hoffset -1.5cm
\textheight 20cm
\textwidth 16cm
\def\dspace{\baselineskip = .30in}

\def\beq{\begin{equation}}
\def\eeq{\end{equation}}
\def\be{\begin{eqnarray}}
\def\ee{\end{eqnarray}}


\begin{document}
{\bf Problem 1.}

{\bf (a)}  Size of an armored truck is approximately 1.5 m height (a guy should bend to get in)
3 m length and 2 m  wide. Thus, the volume is $V_{truck} \approx 9 \; m^3$.

Now, let's estimate  the volume \$20 bills. The 100 bills stack is about 2 cm thick, and the area
size of a bill is about 5 cm by 15 cm. So, 100 of \$20 bills will make \$2000 and the volume of a stack is about $V_{stack}=2\times5\times15 \; cm^3 =150 \; cm^3 =150 \times10^{-6} m^3$.

Thus, we can fit in a truck about $V_{truck}/V_{stack}=( 9 \; m^3)/(150 \times10^{-6} m^3)=6 
\times 10^4$ stacks or in dollar value about  $10^8$.
The maximal possible gain is about 100 million dollars. 

{\bf (b)} 1 troy oz is about 30 grams and costs about \$350. For our estimate, we can approximate that 1 gram of gold is about \$10. Thus, to make $10^8$ dollars we need $10^7 \; g=10^4 \;kg$ of gold (pretty heavy!!!).  As we remember from part (a) there were $6\times 10^4$ stacks in the truck,
so the question reduces to the following: which weights more 6 stacks of \$20 or 1 kg?
It seems that 1 stack is less than 100 grams, so gold would weight about twice more
(but this depends on humidity, isn't it? ;).
\\

{\bf Problem 2.}

Constant acceleration means $x=x_{0} + v_{0} t + a t^2 /2$. The dragster starts at $x_{0}=0$
with zero initial velocity $v_{0}=0$, thus we can find acceleration $a= 2 x /t^2$.
The distance is  $x=$1/4 mile$=402\; m$, time is $t=5.5 \;s$, so $a=26.6\; m/s^2=2.7 \;g$, 
where $g=9.8\;m/s^2$ is gravitational acceleration. Final speed is 
$v=a t =26.6\; m/s^2 \times 5.5\; s=146\;m/s=330\:miles/hour$.  The figures in the problem actually a bit higher then ones you can find on the web for speeds and times.
\\

{\bf Problem 3.}

{\bf (a)}  $v_0=5\;m/s$, $g=10\;m/s$ thus $x=v_0 t -g t^2 /2=0.55 \;m$ after $0.1\;s$
If we'll approximate with constant velocity one gets $x=v_0 t =0.5 \;m$ which is pretty close
for "small times".

{\bf (b)} The new velocity after $0.1\;s$ is $v_1=v_0-gt=4\;m/s$. Correspondingly in next
$0.1s$ the stone will travel about $0.4\;m$.

{\bf (c)}\\
\begin{tabular}{|c|c|c||c|c|c|}
\hline
time s & velocity  m/s & $\Delta x$ m &time s & velocity  m/s &$\Delta x$ m \\
\hline
0.1& 5&0.5&0.9& -3&-0.3\\
0.2& 4&0.4&1.0& -4&-0.4\\
0.3& 3&0.3&1.1& -5&-0.5\\
0.4& 2&0.2&1.2& -6&-0.6\\
0.5& 1&0.1&1.3& -7&-0.7\\
0.6& 0&0.0&1.4& -8&-0.8\\
0.7& -1&-0.1&1.5& -9&-0.9\\
0.8& -2&-0.2 &&&total $\Delta x=-3$\\
\hline
\end{tabular}

{\bf (d)} The stone returnes to $x=0$ after $1.1\;s$, the exact calculation gives
$x=v_0 t - g t^2/2=0$, from here $t=2 v_0/g =1\;s$. Thus, for our approximate 
calculation we got 10 percent error due to averaging. 
\\

{\bf Problem 4.}

Weight one can lift is proportional to the force one can exert and proportional to the linear size squared (because muscles are mostly at the surface of a body their strength is proportional to area of a body)  $W\sim F\sim l^2$.  Now, if you change linear size $l$ by 1 percent, that is
 $l_1=1.01 l$. Thus the weight which can be lifted by the bigger twin is $W_1\sim l_1^2$, or
 since the proportionality coefficient should be the same $W_1=1.01^2 W\approx 1.02 W$. The bigger one can lift 2 percent more. 
 
Number of its own weight $W_0$ one can lift is $N=W/W_0$. The weight is proportional to 
the volume of the body or to the linear size cubed $W_0\sim V\sim l^3$. That's why for
the number of own weight one can lift we get $N\sim l^{-1}$. Which mean, the smaller
size the more own weights one can lift.

Let's look at example: average human can lift 1 its own weight. If we'll reduce him 100 times
(to the size of an ant) it would be able to lift 100 its own weights. On the other hand, an ant
can lift only about 20 its own weights.Thus, not ants are strong but humans are big ;).

 



\end{document}