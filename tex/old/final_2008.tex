\documentclass[12pt]{article}
\usepackage{graphics}
\newcommand{\kg}{\mathrm{kg}}
\newcommand{\m}{\mathrm{m}}
\newcommand{\s}{\mathrm{s}}
\renewcommand{\deg}{\mathrm{deg}}
\newcommand{\cm}{\mathrm{cm}}
\newcommand{\mps}{\m\,\s^{-1}}
\newcommand{\mpss}{\m\,\s^{-2}}
\newcommand{\kgpmmm}{\kg\,\m^{-3}}
\newcommand{\N}{\mathrm{N}}
\newcommand{\J}{\mathrm{J}}
\newcommand{\Npmm}{\N\,\m^{-2}}
\newcommand{\tv}[1]{\mathbf{\vec{#1}}}
\newcommand{\dd}{\mathrm{d}}
\newcommand{\cell}[1]{\texttt{{#1}}}
\newcounter{problem}
\addtolength{\oddsidemargin}{-1in}
\addtolength{\textheight}{\headheight}
\setlength{\headheight}{0in}
\addtolength{\textheight}{\headsep}
\setlength{\headsep}{0in}
\setlength{\marginparwidth}{2in}
\begin{document}

\section*{NYU Physics 1---final exam}

Wednesday 2008 December 17

~ \vfill ~

\section*{Name:}

~ \vfill ~

\begin{equation}
\tv{v} = \frac{\dd}{\dd t}\,\tv{r}
\quad ; \quad
\tv{a} = \frac{\dd}{\dd t}\,\tv{v}
\end{equation}
\begin{equation}
\omega = \frac{\dd}{\dd t}\,\theta
\quad ; \quad
\alpha = \frac{\dd}{\dd t}\,\omega
\end{equation}
\begin{equation}
a_{\mathrm{circ}} = \frac{v^2}{R} = \omega^2\,R
\end{equation}
\begin{equation}
\tv{F} = m\,\tv{a}
       = \frac{\dd}{\dd t}\,\tv{p}
       = \frac{\dd}{\dd t}\,(\gamma\,m\,\tv{v})
\end{equation}
\begin{equation}
\gamma = \frac{1}{\sqrt{1 - \frac{v^2}{c^2}}}
       = \frac{1}{\sqrt{1 - \beta^2}}
\end{equation}
\begin{equation}
\tv{r}\times\tv{F} = \tv{\tau}
                   = \frac{\dd}{\dd t}\,\tv{L}
		   = \frac{\dd}{\dd t}\,(\tv{r}\times\tv{p})
\end{equation}
\begin{equation}
\tv{r}\times\tv{F} = \tv{\tau}
                   = I\,\tv{\alpha}
                   = \frac{\dd}{\dd t}\,\tv{L}
                   = \frac{\dd}{\dd t}\,(I\,\tv{\omega})
\end{equation}
\begin{equation}
|\tv{F}_{\mathrm{grav}}| = \frac{G\,M\,m}{r^2} = m\,g
\end{equation}
\begin{equation}
|\tv{F}_{\mathrm{static}}| \leq \mu\,|\tv{N}|
\quad ; \quad
|\tv{F}_{\mathrm{kinetic}}| = \mu\,|\tv{N}|
\end{equation}
\begin{equation}
U = -\frac{G\,M\,m}{r}
\quad ; \quad
\Delta U = m\,g\,\Delta h
\quad ; \quad
U = \frac{1}{2}\,k\,x^2
\end{equation}
\begin{equation}
K = \frac{1}{2}\,m\,v^2
\quad ; \quad
K = \frac{1}{2}\,I\,\omega^2
\end{equation}

\clearpage

\paragraph{Problem~\theproblem}\refstepcounter{problem}%
What combination of $G$, the speed of light $c$, and some (arbitrary)
mass $M$ has units of \emph{length}?  Use the formulae on the formula
sheet to infer or remind yourself of the units of $G$.

~ \vfill ~

\paragraph{Problem~\theproblem}\refstepcounter{problem}%
What is the mean acceleration $a$ of a dragster that can travel 1/4~mi
in 5.5~s, starting from a dead stop?  Assume that the dragster
accelerates with constant acceleration throughout the 5.5~s (not a
terrible assumption, but not a good one either).  Give your answer in
terms of the gravitational acceleration $g$.

~ \vfill ~

\clearpage

\paragraph{Problem~\theproblem}\refstepcounter{problem}%
A block of mass $m$ sits on a plane inclined at $\theta=20\,\deg$ to
the horizontal.  Between the block and the plane is a coefficient of
friction $\mu=0.9$.  What is the magnitude of the frictional force on
the block?

~ \vfill ~

\paragraph{Problem~\theproblem}\refstepcounter{problem}%
Roughly how many carbon atoms are there in a gallon of gasoline?
State your assumptions clearly.  Recall that the atomic weight of
carbon is 12.

~ \vfill ~

\clearpage

\paragraph{Problem~\theproblem}\refstepcounter{problem}%
A Japanese Shinkansen (bullet train) (of mass $M=10^6~\mathrm{kg}$),
moving at $v_S=300~\mathrm{km\,h^{-1}}$ with respect to the tracks,
hits and collides elastically with a superball (of mass
$m=30~\mathrm{g}$), which is initially at rest.  The front face of the
train is inclined at an angle of $\theta=45~\mathrm{deg}$ to the
horizontal, as shown here in the rest frame of the tracks.\\
\rule{0.1\textwidth}{0pt}
\resizebox{0.8\textwidth}{!}{\includegraphics{../mp/shinkansen.eps}}

\textsl{(a)}~Draw a diagram of the ball--train system, just before the
collision, in the center-of-mass rest frame.  Clearly show the
velocities of the ball and train in this frame.

\textsl{(b)}~Draw a diagram, just after the collision, in the
center-of-mass frame.  Clearly show the velocities.

~ \vfill ~

\paragraph{Problem~\theproblem}\refstepcounter{problem}%
Immediately after being hit, at $t=0$, a cue ball of mass $M$, radius
$R$, and moment of inertia $I=(2/5)\,M\,R^2$ slides along the felt at
speed $v_i$, not rotating at all.  As time goes on, the ball slows
down (because of friction) and, at the same time, starts to spin.
Draw a free-body diagram for the cue ball.  At what time
$t_\mathrm{r}$ does the ball get to the situation of ``rolling without
slipping''?  Assume that there is a coefficient $\mu$ of sliding
friction.

~ \vfill ~

\clearpage

\paragraph{Problem~\theproblem}\refstepcounter{problem}%
A figure skater spins in place on frictionless ice at angular speed
$\omega_i$ with her hands outstretched.  She has a total moment of
inertia $I_i$.  As the skater draws her hands into her body, her
moment of inertia decreases to $I_f=I_i/2$.  Does her kinetic energy
$K$ increase, decrease, or stay the same?  If it increases, where does
the energy come from?  If it decreases, where does the energy go to?
Make sure your explanation is \emph{30 words or less}.

~ \vfill ~

\paragraph{Problem~\theproblem}\refstepcounter{problem}%
Draw a spacetime diagram showing the four events
$A=(c\,t_A,x_A)=0\,\m,0\,\m)$, $B=(1\,\m,1\,\m)$, $C=(1\,\m,0\,\m)$,
and $D=(0\,\m,1\,\m)$.  Clearly label the four events.

Transform these events to a new frame using the Lorentz
Transform
\begin{eqnarray}\displaystyle
c\,t' & = & \gamma\,c\,t - \beta\,\gamma\,   x \nonumber\\
   x' & = & \gamma\,   x - \beta\,\gamma\,c\,t \quad ,
\end{eqnarray}
with $\beta=(3/5)$.  Draw \emph{another} spacetime diagram of the
four events in the \emph{new} frame.

~ \vfill ~

\clearpage

\paragraph{Problem~\theproblem}\refstepcounter{problem}%
What is the magnitude of the 4-vector
$\vec{u}=(\gamma\,c,\gamma\,v_x,\gamma\,v_y,\gamma\,v_z)$?  Simplify
your answer as much as you can.

~ \vfill ~

\paragraph{Problem~\theproblem}\refstepcounter{problem}%
What speed can you make by combining acceleration $g$ and length $R$?
Evaluate your expression for the acceleration due to gravity at the
Earth's surface and the radius of the Earth.  Estimate the radius of
the Earth if you don't know it.

~ \vfill ~

\clearpage

\paragraph{Problem~\theproblem}\refstepcounter{problem}%
Explain, in \emph{less than 30 words} why the astronauts in the Space
Shuttle are (or feel) weightless.

~ \vfill ~

\paragraph{Problem~\theproblem}\refstepcounter{problem}%
Consider a particle subject to a potential energy of the form
\begin{equation}
U = \frac{B}{x^2} - \frac{A}{x} \quad ,
\label{eq:hard}
\end{equation}
where $A$ and $B$ are positive constants.  Sketch the potential
$U(x)$.

By taking derivatives, compute the equilibrium position
$x_\mathrm{eq}$.  That is, compute the $x$ position where there is no
force and the potential is at a minimum.

~ \vfill ~

\end{document}
