\documentclass[12pt]{article}
\usepackage{graphics}
\begin{document}

\section*{NYU Engineering Physics 1---In-class Exam 3}

\vfill

\paragraph{Name:} ~

\paragraph{email:} ~

\vfill

This exam consists of two problems.  Write only in this booklet.  Be
sure to show your work.

\vfill ~

\clearpage

\section*{Problem 1}

In class we showed that a circular object of mass $m$, radius $R$ and
moment of inertia $I$ rolls without slipping down an inclined plane
(angle $\theta$ to the horizontal) with acceleration
$$
a=\frac{g\,\sin\theta}{1+\frac{I}{m\,R^2}} \quad .
$$ What is the force of friction $\vec{f}$ (magnitude and direction)
acting on the rolling object?

\vfill

Now compare a thin hoop with a uniform disk, both of the same mass and
radius, rolling down the inclined plane.  Which is subject to the
greater frictional force?  \emph{Note that you do not have to know the
exact moment of inertia expressions---nor do you have to get the first
part correct---in order to get this part correct.}

\vfill ~

\clearpage

\section*{Problem 2}

~\hfill\includegraphics{../mp/stick_n_force.eps}\hfill~\\ A stick of
mass $m$ and length $\ell$, initially at rest, experiences a force $F$
at one end, as shown, for a \emph{very short} time $\Delta t$.  This
impulse causes the stick to move (translate) and to rotate.  At what
angular speed $\omega$ does the stick end up rotating around its
center of mass?  Its moment of inertia is $(1/12)\,m\,\ell^2$.
\emph{Assume this is happening in outer space; there is no gravity or
friction.}

\clearpage

\centerline{\textsl{[this page intentionally left blank]}}

\end{document}
