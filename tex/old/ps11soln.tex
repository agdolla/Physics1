\documentclass[12pt]{article}
\usepackage{amssymb}
\usepackage{amsfonts}
\usepackage{epsfig,latexsym}
\voffset -.5cm
\hoffset -1.5cm
\textheight 21cm
\textwidth 16cm
\def\dspace{\baselineskip = .30in}

\def\beq{\begin{equation}}
\def\eeq{\end{equation}}
\def\be{\begin{eqnarray}}
\def\ee{\end{eqnarray}}

\begin{document}
\begin{center}
{\bf\large Homework 11.}
\end{center}

{\bf Problem 1.}

(a) $\tau=mg\sin\theta_i l\approx 9.6\;Nm$ at $t=0$.
At time $t=0.1s$ we use $\tau=d(\omega I)/dt$, from here $\omega\approx \tau\Delta t /I\approx
0.01\;s^{-1}$. Thus, for angle we have $\theta=\theta_i-\omega t\approx 0.009$.

(b)

\begin{tabular}{|c|c|c||c|c|c||c|c|c|}
\hline
time & $\omega$ &$\theta$& time& $\omega$ & $\theta$&time&$\omega$&$\theta$\\
\hline
0&0&0.01&0.7&0.0065&0.0073&1.4&0.00987&0.0012\\
0.1& 0.0010&0.0099&0.8& 0.0072&0.0066&1.5&0.00999&0.0002\\
0.2& 0.0020&0.0097&0.9& 0.0078&0.0058&1.6&0.01&-0.0008\\
0.3& 0.0030&0.0094&1.0& 0.0084&0.0050&1.7&0.0099&-0.0018\\
0.4& 0.0039&0.0090&1.1& 0.0089&0.0041&1.8&0.0097&-0.0028\\
0.5& 0.0048&0.0085&1.2& 0.0093&0.0032&1.9&0.0095&-0.0037\\
0.6& 0.0057&0.0080&1.3& 0.0096&0.0023&2.0&0.0091&-0.0046\\
\hline
\end{tabular}\\
\\

{\bf Problem 2.}

At $t=0$, $x=0.1\;m$ and $v_x=\dot{x}=-0.3\;m/s$.\\
\\
(1) $x(t)=A\cos\omega t + B\sin\omega t,\;\;\;\;\;\dot{x}(t)=-\omega A\sin\omega t + \omega B\cos\omega t$, thus for $t=0$ we get $x(0)=A,\;\;\;\;\;\dot{x}(0)= \omega B$. So, the parameters are
$A=x(0)=0.1\;m,\;\;\;\;B=\dot{x}(0)/\omega=\dot{x}(0) T/(2\pi)\approx -0.48\;m$.\\
\\
(2) $x(t)=x_0 \cos(\omega t +\phi),\;\;\;\;\dot{x}(t)=-\omega x_0 \sin(\omega t +\phi)$, thus for $t=0$
we get\\  $x(0)=x_0\cos\phi$ and $\dot{x}(0)=-\omega x_0 \sin\phi$. Solving with respect to $\phi$ and $x_0$ we find the parameters $\phi=\arctan\left(-x(0)\omega/\dot{x}(0)\right)\approx 12^o$ and
$x_0=\sqrt{x(0)^2+\left(\dot{x}(0)/\omega\right)^2}\approx 0.49\;m$.\\


{\bf Problem 3.}

$x(t)=A e^{-\gamma t/2}\cos\omega_1 t$, where $\omega_1=\sqrt{\omega^2-\gamma^2/4}$. The equation is 
$\ddot{x}+\gamma\dot{x}+\omega^2 x=0$.\\
$\dot{x}=-\gamma A/2 e^{-\gamma t/2}\cos\omega_1 t -\omega_1 A e^{-\gamma t/2}\sin\omega_1 t$,\\
$\ddot{x}=\gamma^2 A/4 e^{-\gamma t/2}\cos\omega_1 t +\gamma\omega_1 A e^{-\gamma t/2}\sin\omega_1 - \omega_1^2 A e^{-\gamma t/2}\cos\omega_1 t$,
now it is just the matter of trivial algebra to check out that $x(t)$ is the solution.
\\

{\bf Problem 4.}

Let's suppose that a child is displace by angle $\alpha$ from equilibrium. The force tangent to the  surface of the hemisphere  acting on the child is $F=mg\sin\alpha$. Thus equation of motion 
is $ma_t=mg\sin\alpha$, where $a_t=R \ddot{\alpha}$ is acceleration along hemisphere.
Thus our equation of motion takes form
$mR\ddot{\alpha}-mg\sin\alpha=0$, or we can write as $ \ddot{\alpha}-\frac{g}{R}\sin\alpha=0.$
Since we know that displacement is small or $\alpha\ll 1$ we can make an approximation
$\sin\alpha\approx \alpha+O(\alpha^3)$, than equation takes the form
$ \ddot{\alpha}-\frac{g}{R}\alpha=0$. The  solution is $\alpha=A \exp(-\sqrt{\frac{g}{R}}t)+B \exp(\sqrt{\frac{g}{R}}t)$,
one can check it substituting into the equation.




\end{document}