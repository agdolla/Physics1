\documentclass[12pt]{article}
\newcommand{\s}{\mathrm{s}}
\renewcommand{\deg}{\mathrm{deg}}
\newcommand{\m}{\mathrm{m}}
\newcounter{problem}
\begin{document}
\thispagestyle{empty}

\section*{NYU Physics 1---Problem set 4}

Due Tuesday 2009 October 13 at the beginning of lecture.

\paragraph{Problem~\theproblem}\refstepcounter{problem}%
\textsl{(a)}~In lecture we calculated the speed a car must go to get
around a banked circular turn at constant speed, in the absence of
friction.  Compute the value of this speed for a turn with radius
$30\,\m$ banked at $10\,\deg$ to the horizontal.

\textsl{(b)}~Now imagine that the car has a coefficient of static
friction for transverse forces (that is, opposing sliding up or down
the bank) of 0.2.  What is the maximum speed at which the car can go
around the turn?  At the maximum speed, the transverse frictional
force will be just 0.2 times the normal force.

\textsl{(c)}~In coming to your answer, make sure you draw a very clear
free-body diagram.

\paragraph{Problem~\theproblem}\refstepcounter{problem}%
\textsl{(a)}~Estimate the amount of energy (in J) in a gallon of
gasoline by imagining that there is about 4 eV of energy for every
carbon atom.  You will have to make assumptions, since there is no
simple molecular formula for gasoline!  Compare your answer to an
answer on the web.  If they differ substantially, explain why you
think that might be.  Be sure to use a reliable source on the web, or
obtain multiple answers, since web pages can be mistaken.

\textsl{(b)}~Now consider a kiloton of gasoline.  Does it have more or
less energy than a kiloton of TNT?  By what factor?

\textsl{(c)}~How much kinetic energy is there in a rock that is a
cube, $10\,\m$ on a side, moving at speed $\sqrt{g\,R}$, where $g$ is
the acceleration due to gravity and $R$ is the radius of the Earth?
Give your answer in units of kilotons of TNT equivalent.  Why do you
think I ask about that particular velocity?

\paragraph{Problem~\theproblem}\refstepcounter{problem}%
\textsl{(a)}~For a good golfer, a $150\,\m$ drive is easy.  If you
assume that air resistance does not matter, how much kinetic energy
$K$ does the ball have at the beginning of its flight?  (You will have
to look up the mass online.)  Assume it is hit at $45\,\deg$ to the
horizontal.

\textsl{(b)}~Now consider a realistic (though not exact) air
resistance force that is directly opposed to the direction of motion
and has magnitude
\begin{equation}
F_\mathrm{air} = \frac{1}{2}\,\rho\,A\,v^2 \quad ,
\end{equation}
where $\rho$ is the density of air, $A$ is the cross-sectional area of
the golf ball (you will have to look this up online) and $v$ is the
speed of the ball through the air.

Make a spreadsheet or computer program, with time in intervals of
$0.01\,\s$, integrating the trajectory with gravity plus this air
resistance force.  You will want columns or variables that are at
least $t$, $x$, $y$, $v_x$, $v_y$, $v$ (magnitude of velocity),
$F_\mathrm{air}$ (magnitude of air resistance force), $F_x$, and
$F_y$, where the last two are the $x$ and $y$ components of the net
force (gravity plus air resistance).  You will have to make sure you
deal with the direction of the air resistance force correctly.

Now adjust the initial speed until the shot goes approximately
$100\,\m$.  That is, increase the velocity at $t=0$ until $x$ is
approximately $150\,\m$ in the time interval at which $y$ goes
negative (it hits the ground).  Assume that the shot is still at
$45\,\deg$.

Hand your spreadsheet or table of internal variables.

\textsl{(c)}~Compute the initial kinetic energy for your computed
trajectory and compare it to the value you got in part~(a).

\textsl{(d)}~If you try to get your spreadsheet to make a shot that is
$250\,\m$, you will find it is virtually impossible!  Why?  Good
golfers can easily hit a ball $250\,\m$.  What do they do differently?

\end{document}
