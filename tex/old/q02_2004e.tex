\documentclass[12pt]{article}
\usepackage{graphics}
\begin{document}

\section*{NYU Engineering Physics 1---In-class Exam 2}

\vfill

\paragraph{Name:} ~

\paragraph{email:} ~

\vfill

This exam consists of three problems.  Write only in this booklet.  Be
sure to show your work.

\vfill ~

\clearpage

\section*{Problem 1}

In the diagram below, all surfaces are frictionless, and all strings
and pulleys are massless and frictionless.
\\
\resizebox{\textwidth}{!}{\includegraphics{../mp/pulley_ramp.eps}}

\textsl{(a)}~Draw free body diagrams for both blocks, showing all
forces acting.

\vfill

\textsl{(b)}~Find the acceleration $\vec{a}_2$ (magnitude and
direction) of block $m_2$, in terms of the labeled quantities and
anything else you need ($g$, for example).  Be sure to show clearly
the direction corresponding to positive acceleration for each block.

\vfill ~

\clearpage

\section*{Problem 2}

A major-league baseball pitcher can throw a baseball at about
$90~\mathrm{mi\,hr^{-1}}$.  Estimate the impulse $\Delta p$ given to
the ball by the pitcher.  Make an estimate of the work done $\Delta W$
by the pitcher on the ball during the throw.  Use the work to figure
out the average force $F$ applied by the pitcher to the ball in SI
units (newtons).  Does your answer seem reasonable?  \emph{Clearly
state your assumptions, and justify any that are non-trivial.}

\clearpage

\section*{Problem 3}

A physicist of mass $48~\mathrm{kg}$ stands (carefully) on a perfectly
frictionless ice rink, next to a block of ice of mass
$16~\mathrm{kg}$.  Both begin at rest in the ``rink frame.''  The
physicist pushes on the block until it is moving away \emph{relative
to the physicist} at $1~\mathrm{m\,s^{-1}}$.

When viewed in the rink frame, the block ends up moving in the
positive $x$ direction and the physicist ends up moving in the
negative $x$ direction, by conservation of momentum.  What are the
final speeds of the block and physicist, in the rink frame?

\end{document}
