\documentclass[12pt]{article}
\usepackage{graphics}
\newcommand{\mi}{\mathrm{mi}}
\newcommand{\m}{\mathrm{m}}
\newcommand{\kg}{\mathrm{kg}}
\newcommand{\rad}{\mathrm{rad}}
\newcommand{\s}{\mathrm{s}}
\newcounter{problem}
\begin{document}
\thispagestyle{empty}

\section*{NYU Physics 1---Problem set 5}

Due Tuesday 2009 October 20 at the beginning of lecture.

\paragraph{Problem~\theproblem:}\refstepcounter{problem}%
\textsl{(a)}~Draw free-body diagrams for all the masses and pulleys in
this mechanism.  Find the tensions in all three strings and the
accelerations of the two blocks.
\\ \rule{0.35\textwidth}{0pt}
\resizebox{0.3\textwidth}{!}{\includegraphics{../mp/tackle_blocks.eps}}
\\
Be careful with your kinematic constraints---the relationships among
the accelerations of the blocks---and treat all the strings and
pulleys as massless and frictionless.

\textsl{(b)}~Now set the mass $m_2$ to the value it must have if the
system is to be perfectly ``balanced''; that is, for there to be no
net acceleration of either block.  In this situation, if block $m_1$
is lowered a distance $h$, what is the net change in potential energy,
accounting for displacements of both blocks?

\paragraph{Problem~\theproblem:}\refstepcounter{problem}%
Consider a simple harmonic oscillator with a mass of $m=0.5\,\kg$, an
angular frequency of $\omega=3.9\,\rad\,\s^{-1}$, and oscillating with
an amplitude of $0.03\,\m$.  Make time-aligned plots of the position
$x$, velocity $v$, potential energy $U$, kinetic energy $K$, and total
energy $U+K$, all as a function of time for the time period
$0<t<3\,\s$.  Clearly label the period, amplitude, and maximum and
minimum values of these plots.  In fact, label anything on your five
plots that could be useful for understanding the system.

\paragraph{Problem~\theproblem:}\refstepcounter{problem}%
Determine the value of $\pi$ by integration!

\textsl{(a)}~Make a spreadsheet with a dimensionless column marked
``$t$'' that goes from 0.0 to 20.0 in units of 0.05.  Now make columns
marked ``$c$'' and ``$s$''.  Integrate the functions $c(t)$ and
$s(t)$ with the properties that:
\begin{eqnarray}\displaystyle
c(0) & = & 1.0 \\
s(0) & = & 0.0 \\
\Delta c & = & -s\,\Delta t \\
\Delta s & = & c\,\Delta t
\end{eqnarray}
Make a graph of $c$ vs $t$ and $s$ vs $t$.  Remind you of anything?

\textsl{(b)}~Make an expanded graph in the region $1.5<t<1.6$ and
look where the curve crosses zero.  Multiply your answer by 2 and you
have an estimate of $\pi$!  Why does that work?

\textsl{(c)}~(Advanced) In part (a) you probably incremented $c$ using
the previous-line value of $s$ and incremented $s$ using the
previous-line value of $c$.  Now switch your spreadsheet to increment
$s$ using the \emph{same-line} value of $c$ (keeping the incrementing
of $c$ the same).  What changed about your graphs and why?  Did your
estimate of $\pi$ get better or worse or stay the same?

\paragraph{Extra Problem (will not be graded for credit):}\refstepcounter{problem}%
\textsl{(a)}~How much energy does it take to raise a 1-ton airplane to
a height of $100\,\mi$ above the Earth's surface?  This feat won an
``X-prize''.

\textsl{(b)}~Compare the answers you get using the standard formula
for distances near the surface of the Earth with the correct formula
taking into account the variation of the force of gravity with
distance.  Is the discrepancy large or small?  Explain the
discrepancy.

\textsl{(c)}~Compute now the orbital speed---the speed required for
the airplane to be in orbit, and not just ``flying''---at an altitude
of $100\,\mi$, using that the orbital centripetal acceleration is
provided by gravity.  Is the corresponding kinetic energy much larger
or much smaller than the energy it took to get from the Earth's
surface to the high altitude?

\end{document}
