\documentclass[12pt]{article}
\begin{document}
\renewcommand{\deg}{\mathrm{deg}}
\newcommand{\kg}{\mathrm{kg}}
\newcommand{\m}{\mathrm{m}}
\newcommand{\s}{\mathrm{s}}
\newcommand{\ns}{\mathrm{ns}}
\newcommand{\mps}{\m\,\s^{-1}}
\newcounter{problem}
\thispagestyle{empty}

\section*{NYU Physics 1---spacetime diagrams}

\paragraph{\theproblem}\refstepcounter{problem}%
Draw a spacetime diagram for your own rest-frame.  On the spacetime
diagram, show your own world-line.

\paragraph{\theproblem}\refstepcounter{problem}%
Imagine there is a galaxy flying away from you with a velocity $v =
0.5\,c$. When the galaxy is moving away, it sends back to you a light
signal every $T'=3.3\,\ns$ (as recorded in the galaxy's rest frame).
Draw the world-line of this galaxy on your spacetime diagram and mark
the events corresponding to the departures of the signals from the
galaxy.  Draw at least five such events.

\paragraph{\theproblem}\refstepcounter{problem}%
Draw all the world-lines for all the the signals.  Mark the events of
the signals reaching you.

\paragraph{\theproblem}\refstepcounter{problem}%
Calculate the time intervals between the arrival events (arrivals of
the signals from the galaxy) according to you (that is, in your
frame).  Give your answer in terms of $T'$, $\beta$, and $\gamma$.
\textsl{Hint:} It should be longer than what is suggested by the
simple time-dilation formula.

\paragraph{\theproblem}\refstepcounter{problem}%
Why do the time intervals in the previous problem \emph{not} agree
with the time-dilation formula?  What, on the spacetime diagram,
\emph{does} agree with the time-dilation formula?

\end{document}
