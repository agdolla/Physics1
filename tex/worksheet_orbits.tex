\documentclass[12pt]{article}
\usepackage{url, graphicx}

% page layout
\setlength{\topmargin}{-0.25in}
\setlength{\textheight}{9.5in}
\setlength{\headheight}{0in}
\setlength{\headsep}{0in}

% problem formatting
\newcommand{\problemname}{Problem}
\newcounter{problem}

% math
\newcommand{\dd}{\mathrm{d}}

% primary units
\newcommand{\rad}{\mathrm{rad}}
\newcommand{\kg}{\mathrm{kg}}
\newcommand{\m}{\mathrm{m}}
\newcommand{\s}{\mathrm{s}}

% secondary units
\renewcommand{\deg}{\mathrm{deg}}
\newcommand{\km}{\mathrm{km}}
\newcommand{\mi}{\mathrm{mi}}
\newcommand{\h}{\mathrm{h}}
\newcommand{\ns}{\mathrm{ns}}
\newcommand{\J}{\mathrm{J}}
\newcommand{\eV}{\mathrm{eV}}
\newcommand{\W}{\mathrm{W}}

% derived units
\newcommand{\mps}{\m\,\s^{-1}}
\newcommand{\mph}{\mi\,\h^{-1}}
\newcommand{\mpss}{\m\,\s^{-2}}

% random stuff
\sloppy\sloppypar\raggedbottom\frenchspacing\thispagestyle{empty}

\begin{document}

\section*{NYU Physics I---dimensional analysis of orbits}

Here we exercise dimensional analysis to derive some properties of
gravitational orbits, including one of Kepler's laws.
For context, the Earth ($m= 6\times 10^{24}\,\kg$) orbits the Sun ($M= 2\times
10^{30}\,\kg$) at a distance of $R= 1.5\times 10^{11}\,\m$.

\paragraph{\theproblem}\refstepcounter{problem}%
What is the orbital period $T$ of the Earth in units of $\s$?

\paragraph{\theproblem}\refstepcounter{problem}%
What combination of $M$, $R$, and $T$ has dimensions of acceleration?
What has dimensions of force?  What has dimensions of momentum?

\paragraph{\theproblem}\refstepcounter{problem}%
The force on the Earth from the Sun that keeps it in orbit is
gravitational.  What is the magnitude of the gravitational force, in
terms of $G$, $m$, $M$, and $R$? What about acceleration?

\paragraph{\theproblem}\refstepcounter{problem}%
If you assume that the gravitational force provides the
dimensional-analysis force you derived above, how much longer would
the year be if the Earth's orbit were four times larger in radius than
it is?  The general answer is one of Kepler's laws!

\paragraph{\theproblem}\refstepcounter{problem}%
Now solve the problem and find speed $v$ in terms of $G$, $m$, $M$,
and $R$ for a circular orbit. Do the same for period $T$. By what
factor were your dimensional-analysis answers wrong?

\paragraph{\theproblem}\refstepcounter{problem}%
You got Kepler's law correct even though your dimensional-analysis result
was off by a constant factor.  Why?

\paragraph{\theproblem}\refstepcounter{problem}%
If the Earth were four times less massive than it is, but were still on
a circular orbit at its current radius, how much longer or shorter
would the year be?

\paragraph{\theproblem}\refstepcounter{problem}%
If the Sun were four times less massive than it is, but the Earth were
still on a circular orbit at its current radius, how much longer or
shorter would the year be?

\paragraph{\theproblem}\refstepcounter{problem}%
The Sun orbits the Milky Way on a close-to-circular orbit at a speed
of $2\times 10^5\,\mps$ and a distance of $R= 3\times 10^{20}\,\m$.
What is the orbital time of the Sun around the Milky Way?

\paragraph{\theproblem}\refstepcounter{problem}%
What, therefore, is the mass of the Milky Way, approximately, in units
of the mass of the Sun?

\end{document}
