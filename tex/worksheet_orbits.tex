\documentclass[12pt]{article}
\newcommand{\kg}{\mathrm{kg}}
\newcommand{\m}{\mathrm{m}}
\newcommand{\s}{\mathrm{s}}
\newcommand{\mps}{\m\,\s^{-1}}
\newcounter{problem}
\begin{document}\sloppy\sloppypar\raggedbottom\frenchspacing\thispagestyle{empty}

\section*{NYU Physics 1---dimensional analysis of orbits}

Here we exercise dimensional analysis to derive some properties of
gravitational orbits, including one of Kepler's laws.

\paragraph{\theproblem}\refstepcounter{problem}%
The Earth ($m= 6\times 10^{24}\,\kg$) orbits the Sun ($M= 2\times
10^{30}\,\kg$) at a distance of $R= 1.5\times 10^{11}\,\m$.  What is
the orbital period of the Earth in units of $\s$?

\paragraph{\theproblem}\refstepcounter{problem}%
What combination of $m$, $R$, and $T$ has dimensions of acceleration?
What has dimensions of force?  What has dimensions of momentum?

\paragraph{\theproblem}\refstepcounter{problem}%
The force on the Earth from the Sun that keeps it in orbit is
gravitational.  What is the magnitude of the gravitational force, in
terms of $G$, $m$, $M$, and $R$?

\paragraph{\theproblem}\refstepcounter{problem}%
If you assume that the gravitational force provides the
dimensional-analysis force you derived above, how much longer would
the year be if the Earth's orbit were four times larger in radius than
it is?  The general answer is one of Kepler's laws!

\paragraph{\theproblem}\refstepcounter{problem}%
You get Kepler's law correct even if your dimensional analysis result
is off by a factor of $2\,\pi$.  Why?

\paragraph{\theproblem}\refstepcounter{problem}%
If the Earth were four times less massive than it is, but were still on
a circular orbit at its current radius, how much longer or shorter
would the year be?

\paragraph{\theproblem}\refstepcounter{problem}%
If the Sun were four times less massive than it is, but the Earth were
still on a circular orbit at its current radius, how much longer or
shorter would the year be?

\paragraph{\theproblem}\refstepcounter{problem}%
The Sun orbits the Milky Way on a close-to-circular orbit at a speed
of $2\times 10^5\,\mps$ and a distance of $R= 3\times 10^{20}\,\m$.
What is the orbital time of the Sun around the Milky Way?

\paragraph{\theproblem}\refstepcounter{problem}%
What is the mass of the Milky Way, approximately, in units of the mass
of the Sun?

\end{document}
