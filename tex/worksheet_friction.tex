\documentclass[12pt]{article}
\usepackage{url, graphicx}

% page layout
\setlength{\topmargin}{-0.25in}
\setlength{\textheight}{9.5in}
\setlength{\headheight}{0in}
\setlength{\headsep}{0in}

% problem formatting
\newcommand{\problemname}{Problem}
\newcounter{problem}

% math
\newcommand{\dd}{\mathrm{d}}

% primary units
\newcommand{\rad}{\mathrm{rad}}
\newcommand{\kg}{\mathrm{kg}}
\newcommand{\m}{\mathrm{m}}
\newcommand{\s}{\mathrm{s}}

% secondary units
\renewcommand{\deg}{\mathrm{deg}}
\newcommand{\km}{\mathrm{km}}
\newcommand{\mi}{\mathrm{mi}}
\newcommand{\h}{\mathrm{h}}
\newcommand{\ns}{\mathrm{ns}}
\newcommand{\J}{\mathrm{J}}
\newcommand{\eV}{\mathrm{eV}}
\newcommand{\W}{\mathrm{W}}

% derived units
\newcommand{\mps}{\m\,\s^{-1}}
\newcommand{\mph}{\mi\,\h^{-1}}
\newcommand{\mpss}{\m\,\s^{-2}}

% random stuff
\sloppy\sloppypar\raggedbottom\frenchspacing\thispagestyle{empty}

\begin{document}\sloppy\sloppypar\raggedbottom\frenchspacing\thispagestyle{empty}

\section*{NYU Physics 1---friction}

In lecture we did a problem of a block on an inclined plane.
Here we consider the same ``block on plane'' problem but with
friction.  When we ``switch on'' friction, the contact force between
the block and the plane is no longer purely normal but has both normal
and transverse (frictional) components.  For definiteness, imagine a
plane or bank inclined at about $20\,\deg$ to the horizontal in what
follows.

\paragraph{\theproblem}\refstepcounter{problem}%
Following lecture, or whatever method you like, work out the problem
of a block on an inclined plane in the \emph{absence of friction}.
That is, compute the magnitude of the normal force and the magnitude
and direction of the acceleration of the block.

\paragraph{\theproblem}\refstepcounter{problem}%
Make dimensional and scaling arguments that it makes sense for the
frictional force magnitude to be some dimensionless constant $\mu$
times the normal force magnitude. What typical values of $\mu$ do you
find in the world?

\paragraph{\theproblem}\refstepcounter{problem}%
There are two kinds of friction, sliding (or kinetic) and static. What
is the difference between them? You might want to use the internet or
a book to find this out.

\paragraph{\theproblem}\refstepcounter{problem}%
Imagine that between the block and the plane there is a coefficient of
friction $\mu=0.05$.  Draw the forces on the block, separating the
contact force into its normal and transverse components.

\paragraph{\theproblem}\refstepcounter{problem}%
Solve for the frictional (transverse) component of the contact force.
Which way does it point, and why?

\paragraph{\theproblem}\refstepcounter{problem}%
What is the acceleration of the block in this case?

\paragraph{\theproblem}\refstepcounter{problem}%
Describe a situation in which the frictional force would point
\emph{down} the plane.  When you answered the previous questions, you
made an assumption.  What was it?

\paragraph{\theproblem}\refstepcounter{problem}%
Now imagine that $\mu=0.9$.  What is the magnitude of the frictional
force? Give a simple argument that it \emph{cannot} be
$\mu\,m\,g\,\cos\theta$.

\end{document}
