\documentclass[12pt]{article}
\usepackage{graphics}
\begin{document}

\section*{NYU Engineering Physics 1---In-class Exam 4}

\vfill

\paragraph{Name:} ~

\paragraph{email:} ~

\vfill

This exam consists of two problems.  Write only in this booklet.  Be
sure to show your work.

\vfill ~

\clearpage

\section*{Problem 1}

As a physics demonstration, NYU decides to hang a $M=50~\mathrm{kg}$
pendulum bob from the roof of Bobst Library, by a light cable, in the
12-story-tall lobby.  The cable has a length $\ell$ of 12 stories, and
the pendulum is set swinging with a horizontal amplitude of $x_0=5$~m.

(a) What is the period $T=(2\pi/\omega)$ you expect for the
oscillations of this pendulum?  Give a symbolic answer and a numerical
answer in s.

\vfill

(b) What is the maximum horizontal speed $v_\mathrm{max}$ of the
pendulum bob swinging with this amplitude?  Give a symbolic answer and
a numerical answer in $\mathrm{m\,s^{-1}}$.

\vfill

(c) How much total mechanical energy $E$ is stored in the oscillation?
Give a symbolic answer and a numerical answer in J.

\vfill ~

\clearpage

\section*{Problem 2}

When an olympic diver of mass $M$ walks out to the tip of the diving
board, the diver's weight $M\,g$ causes the tip to sag to a new
equilibrium position a distance $y_\mathrm{eq}$ below its ``unloaded''
equilibrium position.  Estimate, roughly, this distance
$y_\mathrm{eq}$ using the following fact: When the diver is standing
at the very end of a diving board, he or she oscillates vertically
with a period $T$ of roughly 0.5~s.  Give a symbolic answer and plug
in reasonable numbers.

\emph{Hint 1:} Assume the diving board is much lighter than the diver.

\emph{Hint 2:} Model the diving board as a massless spring.

\emph{Hint 3:} Start by getting a \emph{dimensionally correct} guess
before attempting your calculation.

\vfill ~

\clearpage

[This page intentionally left blank for calculations or other work.]

\end{document}
