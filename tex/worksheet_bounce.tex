\documentclass[12pt]{article}
\usepackage{url, graphicx}

% page layout
\setlength{\topmargin}{-0.25in}
\setlength{\textheight}{9.5in}
\setlength{\headheight}{0in}
\setlength{\headsep}{0in}

% problem formatting
\newcommand{\problemname}{Problem}
\newcounter{problem}

% math
\newcommand{\dd}{\mathrm{d}}

% primary units
\newcommand{\rad}{\mathrm{rad}}
\newcommand{\kg}{\mathrm{kg}}
\newcommand{\m}{\mathrm{m}}
\newcommand{\s}{\mathrm{s}}

% secondary units
\renewcommand{\deg}{\mathrm{deg}}
\newcommand{\km}{\mathrm{km}}
\newcommand{\mi}{\mathrm{mi}}
\newcommand{\h}{\mathrm{h}}
\newcommand{\ns}{\mathrm{ns}}
\newcommand{\J}{\mathrm{J}}
\newcommand{\eV}{\mathrm{eV}}
\newcommand{\W}{\mathrm{W}}

% derived units
\newcommand{\mps}{\m\,\s^{-1}}
\newcommand{\mph}{\mi\,\h^{-1}}
\newcommand{\mpss}{\m\,\s^{-2}}

% random stuff
\sloppy\sloppypar\raggedbottom\frenchspacing\thispagestyle{empty}

\begin{document}

\section*{NYU Physics I---bouncing}

\paragraph{\theproblem}\refstepcounter{problem}%
Imagine carefully dropping a basketball and a small rubber ball so
that they fall ``in a stack'' with the basketball immediately
underneath the rubber ball.  They drop a height $h$ and then the
basketball hits the ground, bounces off, and then immediately hits the
rubber ball, which bounces off of it.  In the limit that the rubber
ball is much smaller in mass than the basketball, how high does the
rubber ball fly after the double-bounce?  \emph{Hint:} Treat the two
bounces separately. \emph{Another hint:} For each collision (ball on ground, ball on ball),
consider the reference frame
in which the far heavier object is at rest. Why this frame?

\paragraph{\theproblem}\refstepcounter{problem}%
What happens if there are three balls in a hierarchy?  Tiny rubber
ball on top of small rubber ball on top of basketball?

\paragraph{\theproblem}\refstepcounter{problem}%
Consider a pool shot to be an elastic collision between balls of equal
masses.  The cue ball is moving before the collision and the object
ball is at rest.  You may have observed that after the collision, the
object ball and the cue ball move at right angles to one another?
Prove this, for a general collision, using conservation of kinetic
energy.  \emph{Hint:} The kinetic energy equation is quadratic, as is
what important theorem related to right-angle triangles?

\paragraph{\theproblem}\refstepcounter{problem}%
Now imagine the cue ball is heavier than the object ball.  Will the
cue ball still go on the right-angle path?  If not, will it ``bounce
back more'' or ``follow on more''?

\paragraph{\theproblem}\refstepcounter{problem}%
If a ball bounces off of a wall, the angle of incidence equals the
angle of reflection.  What do you have to assume to make that true?
You have to assume that the collision is elastic, and you have to
assume something about the direction of the force from the wall.  What
is it?  In a pool ``bank shot'' hit hard, the ball reflects
\emph{closer} to the normal direction than you would predict from the
ideal case.  What possible effects are there that might make this
happen?  Think about what happens physically during the bounce, and
what the cushions are made of.

\end{document}
