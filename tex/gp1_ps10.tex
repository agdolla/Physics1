\documentclass[12pt]{article}
\newcommand{\m}{\mathrm{m}}
\newcommand{\cm}{\mathrm{cm}}
\newcommand{\s}{\mathrm{s}}
\newcommand{\dd}{\mathrm{d}}
\newcounter{problem}
\begin{document}\thispagestyle{empty}

\section*{NYU General Physics 1---Problem set 10}

\paragraph{Problem~\theproblem:}\refstepcounter{problem}%
Consider a function $$x(t) = A\,\sin(\omega\,t + \phi) \quad,$$ where
$A$, $\omega$, and $\phi$ are constants.  Now consider the
``differential equation'' $$m\,\frac{\dd^2 x}{\dd t^2} = -k\,x
\quad.$$ Take two derivatives of $x(t)$ and plug the answer into the left-hand
side of the differential equation.  Under what conditions will the
given $x(t)$ satisfy the differential equation?  That is, what needs
to be true about $A$, $\omega$, and $\phi$?  How is the story
different if you have $\cos$ instead of $\sin$?

\paragraph{Problem~\theproblem:}\refstepcounter{problem}%
Consider a string stretched in the $x$ direction, waving transversely
(look it up), with the $y$ displacement being a function of position
$x$ and time $t$ according to
$$ y(x,t) = A\,\cos(\frac{2\pi\,x}{\lambda} - \frac{2\pi\,t}{T}) $$
where $A$ is an amplitude, $\lambda$ is the wavelength, and $T$ is
the period.  For definiteness, set $A=1\,\cm$, $\lambda=0.75\,\m$, and
$T=0.25\,\s$.

\textsl{(a)} Draw a picture of $y(t)$ over the time period $0<t<1\,\s$
for the position $x=0.0\,\m$

\textsl{(b)} Draw a picture of $y(x)$ over the spatial interval $0<x<3\,\m$
for the time $t=0.00\,\s$

\textsl{(c)} Draw a picture of $y(x)$ over the spatial interval $0<x<3\,\m$
for the time $t=0.05\,\s$.

\textsl{(c)} Draw a picture of $y(x)$ over the spatial interval $0<x<3\,\m$
for the time $t=0.10\,\s$.

\textsl{(d)} Which way is the wave moving, and how fast?

\paragraph{Problem~\theproblem:}\refstepcounter{problem}%
\textsl{(a)}~For an ideal gas, find out what the ``intensive''
properties are.  What combinations of them have units of speed?  Can
you find more than one?  Find the true speed of sound on the internet
and check your dimensional answer(s) for standard temperature and
pressure.  How far off are you and are you upset by that?

\textsl{(b)} An organ pipe of length $L$ (filled with air at STP)
supports a standing wave (like the standing wave on the string done in
lecture) but with $L$ being one-quarter wavelength.  What length $L$
do you need to make a pipe that plays the note middle C?  Use the
Wikipedia-reported value for the speed of sound $c_s$ in air at STP,
and use the fact that frequency $f$ and wavelength $\lambda$ are
related to the speed of sound $c_s$ in the most trivial way possible
given their dimensions!

\end{document}
