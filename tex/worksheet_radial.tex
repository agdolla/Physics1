\documentclass[12pt]{article}
\begin{document}
\newcommand{\kg}{\mathrm{kg}}
\newcommand{\m}{\mathrm{m}}
\newcommand{\km}{\mathrm{km}}
\newcommand{\s}{\mathrm{s}}
\newcommand{\mps}{\m\,\s^{-1}}
\newcommand{\dd}{\mathrm{d}}
\newcounter{problem}
\thispagestyle{empty}

\section*{NYU Physics 1---radial orbits in gravity}

\paragraph{\theproblem}\refstepcounter{problem}%
If you have an object that is a distance $r$ from the center of mass
of the Earth (but still subject to it's gravitational pull) and you
let it go from rest (with respect to the Earth), what does it's
``orbit'' or trajectory look like?

\paragraph{\theproblem}\refstepcounter{problem}%
How do you reconcile your answer to the above with the statement that
all orbits in gravity are ellipses (or conic sections)?  What kind of
ellipse is your trajectory?

\paragraph{\theproblem}\refstepcounter{problem}%
What is the magnitude of the gravitational force on an object of mass
$m$ a distance $r$ from a much larger object of mass $M$?

\paragraph{\theproblem}\refstepcounter{problem}%
If you lift the object of mass $m$ a tiny additional distance $\dd r$,
so it goes a tiny bit higher, how much work $\dd W$ do you do?
Treat the displacement as so small that you can treat the
gravitational force as constant through the change.

\paragraph{\theproblem}\refstepcounter{problem}%
Integrate your answer to the previous question to get the \emph{total}
work $\Delta W$ it takes to lift the object of mass $m$ from one
radius $R_1$ to a significantly larger radius $R_2$.  If you get the
wrong \emph{sign} for your integral, don't worry---the signs are
confusing and not very illuminating.  The important thing is to get
the magnitude correct.

\paragraph{\theproblem}\refstepcounter{problem}%
Now imagine that you do that work, and then \emph{let go} of the
object from rest.  It will fall back down from large radius $R_2$ to
small radius $R_1$.  At what speed $v$ will it be moving when it gets
back to $R_1$?  Ignore all other forces, such as air resistance.

\paragraph{\theproblem}\refstepcounter{problem}%
Now take $R_2$ (the large radius) to infinity.  Does the speed $v$ go
to infinity?  If not, what is the limiting speed $v_\mathrm{escape}$?

\paragraph{\theproblem}\refstepcounter{problem}%
Evaluate the limiting speed $v_\mathrm{escape}$ for the Earth, where
$R_1$ is the radius of the Earth and $M$ is the mass of the Earth.  If
you get something within a factor of two of $10\,\km\,\s^{-1}$, you
are okay; if you don't you must have an error.

\paragraph{\theproblem}\refstepcounter{problem}%
Why do we call this speed ``escape speed''?

\end{document}
