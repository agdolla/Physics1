\documentclass[12pt]{article}
\usepackage{url, graphicx}

% page layout
\setlength{\topmargin}{-0.25in}
\setlength{\textheight}{9.5in}
\setlength{\headheight}{0in}
\setlength{\headsep}{0in}

% problem formatting
\newcommand{\problemname}{Problem}
\newcounter{problem}

% math
\newcommand{\dd}{\mathrm{d}}

% primary units
\newcommand{\rad}{\mathrm{rad}}
\newcommand{\kg}{\mathrm{kg}}
\newcommand{\m}{\mathrm{m}}
\newcommand{\s}{\mathrm{s}}

% secondary units
\renewcommand{\deg}{\mathrm{deg}}
\newcommand{\km}{\mathrm{km}}
\newcommand{\mi}{\mathrm{mi}}
\newcommand{\h}{\mathrm{h}}
\newcommand{\ns}{\mathrm{ns}}
\newcommand{\J}{\mathrm{J}}
\newcommand{\eV}{\mathrm{eV}}
\newcommand{\W}{\mathrm{W}}

% derived units
\newcommand{\mps}{\m\,\s^{-1}}
\newcommand{\mph}{\mi\,\h^{-1}}
\newcommand{\mpss}{\m\,\s^{-2}}

% random stuff
\sloppy\sloppypar\raggedbottom\frenchspacing\thispagestyle{empty}

\begin{document}

\noindent
Name: \rule[-1ex]{0.55\textwidth}{0.1pt}
NetID: \rule[-1ex]{0.2\textwidth}{0.1pt}

\section*{NYU Physics I---Term Exam 2}

\paragraph{\problemname~\theproblem:}\refstepcounter{problem}%
(From Lecture on 2017-09-28.) In Roller-Coaster Design School, we
considered a cart going over a hill with radius of curvature $R$. We
said it would be bad if $v^2 > R\,g$. What would happen if $v^2 > R\,g$?

\vfill

\paragraph{\problemname~\theproblem:}\refstepcounter{problem}%
(From Lecture on 2017-10-03.) In a one-dimensional problem, a $3\,\kg$
block moves to the right at $2\,\mps$, and a $5\,\kg$ block
moves to the left at $1\,\mps$.
Choose a coordinate system and tell me the total momentum of the system
in that coordinate system.

\vfill

\paragraph{\problemname~\theproblem:}\refstepcounter{problem}%
(From Problem Set 3.) Imagine a runner who starts at rest, and then
accelerates at constant acceleration, and then runs at a constant
velocity, all in the $x$ direction.  She starts at rest, accelerates
at $5\,\m\,\s^{-2}$ for $0 < t < 2\,\s$ and then goes at constant
speed for $2 < t < 12\,\s$.  How far does she go in these $12\,\s$?

\vfill
~

\clearpage
\paragraph{\problemname~\theproblem:}\refstepcounter{problem}%
(From Problem Set 4.) What is your kinetic energy when you are walking
down the street? Make reasonable assumptions about your mass and velocity and
anything else you need to assume. Give your answer in SI (standard metric) units.

\vfill

\paragraph{\problemname~\theproblem:}\refstepcounter{problem}%
(From blocks-and-pulleys worksheet.) A massless pulley hangs from the
ceiling from a string which is at tension $T_1$. Over this pulley is another
string at tension $T_2$, on the ends of which are massive blocks attached. What is the
relationship between $T_1$ and $T_2$? If you have to assume additional things to
solve this problem, state them.

\vfill

\paragraph{\problemname~\theproblem:}\refstepcounter{problem}%
(From friction worksheet.) You have a block of mass $m$ on an inclined
plane, inclined at an angle $\theta=15\,\deg$ to the horizontal. The
coefficient of friction is $\mu=0.5$. What is the magnitude of the
frictional force on the block? The acceleration due to gravity is $g$.
Once again, state any assumptions you need to make.

\vfill
~
\end{document}
