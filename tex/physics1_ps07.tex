\documentclass[12pt]{article}
\usepackage{url, graphicx}

% page layout
\setlength{\topmargin}{-0.25in}
\setlength{\textheight}{9.5in}
\setlength{\headheight}{0in}
\setlength{\headsep}{0in}

% problem formatting
\newcommand{\problemname}{Problem}
\newcounter{problem}

% math
\newcommand{\dd}{\mathrm{d}}

% primary units
\newcommand{\rad}{\mathrm{rad}}
\newcommand{\kg}{\mathrm{kg}}
\newcommand{\m}{\mathrm{m}}
\newcommand{\s}{\mathrm{s}}

% secondary units
\renewcommand{\deg}{\mathrm{deg}}
\newcommand{\km}{\mathrm{km}}
\newcommand{\mi}{\mathrm{mi}}
\newcommand{\h}{\mathrm{h}}
\newcommand{\ns}{\mathrm{ns}}
\newcommand{\J}{\mathrm{J}}
\newcommand{\eV}{\mathrm{eV}}
\newcommand{\W}{\mathrm{W}}

% derived units
\newcommand{\mps}{\m\,\s^{-1}}
\newcommand{\mph}{\mi\,\h^{-1}}
\newcommand{\mpss}{\m\,\s^{-2}}

% random stuff
\sloppy\sloppypar\raggedbottom\frenchspacing\thispagestyle{empty}

\begin{document}

\section*{NYU Physics I---Problem Set 7}

Due Thursday 2016 October 27 at the beginning of lecture.

\paragraph{\problemname~\theproblem:}\refstepcounter{problem}%
A typical adult man is holding his left arm at a right angle, so the
upper arm is pointing straight down, and the forearm is pointing
horizontally forwards.  His hand is oriented palm-up.  He is holding a
$20\,\kg$ grocery bag by its handle in his left hand.  Look up the
point of attachment of the relevant tendon and make sensible estimates
(or look them up) for all lengths and masses.  In what follows, treat
the ``hand plus forearm'' to be one monolithic object; that is, we
primarily want to understand the forces at or near the elbow.

\textsl{(a)} Draw a free-body diagram for the hand-plus-forearm
system, identifying all significant forces acting on it (including
from the bag handle, and don't forget the elbow joint---the contact
force from the upper arm bones).

\textsl{(b)} Compute the magnitudes and directions of all forces, and
the magnitudes and directions of all torques, taking the elbow to be
the axis of rotation (that is, the origin or reference point).  For
simplicity, take the tendon direction and joint contact force both to
be precisely vertical.  That is, treat all angles as being right
angles.  This is not a bad approximation.

\textsl{(c)} Look up the definition of ``mechanical advantage'' and
compute the mechanical advantage the grocery bag has over the tendon.
Why would evolution (such a brilliant designer) decide to put tendons
under this kind of stress?

\paragraph{\problemname~\theproblem:}\refstepcounter{problem}%


\paragraph{\problemname~\theproblem:}\refstepcounter{problem}%
Determine the value of $\pi$ by integration!

\textsl{(a)}~Make a spreadsheet with a dimensionless column marked
``$t$'' that goes from 0.0 to 20.0 in units of 0.05.  Now make columns
marked ``$c$'' and ``$s$''.  Integrate the functions $c(t)$ and
$s(t)$ with the properties that:
\begin{eqnarray}\displaystyle
c(0) & = & 1.0 \\
s(0) & = & 0.0 \\
\Delta c & = & -s\,\Delta t \\
\Delta s & = & c\,\Delta t
\end{eqnarray}
Make a graph of $c$ vs $t$ and $s$ vs $t$.  Remind you of anything?

\textsl{(b)}~Make an expanded graph in the region $1.5<t<1.6$ and
look where the curve crosses zero.  Multiply your answer by 2 and you
have an estimate of $\pi$!  Why does that work?

\end{document}
