\documentclass[12pt]{article}
\usepackage{url, graphicx}

% page layout
\setlength{\topmargin}{-0.25in}
\setlength{\textheight}{9.5in}
\setlength{\headheight}{0in}
\setlength{\headsep}{0in}

% problem formatting
\newcommand{\problemname}{Problem}
\newcounter{problem}

% math
\newcommand{\dd}{\mathrm{d}}

% primary units
\newcommand{\rad}{\mathrm{rad}}
\newcommand{\kg}{\mathrm{kg}}
\newcommand{\m}{\mathrm{m}}
\newcommand{\s}{\mathrm{s}}

% secondary units
\renewcommand{\deg}{\mathrm{deg}}
\newcommand{\km}{\mathrm{km}}
\newcommand{\mi}{\mathrm{mi}}
\newcommand{\h}{\mathrm{h}}
\newcommand{\ns}{\mathrm{ns}}
\newcommand{\J}{\mathrm{J}}
\newcommand{\eV}{\mathrm{eV}}
\newcommand{\W}{\mathrm{W}}

% derived units
\newcommand{\mps}{\m\,\s^{-1}}
\newcommand{\mph}{\mi\,\h^{-1}}
\newcommand{\mpss}{\m\,\s^{-2}}

% random stuff
\sloppy\sloppypar\raggedbottom\frenchspacing\thispagestyle{empty}

\begin{document}

\section*{NYU Physics I---Problem Set 7}

Due Thursday 2016 October 27 at the beginning of lecture.

\paragraph{\problemname~\theproblem:}\refstepcounter{problem}%
what

\paragraph{\problemname~\theproblem:}\refstepcounter{problem}%
ever

\paragraph{\problemname~\theproblem:}\refstepcounter{problem}%
Determine the value of $\pi$ by integration!

\textsl{(a)}~Make a spreadsheet with a dimensionless column marked
``$t$'' that goes from 0.0 to 20.0 in units of 0.05.  Now make columns
marked ``$c$'' and ``$s$''.  Integrate the functions $c(t)$ and
$s(t)$ with the properties that:
\begin{eqnarray}\displaystyle
c(0) & = & 1.0 \\
s(0) & = & 0.0 \\
\Delta c & = & -s\,\Delta t \\
\Delta s & = & c\,\Delta t
\end{eqnarray}
Make a graph of $c$ vs $t$ and $s$ vs $t$.  Remind you of anything?

\textsl{(b)}~Make an expanded graph in the region $1.5<t<1.6$ and
look where the curve crosses zero.  Multiply your answer by 2 and you
have an estimate of $\pi$!  Why does that work?

\textsl{(c)}~(Advanced) In part (a) you probably incremented $c$ using
the previous-line value of $s$ and incremented $s$ using the
previous-line value of $c$.  Now switch your spreadsheet to increment
$s$ using the \emph{same-line} value of $c$ (keeping the incrementing
of $c$ the same).  What changed about your graphs and why?  Did your
estimate of $\pi$ get better or worse or stay the same?

\paragraph{\problemname~\theproblem:}\refstepcounter{problem}%
Look up the value and units of Newton's constant $G$.
\textsl{(a)}~Find some combination of $G$, the speed of light $c$, and
some (arbitrary) mass $M$, that has units of \emph{length}.
\textsl{(b)}~Evaluate your expression for the mass of the Sun (you
will have to look it up); that is, find the characteristic length
associated with that mass.  Give your answer in units of km.  What is
the physical meaning of this length (approximately)?

\end{document}
