\documentclass[12pt]{article}
\usepackage{amssymb}
\usepackage{amsfonts}
\usepackage{epsfig,latexsym}
\voffset -.5cm
\hoffset -1.5cm
\textheight 21cm
\textwidth 16cm
\def\dspace{\baselineskip = .30in}

\def\beq{\begin{equation}}
\def\eeq{\end{equation}}
\def\be{\begin{eqnarray}}
\def\ee{\end{eqnarray}}

\begin{document}
\begin{center}
{\bf\large Homework 9.}
\end{center}

{\bf Problem 1.}

(a) Conservation of momentum: $m\vec{v}_0 =2m \vec{v}$, thus $v=v_0/2$ in the same direction.
Angular momentum should be also conserved: $mv_0 r= I_f \omega$.
Moment of inertia of a puck regarding its center mass is $mr^2/2$, if we shift axes of rotation
by $r$ it will be $mr^2/2 +mr^2$ and since we have two pucks we get $I_f=3mr^2$.
So, $\omega=v_0/(3r)$.

Initial kinetic energy is $K_i=mv_0^2/2$, final one is $K_f=2m v^2/2 +I_f \omega^2/2=5/6 \;K_i$.
Fraction of energy lost is $(K_i-K_f)/K_i=1/6$.

(b) Only $v$ would be the same (as stuck-dead defined $\omega=0$). Difference of energy
would go to stop rotation, and for energy loss we would get $(K_i-K_f)/K_i=1/2$.
\\

{\bf Problem 2.}

There are no vertical motion. Horizontally the only force is force of friction $F_f$.
Equations of motion are $M a=F_f$ and $I \alpha =F_f R$, where $I=2/5\;MR^2$ is moment of
inertia of the ball, $\alpha$ is angular acceleration. If the ball slides, it means that force of friction
is maximal $F_f=\mu Mg$. We get $a=\mu g$ and $\alpha=5/2\;\mu g/R$. No sliding condition
is $v=\omega R$ (ball i just rolling). Velocity is function of time $v=v_i -at$ (linear decreasing) ,  angular velocity is $\omega=\alpha t$ (linear increasing). Thus at time $t_r$  we should have
$v=\omega R$, or $v_i -at_r=\alpha t_r R$. Solving for $t_r$ we get 
$$t_r=\frac{v_i}{a+\alpha R}=\frac{2}{7}\frac{v_i}{\mu g}$$
\\

{\bf Problem 3.}

Let's split cone on disk layers of radius $r=R y/h$ and thickness $dy$ , where $y$ is changing from $0$ to $h$ (we have cone with the pick at $y=0$ and getting wider at the top).
Moment of inertia of the disk layer is $dI=dm r^2/2$, where $dm=\rho dV$ is mass of the layer,
$\rho$ is density of the cone, $dV=\pi r^2 dy$ is volume of the layer. 

Total moment of inertia is
$$I=\int dI=\int \frac{1}{2}dm r^2= \frac{1}{2}\int dV \rho r^2= \frac{\pi}{2}\int_{0}^{h} dy r^2 \rho r^2=
 \frac{\pi \rho R^4}{2 h^4}\int_{0}^{h} dy y^4= \frac{\pi \rho R^4 h}{10}.$$
The volume of the cone is 
$$V=\int dV = \pi \int_{0}^{h} dy r^2 = \frac{\pi R^2 h}{3}.$$
Thus we can write $\rho=M/V$, or in other words 
$$I=\frac{3}{10}MR^2.$$
\\
\\


{\bf Problem 4.}

Initial kinetic energy is $K_i=M v_i^2/2$, the final is $K_f=M v_f^2/2 +I\omega_f^2/2$,
where $v_f=v_i-at_r=5/7\;v_i$, $\omega_f=\alpha t_r=5/7\;v_i/R$. Thus $K_f=5/7\;K_i$ and
energy loss (for heat) is $\Delta K=2/7\;K_i$.

Work done by friction force is $W=F_f x$, where $x=v_i t_t -a t_r^2/2=12/49\;v_i^2/(\mu g)$.
So, for work we have $W=24/49\;K_i$, it is  bigger than $\Delta K$, because part of the
work goes to make the ball rotate. You can check that $W-\Delta K=I\omega_f /2$.



\end{document}