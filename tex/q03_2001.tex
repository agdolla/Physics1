\documentclass[12pt]{article}
\usepackage{graphics}
\begin{document}

\section*{NYU Physics 1---In-class Exam 3}

\vfill

\paragraph{Name:} ~

\paragraph{email:} ~

\paragraph{recitation:} ~

\vfill

This exam consists of two problems.  Write only in this booklet.  Be
sure to show your work.

\vfill ~

\clearpage

\section*{Problem 1}

A Japanese Shinkansen (bullet train) (of mass $M=10^6~\mathrm{kg}$),
moving at $v_S=300~\mathrm{km\,h^{-1}}$ with respect to the tracks,
hits and collides elastically with a superball (of mass
$m=3~\mathrm{g}$), which is initially at rest.  The front face of the
train is inclined at an angle of $\theta=45~\mathrm{deg}$ to the
horizontal, as shown.
\\ \rule{0.1\textwidth}{0pt}
\resizebox{0.8\textwidth}{!}{\includegraphics{../mp/shinkansen.eps}}

(a) Draw a diagram of the ball--train system, just before the
collision, in the center-of-mass rest frame.  Clearly show the
velocities of the ball and train in this frame.

\vfill

(b) What is the final speed and direction of the ball, immediately
after the collision, in the rest frame of the tracks (ie, \emph{not}
in the center-of-mass frame)?

\vfill ~

\clearpage

\section*{Problem 2}

A four-wheeled cart rolls without slipping at speed $v$ on a flat
surface.  Each of the cart's four wheels has mass $m$, radius $R$ and
moment of inertia $I$.  The total mass of the cart, including all four
wheels, is $(M+4m)$.

(a) When the cart is moving at speed $v$, what is its total kinetic
energy $K$?  Don't forget to include the kinetic energies of the
rotating wheels, and the rolling-without-slipping condition which
relates wheel angular speed $\omega$ to speed $v$.  Give your answer
in terms of quantities named.

\vfill

(b) Now---more difficult---imagine that the cart rolls without
slipping down a ramp inclined at an angle of $\theta$ to the
horizontal.  What is the magnitude of the cart's linear acceleration
$a$ down the ramp?  You will need to draw a free body diagram for the
whole cart (body plus wheels on one diagram), consider torques on the
four wheels, and consider the sum of forces on the whole cart.  Assume
that the friction on the axles is negligible.

\vfill ~

\clearpage

[This page intentionally left blank for calculations or other work.]

\end{document}
