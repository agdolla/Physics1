\documentclass[12pt]{article}
\usepackage[pdftex]{graphicx}
\newcommand{\dd}{\mathrm{d}}
\newcommand{\kg}{\mathrm{kg}}
\newcommand{\m}{\mathrm{m}}
\newcommand{\cm}{\mathrm{cm}}
\newcommand{\mm}{\mathrm{mm}}
\newcommand{\km}{\mathrm{km}}
\newcommand{\mi}{\mathrm{mi}}
\newcommand{\s}{\mathrm{s}}
\newcommand{\ms}{\mathrm{ms}}
\newcommand{\h}{\mathrm{h}}
\newcommand{\N}{\mathrm{N}}
\newcommand{\W}{\mathrm{W}}
\newcommand{\Pa}{\mathrm{Pa}}
\newcommand{\C}{\mathrm{C}}
\newcommand{\hp}{\mathrm{hp}}
\newcommand{\rad}{\mathrm{rad}}
\newcounter{problem}
\stepcounter{problem}
\newcounter{answer}[problem]
\newenvironment{problem}{\noindent\begin{minipage}{\textwidth}\sloppy\sloppypar\raggedright\textbf{\theproblem.}\refstepcounter{problem}\stepcounter{answer}---}{\end{minipage}\\[2ex]}
\newcommand{\source}[1]{[{#1}]}
\newenvironment{answers}{\\[0.5ex]}{}
\newcommand{\answer}[1]{\textbf{\Alph{answer}:}\refstepcounter{answer}~\mbox{#1\hspace{3ex}}}
\newcommand{\longanswer}[1]{\textbf{\Alph{answer}:}\refstepcounter{answer}~{#1}\\}
\begin{document}\sloppy\sloppypar

\section*{NYU General Physics 1---Term Exam 4}

\begin{problem}
  \source{from lecture 2013-11-19} We considered a ``block of water''
  floating in water.  Which statement is \emph{not} true?
  \begin{answers}
    \longanswer{The gravitational force on the block points downwards.}
    \longanswer{The buoyant force on the block points upwards.}
    \longanswer{The buoyant force is related to a difference in pressure between the top of the block and the bottom.}
    \longanswer{There is a net upward force on the block of water.}
  \end{answers}
\end{problem}

\begin{problem}
  \source{from lecture 2013-11-21} We considered the air pressure in a
  bus (treated as a sealed box) accelerating forwards.  In our
  two-dimensional picture of the bus, where was the pressure highest?
  \begin{answers}
    \answer{top front corner}
    \answer{top rear corner}
    \answer{bottom front corner}
    \answer{bottom rear corner}
    \answer{none of these}
  \end{answers}
\end{problem}

\begin{problem}
  \source{from lecture 2013-11-26} Water was flowing in a graduated
  pipe from a wide part (large diameter) to a narrow part (small
  diameter).  Which statement is true?
  \begin{answers}
    \longanswer{The speed of the water is the same everywhere.}
    \longanswer{The water accelerates because of a pressure gradient.}
    \longanswer{The pressure is higher in the narrow part.}
    \longanswer{As the pressure increases, so does the speed of the water.}
    \longanswer{The pressure is proportional to the square of the speed.}
  \end{answers}
\end{problem}

\begin{problem}
  \source{from lecture 2013-12-03} Bernoulli's constant
  $(P+\frac{1}{2}\,\rho\,v^2+\rho\,g\,h)$ has what units?
  \begin{answers}
    \answer{energy}
    \answer{volume per time}
    \answer{momentum}
    \answer{energy per volume}
    \answer{momentum per area}
  \end{answers}
\end{problem}

\begin{problem}
  \source{from lecture 2013-12-05} If a cell or water droplet has its
  pressure determined by its size and its surface tension, how does
  the pressure inside the cell or droplet change with radius?
  \begin{answers}
    \answer{pressure decreases as radius increases}
    \answer{pressure is independent of radius}
    \answer{pressure increases as radius increases}
  \end{answers}
\end{problem}

\begin{problem}
  \source{from lecture 2013-12-10} We computed the viscous stress
  (pressure) in an artery of length $L$ and radius $R$ carrying blood at
  mean speed $v$.  If you fix the length and the speed but increase
  the radius of the artery, what happens to the volumetric flow (volume
  of blood per unit time through the artery) and the viscous stress?
  \begin{answers}
    \longanswer{the volumetric flow increases and the stress increases}
    \longanswer{the volumetric flow decreases and the stress increases}
    \longanswer{the volumetric flow increases and the stress decreases}
    \longanswer{the volumetric flow decreases and the stress decreases}
  \end{answers}
\end{problem}

\begin{problem}
  \source{from problem set 11, problem 1} In part $(c)$, when you
  plotted
  $$\displaystyle y(x,t) = A \cos(\frac{2\pi\,x}{0.75\,\m})\,\cos(\frac{2\pi\,t}{0.25\,\s})$$
  at time $t=0.05\,\s$, at which position $x$ did $y(x)$ become exactly zero?
  \begin{answers}
    \answer{$x=0.0\,\m$}
    \answer{$x=0.1875\,\m$}
    \answer{$x=0.25\,\m$}
    \answer{$x=0.375\,\m$}
    \answer{$x=0.75\,\m$}
  \end{answers}
\end{problem}

\begin{problem}
  \source{from problem set 11, problem 2} The bulk modulus of blood is
  much greater than the bulk modulus of meat.  What does this mean?
  \begin{answers}
    \longanswer{blood is more easily compressible than meat}
    \longanswer{blood is less easily compressible than meat}
    \longanswer{blood is more solid than meat}
    \longanswer{blood is less solid than meat}
  \end{answers}
\end{problem}

\begin{problem}
  \source{from problem set 11, problem 3} What, roughly, is the
  quality factor $Q$ of a piano string?
  \begin{answers}
    \answer{2}
    \answer{60}
    \answer{2000}
    \answer{60,000}
  \end{answers}
\end{problem}

\begin{problem}
  \source{from problem set 12, problem 1} What is the ratio of the
  density of mercury to the density of air, approximately?
  \begin{answers}
    \answer{10}
    \answer{100}
    \answer{1,000}
    \answer{10,000}
  \end{answers}
\end{problem}

\begin{problem}
  \source{from problem set 12, problem 2} An ice cube (frozen water,
  with no included air bubbles) at $0\,\C$ floats in a glass of water
  at $0\,\C$.  Approximately what fraction of the cube is below the
  surface of the water?
  \begin{answers}
    \answer{0.5}
    \answer{0.9}
    \answer{0.99}
    \answer{0.999}
  \end{answers}
\end{problem}

\begin{problem}
  \source{from problem set 12, problem 2} An ice cube at $0\,\C$
  floats in a glass of water at $0\,\C$. When the ice cube melts, does
  the water level go up or down?
  \begin{answers}
    \answer{down}
    \answer{stays the same}
    \answer{up}
  \end{answers}
\end{problem}

\begin{problem}
  \source{from problem set 12, problem 3} How is the magnitude $F_b$
  of the buoyant force on your body related to the magnitude $F_g$ of
  the gravitational force on your body?  Assume that you are standing
  on the surface of the Earth in air at STP.
  \begin{answers}
    \answer{$\displaystyle F_b \approx\frac{F_g}{800}$}
    \answer{$\displaystyle F_b \approx\frac{F_g}{80}$}
    \answer{$\displaystyle F_b \approx\frac{F_g}{8}$}
    \answer{$\displaystyle F_b \approx F_g$}
  \end{answers}
\end{problem}

\begin{problem}
  \source{from problem set 13, problem 1} If $\rho$ is the density of
  air, what is the pressure difference $\Delta P$ between the front
  and back of a sealed bus of length $L$ moving at speed $v$ and
  accelerating at acceleration $a$?
  \begin{answers}
    \answer{$\Delta P = \rho\,a\,L$}
    \answer{$\Delta P = \rho\,g\,L$}
    \answer{$\Delta P = \frac{1}{2}\,\rho\,v^2$}
    \answer{None of the above}
  \end{answers}
\end{problem}

\begin{problem}
  \source{from problem set 13, problem 2} You solved a problem about
  the pressure difference $\Delta P$ created if an aorta carrying
  $5\,\ell\,\min^{-1}$ narrows from a diameter of $3.5\,\cm$ to a
  diameter of $2.5\,\cm$.  Now imagine instead that the aorta narrows
  from a diameter of $3.5\,\cm$ all the way to a diameter of
  $0.5\,\cm$.  What is the new pressure difference $\Delta P$ in this
  case?  (\emph{Hint:} No need to calculate this exactly; use the fact
  that $[0.5]^2$ is much smaller than $[3.5]^2$.)
  \begin{answers}
    \answer{$\Delta P = 17\,\Pa$}
    \answer{$\Delta P = 170\,\Pa$}
    \answer{$\Delta P = 1,700\,\Pa$}
    \answer{$\Delta P = 17,000\,\Pa$}
  \end{answers}
\end{problem}

\begin{problem}
  \source{from problem set 14, problem 1} What is the surface tension of water, roughly?
  \begin{answers}
    \answer{$0.07\,\N\,\m^{-1}$}
    \answer{$0.7\,\N\,\m^{-1}$}
    \answer{$7\,\N\,\m^{-1}$}
    \answer{$70\,\N\,\m^{-1}$}
  \end{answers}
\end{problem}

\begin{problem}
  \source{from problem set 14, problem 2} Which is larger, viscous
  drag or ram-pressure force?  Consider a car moving at freeway
  speeds, and consider a red blood cell sinking slowly in plasma.
  \begin{answers}
    \longanswer{ram-pressure is larger than viscous for the car, viscous is larger than ram-pressure for the cell}
    \longanswer{viscous is larger than ram-pressure for the car, ram-pressure is larger than viscous for the cell}
    \longanswer{ram-pressure is larger in both cases}
    \longanswer{viscous is larger in both cases}
  \end{answers}
\end{problem}

\begin{problem}
  \source{from \textit{Work--Energy} lab} How did you calibrate the
  force sensor?
  \begin{answers}
    \longanswer{By hanging a known weight from it.}
    \longanswer{By accelerating a known mass at a known acceleration.}
    \longanswer{By stretching an elastic band attached to a cart.}
  \end{answers}
\end{problem}

\begin{problem}
  \source{from \textit{Oscillations of a String} lab} The lab manual
  explained that the normal-mode frequencies $\nu_n$ are
  $$\displaystyle\nu_n = \frac{n}{2\,L}\,\sqrt{\frac{T}{\rho}}\quad,$$
  where $L$ is the string length, $T$ is the string tension, and
  $\rho$ is the string mass per unit length.  The normal-mode
  wavelengths $\lambda_n$ are
  $$\displaystyle\lambda_n = \frac{2\,L}{n}\quad.$$
  How are these related?
  \begin{answers}
    \answer{$\nu_n\propto(\lambda_n)^2$}
    \answer{$\nu_n\propto\lambda_n$}
    \answer{$\displaystyle\nu_n\propto\frac{1}{\lambda_n}$}
    \answer{$\displaystyle\nu_n\propto\frac{1}{(\lambda_n)^2}$}
  \end{answers}
\end{problem}

\begin{problem}
  \source{from \textit{Resonance Tube} lab} The speed of sound in air
  at STP is roughly $340\,\m\,\s^{-1}$.  If you doubled the (absolute)
  temperature of the air, what would you expect the speed of sound to
  become?
  \begin{answers}
    \answer{$170\,\m\,\s^{-1}$}
    \answer{$240\,\m\,\s^{-1}$}
    \answer{$340\,\m\,\s^{-1}$}
    \answer{$480\,\m\,\s^{-1}$}
    \answer{$680\,\m\,\s^{-1}$}
  \end{answers}
\end{problem}

\end{document}
