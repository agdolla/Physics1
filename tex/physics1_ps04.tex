\documentclass[12pt]{article}
\usepackage{url, graphicx}

% page layout
\setlength{\topmargin}{-0.25in}
\setlength{\textheight}{9.5in}
\setlength{\headheight}{0in}
\setlength{\headsep}{0in}

% problem formatting
\newcommand{\problemname}{Problem}
\newcounter{problem}

% math
\newcommand{\dd}{\mathrm{d}}

% primary units
\newcommand{\rad}{\mathrm{rad}}
\newcommand{\kg}{\mathrm{kg}}
\newcommand{\m}{\mathrm{m}}
\newcommand{\s}{\mathrm{s}}

% secondary units
\renewcommand{\deg}{\mathrm{deg}}
\newcommand{\km}{\mathrm{km}}
\newcommand{\mi}{\mathrm{mi}}
\newcommand{\h}{\mathrm{h}}
\newcommand{\ns}{\mathrm{ns}}
\newcommand{\J}{\mathrm{J}}
\newcommand{\eV}{\mathrm{eV}}
\newcommand{\W}{\mathrm{W}}

% derived units
\newcommand{\mps}{\m\,\s^{-1}}
\newcommand{\mph}{\mi\,\h^{-1}}
\newcommand{\mpss}{\m\,\s^{-2}}

% random stuff
\sloppy\sloppypar\raggedbottom\frenchspacing\thispagestyle{empty}

\begin{document}

\section*{NYU Physics I---Problem Set 4}

Due Thursday 2017 October 05 at the beginning of lecture.

\paragraph{Problem~\theproblem:}\refstepcounter{problem}%
A normal, healthy NYU student climbs 10 stories of stairs at a
reasonable, steady pace. Compute the potential-energy gain of the
student. Compute also the kinetic energy of the student during this
climb. You will have to estimate the mass of the student, the pace of
stair climbing, and the height of a story of a building. Make
reasonable and explicit assumptions. Give a simple explanation for why
it is reasonable that one energy is far larger than the other?

\paragraph{Problem~\theproblem:}\refstepcounter{problem}%
Let's extend the problem that Prof Hogg nearly completed in lecture
on 2017-09-21: Consider a car driving around a circular banked turn at
constant altitude with the bank set at $15\,\deg$ to the horizontal,
and the car traveling at $55\,\mph$.

\textsl{(a)} Draw the free-body diagrams and compute the magnitude of
the normal force in the case that the car is going around the curve at
exactly the speed such that no lateral friction force is required at
all, and the car neither loses nor gains altitude (it goes in a
perfect, level, circle). That is, draw the free-body diagram for the
frictionless, perfect-driving case.

\textsl{(b)} What is the radius of curvature $R$ of the turn in this
case?

\textsl{(c)} Now imagine that the car came into this same turn---with
the same bank angle and same radius of turn $R$---but at
$70\,\mph$. That is, a speeder. And, further, assume that there is
plenty of static frictional force for the car to rely on. (Note, it is
\emph{static} friction that keeps a car on the road, not sliding
friction!). What is the magnitude of the frictional force required to
keep the car driving in the level, circular turn? Be very careful with
your coordinate system and your diagrams.

\paragraph{Problem~\theproblem:}\refstepcounter{problem}%
Here is a non-trivial machine that delivers a kind of mechanical
advantage:
\\ \rule{0.33\textwidth}{0pt}
\resizebox{0.33\textwidth}{!}{\includegraphics{../mp/tackle_blocks.pdf}}

\textsl{(a)}~Draw free-body diagrams for all the masses and pulleys in
this mechanism. Assume that the strings and pulleys are light and
frictionless, so that the strings are perfect tension-transmitters.

\textsl{(b)} What is the kinematic constraint---the relationship
between the accelerations of block 1 and block 2? Be very careful with
this one. Treat all the strings as inextensible for simplicity.

\textsl{(c)} Find the tensions in all three strings and the
accelerations of the two blocks.

\textsl{(d)}~Now set the mass $m_2$ to the value it must have if the
system is to be perfectly ``balanced''; that is, for there to be no
net acceleration of either block.  In this situation, if block $m_1$
is lowered a small distance $h$, what is the net change in potential
energy, accounting for the displacements of both blocks?

\paragraph{Problem~\theproblem:}\refstepcounter{problem}%
Gasoline and olive oil are both substances with great chemical energy
content per unit mass.

\textsl{(a)} In the case of gasoline, the chemical energy is mainly in
carbon bonds.  If you assume that gasoline is \emph{entirely} carbon
atoms, and each one releases $4\,\eV$ of energy when it is combusted, how
much energy per unit mass is there in gasoline?  Get an answer in MJ
per kg and compare to what you find on \textit{Wikipedia}.  How far off
are our assumptions?

\textsl{(b)} Now convert your answer to kcal per g and compare it to
what is written on the ``Nutrition Facts'' label on an olive oil
bottle.  How close are you?  It should be close, (Prof Hogg thinks), because
biofuel is made from things like olive oil!

\paragraph{Extra Problem (will not be graded for credit):}%
\textsl{(a)} Assume that a car moving at speed
$v=75\,\mi\,\h^{-1}$ encounters an air resistance force of
$\frac{1}{2}\,\rho\,A\,v^2$, where $\rho$ is the density of air and $A$ is the
cross-sectional area of the car, about $2.5\,\m^2$ How much work does it
take to move the car $30\,\mi$ at this speed?

\textsl{(b)} If a car with these properties was \emph{perfectly
  efficient}, how many miles per gallon would it get? You need to use a value for
the energy density in gasoline.

\textsl{(c)} What does this
make you think about the future of cars that are \emph{far more
  efficient} than cars we have today? For example, could we have cars
that are 100 times more efficient than today's cars?

\end{document}
