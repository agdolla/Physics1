\documentclass[12pt]{article} \usepackage{url, graphicx}

% page layout
\setlength{\topmargin}{-0.25in}
\setlength{\textheight}{9.5in}
\setlength{\headheight}{0in}
\setlength{\headsep}{0in}

% problem formatting
\newcommand{\problemname}{Problem}
\newcounter{problem}

% math
\newcommand{\dd}{\mathrm{d}}

% primary units
\newcommand{\rad}{\mathrm{rad}}
\newcommand{\kg}{\mathrm{kg}}
\newcommand{\m}{\mathrm{m}}
\newcommand{\s}{\mathrm{s}}

% secondary units
\renewcommand{\deg}{\mathrm{deg}}
\newcommand{\km}{\mathrm{km}}
\newcommand{\mi}{\mathrm{mi}}
\newcommand{\h}{\mathrm{h}}
\newcommand{\ns}{\mathrm{ns}}
\newcommand{\J}{\mathrm{J}}
\newcommand{\eV}{\mathrm{eV}}
\newcommand{\W}{\mathrm{W}}

% derived units
\newcommand{\mps}{\m\,\s^{-1}}
\newcommand{\mph}{\mi\,\h^{-1}}
\newcommand{\mpss}{\m\,\s^{-2}}

% random stuff
\sloppy\sloppypar\raggedbottom\frenchspacing\thispagestyle{empty}

\begin{document}

\noindent
Name: \rule[-1ex]{0.55\textwidth}{0.1pt}
NetID: \rule[-1ex]{0.2\textwidth}{0.1pt}

\section*{NYU Physics I---Term Exam 5}

\paragraph{\problemname~\theproblem:}\refstepcounter{problem}%
When you weigh yourself on a scale, you weigh less than your
gravitational force, because there is a buoyant-force correction. What
fraction of your weight is this correction? That is, what is the
buoyant force divided by the gravitational force, roughly?
(from Problem Set 9)

\vfill

\paragraph{\problemname~\theproblem:}\refstepcounter{problem}%
A figure skater spins in place on frictionless ice at angular speed
$\omega_i$ with hands outstretched. The skater has a total moment of inertia
$I_i$. As the skater draws his hands into his body, his moment of
inertia decreases to $I_f$ = $I_i/2$. Does the kinetic energy $K$
increase, decrease, or stay the same? Assume that there are no torques
acting.
(from Problem Set 10)

\vfill

\paragraph{\problemname~\theproblem:}\refstepcounter{problem}%
Immediately after being hit by the cue, a cue ball slides along the
felt in the $x$ direction at speed $v$. Draw a free-body diagram for
the cue ball, showing the forces acting, and clearly label the $x$
direction.
(from Problem Set 10)

\vfill
~
\clearpage

\paragraph{\problemname~\theproblem:}\refstepcounter{problem}%
What would be the length of an Earth year if the Earth was orbiting
on a circular orbit of radius $0.25\,\AU$ (instead of $1\,\AU$)?
Give your answer in units of days.
(from worksheet on orbits)

\vfill

\paragraph{\problemname~\theproblem:}\refstepcounter{problem}%
What is the relationship between linear acceleration $a$ and
angular acceleration $\alpha$ when a ball of radius $R$ is rolling
without slipping down a plane?
(from lecture 2018-11-08)

\vfill

\paragraph{\problemname~\theproblem:}\refstepcounter{problem}%
How is it that the astronauts in the Space Station are weightless?
Write a grammatically correct answer in one sentence, in fewer than 20
words.  Box your sentence!
(from lecture 2018-11-15)

\vfill
~
\end{document}
