\documentclass[12pt]{article} \usepackage{url, graphicx}

% page layout
\setlength{\topmargin}{-0.25in}
\setlength{\textheight}{9.5in}
\setlength{\headheight}{0in}
\setlength{\headsep}{0in}

% problem formatting
\newcommand{\problemname}{Problem}
\newcounter{problem}

% math
\newcommand{\dd}{\mathrm{d}}

% primary units
\newcommand{\rad}{\mathrm{rad}}
\newcommand{\kg}{\mathrm{kg}}
\newcommand{\m}{\mathrm{m}}
\newcommand{\s}{\mathrm{s}}

% secondary units
\renewcommand{\deg}{\mathrm{deg}}
\newcommand{\km}{\mathrm{km}}
\newcommand{\mi}{\mathrm{mi}}
\newcommand{\h}{\mathrm{h}}
\newcommand{\ns}{\mathrm{ns}}
\newcommand{\J}{\mathrm{J}}
\newcommand{\eV}{\mathrm{eV}}
\newcommand{\W}{\mathrm{W}}

% derived units
\newcommand{\mps}{\m\,\s^{-1}}
\newcommand{\mph}{\mi\,\h^{-1}}
\newcommand{\mpss}{\m\,\s^{-2}}

% random stuff
\sloppy\sloppypar\raggedbottom\frenchspacing\thispagestyle{empty}

\begin{document}

\noindent
Name: \rule[-1ex]{0.55\textwidth}{0.1pt}
NetID: \rule[-1ex]{0.2\textwidth}{0.1pt}

\section*{NYU Physics I---Term Exam 5}

\paragraph{\problemname~\theproblem:}\refstepcounter{problem}%
What is are the units of $P\,V$ (that is, pressure times volume)?
(From worksheet on the ideal gas.)

\vfill

\paragraph{\problemname~\theproblem:}\refstepcounter{problem}%
What is the kinematic relationship between acceleration $a$ (in length
per time-squared) and angular acceleration $\alpha$ (in angle per
time-squared) for something that is rolling without slipping?
(From lecture on 2017-11-07.)

\vfill

\paragraph{\problemname~\theproblem:}\refstepcounter{problem}%
A blimp has a volume of $7000\,\m^3$ of He (atomic mass 4) at STP,
floating in air (atomic mass around 28) at STP. How much mass in $\kg$
can the blimp carry, roughly? That mass will include the skin, the
cabin, the motors, the crew and cargo!
(From Problem Set 9.)

\vfill
~
\clearpage

\paragraph{\problemname~\theproblem:}\refstepcounter{problem}%
Compare a point $1\,\m$ below the surface of a lake at sea level to
a point $5\,\m$ below the surface. What is the pressure
difference between these points? Which one is at higher pressure?
(From lecture on 2017-11-02.)

\vfill

\paragraph{\problemname~\theproblem:}\refstepcounter{problem}%
If the acceleration due to gravity at the surface of the Earth is
$10\,\mpss$, the mass of the Earth is $6\times 10^{24}\,\kg$, and the
radius of the Earth is $6000\,\km$, what would you compute to be the
value of Newton's Constant $G$? Give your answer in SI units, with units.
(From lecture on 2017-11-14.)

\vfill

\paragraph{\problemname~\theproblem:}\refstepcounter{problem}%
A figure skater spins in place on frictionless ice at angular speed
$w_i$ with her hands outstretched. She has a total moment of inertia
$I_i$. As the skater draws her hands into her body, her moment of
inertia decreases to $I_f$ = $I_i/2$. Does her kinetic energy $K$
increase, decrease, or stay the same? Assume that there are no torques
acting.
(From Problem Set 9.)

\vfill
~
\end{document}
