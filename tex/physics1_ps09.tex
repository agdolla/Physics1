\documentclass[12pt]{article}
\usepackage{url, graphicx}

% page layout
\setlength{\topmargin}{-0.25in}
\setlength{\textheight}{9.5in}
\setlength{\headheight}{0in}
\setlength{\headsep}{0in}

% problem formatting
\newcommand{\problemname}{Problem}
\newcounter{problem}

% math
\newcommand{\dd}{\mathrm{d}}

% primary units
\newcommand{\rad}{\mathrm{rad}}
\newcommand{\kg}{\mathrm{kg}}
\newcommand{\m}{\mathrm{m}}
\newcommand{\s}{\mathrm{s}}

% secondary units
\renewcommand{\deg}{\mathrm{deg}}
\newcommand{\km}{\mathrm{km}}
\newcommand{\mi}{\mathrm{mi}}
\newcommand{\h}{\mathrm{h}}
\newcommand{\ns}{\mathrm{ns}}
\newcommand{\J}{\mathrm{J}}
\newcommand{\eV}{\mathrm{eV}}
\newcommand{\W}{\mathrm{W}}

% derived units
\newcommand{\mps}{\m\,\s^{-1}}
\newcommand{\mph}{\mi\,\h^{-1}}
\newcommand{\mpss}{\m\,\s^{-2}}

% random stuff
\sloppy\sloppypar\raggedbottom\frenchspacing\thispagestyle{empty}

\begin{document}

\section*{NYU Physics I---Problem Set 9}

Due Thursday 2016 November 10 at the beginning of lecture.

\paragraph{\problemname~\theproblem:}\refstepcounter{problem}%
Here we consider the construction and tuning of a standard grand piano.
Each string of the piano has a mass $M$, a length $L$, and a tension $T$.

\textsl{(a)} Use dimensional analysis to estimate the natural angular
frequency $\omega$ of a piano string with these properties. That is,
what combination has units of frequency?

\textsl{(b)} Look up the natural frequency of a guitar or piano string
in its lowest harmonic. You might have to look up ``standing wave'' or
something like that, and you might also have to look up ``transverse
wave speed'' in a string. A string fixed at both ends (like a piano
string) is different from an open organ pipe!

\textsl{(c)} Look inside a piano at or near middle C. Roughly what are
the diameters of the strings? And what are the lengths of the strings?
Use these quantities and the density of steel to estimate the masses
of the strings.

\textsl{(d)} Given what you know about the piano---the number of keys,
the number of strings per key (which isn't one for most keys), and the
range of frequencies and string lengths, estimate \emph{very roughly}
what the total stress is on a piano frame, in Newtons. That is,
estimate the total of all the tension forces. Do you understand why
piano frames and harps are so heavy?

\paragraph{\problemname~\theproblem:}\refstepcounter{problem}%
something about oscillations

\paragraph{\problemname~\theproblem:}\refstepcounter{problem}%
something about fluid or buoyancy

\paragraph{Extra Problem (will not be graded for credit):}%
Look up the Youngs modulus for steel, and estimate the spring
potential energy stored in the middle C string, using what you figured
out in the piano problem. Now estimate the total mechanical energy
stored in the tuned piano!

\end{document}
