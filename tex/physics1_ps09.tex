\documentclass[12pt]{article}
\usepackage{url, graphicx}

% page layout
\setlength{\topmargin}{-0.25in}
\setlength{\textheight}{9.5in}
\setlength{\headheight}{0in}
\setlength{\headsep}{0in}

% problem formatting
\newcommand{\problemname}{Problem}
\newcounter{problem}

% math
\newcommand{\dd}{\mathrm{d}}

% primary units
\newcommand{\rad}{\mathrm{rad}}
\newcommand{\kg}{\mathrm{kg}}
\newcommand{\m}{\mathrm{m}}
\newcommand{\s}{\mathrm{s}}

% secondary units
\renewcommand{\deg}{\mathrm{deg}}
\newcommand{\km}{\mathrm{km}}
\newcommand{\mi}{\mathrm{mi}}
\newcommand{\h}{\mathrm{h}}
\newcommand{\ns}{\mathrm{ns}}
\newcommand{\J}{\mathrm{J}}
\newcommand{\eV}{\mathrm{eV}}
\newcommand{\W}{\mathrm{W}}

% derived units
\newcommand{\mps}{\m\,\s^{-1}}
\newcommand{\mph}{\mi\,\h^{-1}}
\newcommand{\mpss}{\m\,\s^{-2}}

% random stuff
\sloppy\sloppypar\raggedbottom\frenchspacing\thispagestyle{empty}

\begin{document}

\section*{NYU Physics I---Problem Set 9}

Due Thursday 2016 November 10 at the beginning of lecture.

\paragraph{\problemname~\theproblem}\refstepcounter{problem}%
In a bungy jump, the bungy cord has a rest (unstretched) length of
$\ell_0 = 5\,\m$, and has a spring constant $k$ such that it stretches
by 1~m for every 200~N of force.

\textsl{(a)} If an adult of mass $M=80\,\kg$ jumps off of a very tall
bridge at time $t=0$ with this bungy cord attached between her or
himself and also the bridge, to what maximum distance $h_\mathrm{max}$
below the bridge will he or she fall?  (You might use energy
conservation). Give your answer in terms of the symbols $M$, $\ell_0$,
$k$, and $g$ as well as numerically. That is, we want the numerical
and symbolic answers both.

\textsl{(b)} What is the maximum tension $T$ in the bungy? Again, give
both answers.

\textsl{(c)} If you stiffen the cord (increase $k$), but keep
everything else fixed, do you increase or decrease the maximum
tension?

\textsl{(d)} What acceleration $a$ does the adult feel at the maximum
extension of the bungy (that is, at the bottom)? Again, give your
answer both symbolically (in terms of the same symbols) and numerically.

\paragraph{\problemname~\theproblem:}\refstepcounter{problem}%
Here we consider the construction and tuning of a standard grand piano.
Each string of the piano has a mass $M$, a length $L$, and a tension $T$.

\textsl{(a)} Use dimensional analysis to estimate the natural angular
frequency $\omega$ of a piano string with these properties. That is,
what combination has units of frequency?

\textsl{(b)} Look up the natural frequency of a guitar or piano string
in its lowest harmonic. You might have to look up ``standing wave'' or
something like that, and you might also have to look up ``transverse
wave speed'' in a string. A string fixed at both ends (like a piano
string) is different from an open organ pipe!

\textsl{(c)} Look inside a piano at or near middle C. Roughly what are
the diameters of the strings? And what are the lengths of the strings?
Use these quantities and the density of steel to estimate the masses
of the strings.

\textsl{(d)} Given what you know about the piano---the number of keys,
the number of strings per key (which isn't one for most keys), and the
range of frequencies and string lengths, estimate \emph{very roughly}
what the total stress is on a piano frame, in Newtons. That is,
estimate the total of all the tension forces. Do you understand why
piano frames and harps are so heavy?

\paragraph{Problem~\theproblem:}\refstepcounter{problem}%
Everything submerged in the Earth's atmosphere is subject to a buoyant
force from the air.  In the following, use a sensible (reasonably
accurate) measure of the density of air at STP.

\textsl{(a)} When you measure your weight on a standard bathroom
scale, you are measuring the \emph{normal force} between yourself and
the floor.  This normal force opposes the \emph{combination} of
gravity and buoyancy.  What is the correction to your weight coming
from buoyancy, roughly?  Express it as a \emph{fraction} of the
gravitational force.  Is this correction positive or negative---that
is, does it increase or decrease the weight measured by the scale?

\textsl{(b)} Look up the ``volume'' of the Goodyear blimp model GZ-22.
Imagine that it is floating in an air atmosphere at STP, and that the
gas inside the blimp is \emph{also} at STP.  What is the approximate
buoyant force on the blimp if it is filled with helium?  What about if
it were filled with hydrogen (molecular hydrogen)?  Compare these
numbers with the gross weight and capacity of the blimp.

\textsl{(c)} Will the buoyant force increase, decrease, or stay the
same as you decrease the pressure or decrease the temperature?
Assume that the blimp contents are always at the same pressure as the
exterior air (and therefore the volume must change).

\paragraph{Extra Problem (will not be graded for credit):}%
A child of mass $m$ sits exactly on top of a hemispherical mound of
frictionless ice of radius $R$.  If the child is displaced a tiny (ie,
small relative to $R$) horizontal distance $x$ from the top of the
mound of ice, what is the $x$-component $F_x$ of the net force on the
child?  Write down the differential equation relating the $x(t)$ to
its second derivative (with respect to time).  Use the small-angle
approximation to get rid of trigonometric functions! What functions
$x(t)$ solve your equation?  Try to be as general as possible.
\emph{Hint: Try exponentials!}.

\paragraph{Extra Problem (will not be graded for credit):}%
Look up the Youngs modulus for steel, and estimate the spring
potential energy stored in the middle C string, using what you figured
out in the piano problem. Now estimate the total mechanical energy
stored in the tuned piano!

\end{document}
