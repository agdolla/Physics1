\documentclass[12pt]{article}
\usepackage{url, graphicx}

% page layout
\setlength{\topmargin}{-0.25in}
\setlength{\textheight}{9.5in}
\setlength{\headheight}{0in}
\setlength{\headsep}{0in}

% problem formatting
\newcommand{\problemname}{Problem}
\newcounter{problem}

% math
\newcommand{\dd}{\mathrm{d}}

% primary units
\newcommand{\rad}{\mathrm{rad}}
\newcommand{\kg}{\mathrm{kg}}
\newcommand{\m}{\mathrm{m}}
\newcommand{\s}{\mathrm{s}}

% secondary units
\renewcommand{\deg}{\mathrm{deg}}
\newcommand{\km}{\mathrm{km}}
\newcommand{\mi}{\mathrm{mi}}
\newcommand{\h}{\mathrm{h}}
\newcommand{\ns}{\mathrm{ns}}
\newcommand{\J}{\mathrm{J}}
\newcommand{\eV}{\mathrm{eV}}
\newcommand{\W}{\mathrm{W}}

% derived units
\newcommand{\mps}{\m\,\s^{-1}}
\newcommand{\mph}{\mi\,\h^{-1}}
\newcommand{\mpss}{\m\,\s^{-2}}

% random stuff
\sloppy\sloppypar\raggedbottom\frenchspacing\thispagestyle{empty}

\begin{document}

\section*{NYU Physics I---ideal gas law}

Here we investigate the statistical properties of a large number of
free particles in a box.
Imagine you have $N$ particles, each of mass $m$, in a box of volume $V$.
For definiteness, imagine that the box is a cube of side length $\ell$.
Work with a partner and make sure you both agree on each part and understand
why at each stage. This is an introduction to statistical mechanics.

\paragraph{\theproblem}\refstepcounter{problem}%
Just for kicks, imagine that \emph{half} of the particles are moving
at speed $+v$ in the $x$-direction, and half are moving at speed $-v$
in the $x$-direction. When the particles hit the box walls, they
bounce elastically off them. How many particles per unit time, on
average, hit the wall that is perpendicular to the $x$ direction?
Make sure you have a good, simple argument, and that your answer is
dimensionally correct.

\paragraph{\theproblem}\refstepcounter{problem}%
If each particle bounces elastically and normally off the wall (that
is, there is no frictional force at the wall), then how much momentum
is delivered by the wall at each individual-particle impact? That is,
what is the impulse of each impact?

\paragraph{\theproblem}\refstepcounter{problem}%
If the average force is the average of momentum provided per unit
time, what is the mean force provided by the ``gas'' of particles to
the wall, and what is the mean pressure (force per area)? Do you see
why there is a mean force? If not, argue it out.

\paragraph{\theproblem}\refstepcounter{problem}%
Now imagine that each particle has not the same velocity, but some
mean squared velocity $\bar{v_x^2}$ in the $x$-direction. What is the
mean pressure in terms of the mean squared velocity?

\paragraph{\theproblem}\refstepcounter{problem}%
In high school, did you learn that $P\,V = n\,R\,T$? What were the
units of $P$, $V$, $P\,V$, $n$, $R$, $T$, and $n\,R\,T$?

\paragraph{\theproblem}\refstepcounter{problem}%
In statistical mechanics, the equation is $P\,V = N\,k\,T$, where $N$
is the number of molecules, and $k$ is the Boltzmann constant. What is
the ratio $R/k$, both conceptually and numerically?

\paragraph{\theproblem}\refstepcounter{problem}%
Space is three-dimensional, so the mean squared three-space
velocity can be written in terms of one-dimensional means $\bar{v^2} =
\bar{v_x^2} + \bar{v_y^2} + \bar{v_z^2}$. If the gas of particles is
isotropic, these three contributions to the mean velocity magnitude
ought to all be equal, or three times the mean-squared $x$ component.
Relate the pressure you got above to the mean-squred three-space velocity,
and also the mean kinetic energy per molecule $(1/2)\,m\,\bar{v^2}$.

\paragraph{\theproblem}\refstepcounter{problem}%
Rearrange your equation into the form $P\,V=\mbox{something}$. How can
you define $T$ in terms of the per-particle kinetic energy to make the
statistical mechanics equation $P\,V = N\,k\,T$ true? Do you have any
comments to make?

\paragraph{\theproblem}\refstepcounter{problem}%
Look up equipartition on the web and comment on the relationship
between what you got and what you expect from your web reading. Any
comments on the derivation above?

\end{document}
