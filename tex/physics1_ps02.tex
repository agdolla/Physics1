\documentclass[12pt]{article}
\newcommand{\s}{\mathrm{s}}
\renewcommand{\deg}{\mathrm{deg}}
\newcommand{\mps}{\mathrm{m}\,\mathrm{s}^{-1}}
\newcounter{problem}
\begin{document}\sloppy\sloppypar\raggedbottom\frenchspacing\thispagestyle{empty}

\section*{NYU Physics I---Problem set 2}

Due Thursday 2016 September 22 at the beginning of lecture.

\paragraph{Problem~\theproblem:}\refstepcounter{problem}%
Re-do the numerical integration worksheet problem from recitation this
week (download it from the course web site if you have forgotten), but
this time do it with a computer spreadsheet and use a time resolution
($\Delta t$) of $0.01\,\s$.  No need to hand in the whole spreadsheet,
or the answers to all the questions, but hand in a graph (plotted by
your spreadsheet program) of the position as a function of time for
the duration $0<t<2\,\s$.  Make sure your axes are clearly labeled and
``calibrated'' in units of m and s.

\paragraph{Problem~\theproblem}\refstepcounter{problem}%
Now do something similar to the above, but analytically.  Consider a
stone thrown at $t=0$ precisely upwards (in the $y$ direction, for
definiteness) at $1.5\,\mps$, with an initial position (launch point)
at $y=0$.  Ignore air resistance!  Make very careful plots of the
vertical position $y$, the vertical velocity $v_y$, and the vertical
acceleration $a_y$ of the stone as a function of time for the duration
$0<t<0.4\,\s$.  Carefully label the time and $y$-position of
the peak of the trajectory (the highest point) in all three curves,
and the time at which the trajectory passes back through $y=0$, if
that ever happens. Be very careful to include units with all of your
numbers and labels!

\paragraph{Problem~\theproblem}\refstepcounter{problem}%
When an airplane turns a corner, it banks (tilts). When the plane is
flown correctly, the tangent (yes, the trig function ``tan'') of this
tilt angle is set by the ratio of the transverse acceleration
(centripetal acceleration) to the acceleration due to gravity.

\textsl{(a)} If you see a commercial jet aircraft tilted at $30\,\deg$
because it is turning, about what do you think the radius of the turn
is? Clearly state your assumptions! You might have to look up or
estimate the speed at which planes fly. Does your answer seem
reasonable given what you know about planes and travel?

\textsl{(b)} Draw a free-body diagram for the turning airplane,
showing the gravitational force, the lift force from the wings, the
thrust force from the engines, and the drag force from air resistance.
These are the \emph{only} four forces you need to have to explain
the turning airplane.

\end{document}
