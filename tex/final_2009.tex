\documentclass[12pt]{article}
\usepackage{graphics}
\newcommand{\kg}{\mathrm{kg}}
\newcommand{\m}{\mathrm{m}}
\newcommand{\s}{\mathrm{s}}
\newcommand{\h}{\mathrm{h}}
\renewcommand{\deg}{\mathrm{deg}}
\newcommand{\km}{\mathrm{km}}
\newcommand{\mi}{\mathrm{mi}}
\newcommand{\cm}{\mathrm{cm}}
\newcommand{\mps}{\m\,\s^{-1}}
\newcommand{\miph}{\mi\,\h^{-1}}
\newcommand{\mpss}{\m\,\s^{-2}}
\newcommand{\kgpmmm}{\kg\,\m^{-3}}
\newcommand{\N}{\mathrm{N}}
\newcommand{\J}{\mathrm{J}}
\newcommand{\W}{\mathrm{W}}
\newcommand{\hp}{\mathrm{hp}}
\newcommand{\Npmm}{\N\,\m^{-2}}
\newcommand{\tv}[1]{\mathbf{\vec{#1}}}
\newcommand{\dd}{\mathrm{d}}
\newcommand{\cell}[1]{\texttt{{#1}}}
\newcounter{problem}
% \addtolength{\oddsidemargin}{0in}
\addtolength{\textheight}{\headheight}
\setlength{\headheight}{0in}
\addtolength{\textheight}{\headsep}
\setlength{\headsep}{0in}
% \setlength{\marginparwidth}{2in}
\begin{document}

\section*{NYU Physics 1---final exam}

Thursday 2009 December 17.

\section*{Name:}

~ \vfill ~

\clearpage

~ \vfill ~

\begin{center}
\textsl{[This page intentionally left blank.]}
\end{center}

~ \vfill ~

\clearpage

\paragraph{Problem~\theproblem:}\refstepcounter{problem}%
An object is moving at speed $v$.  Describe, in \emph{20 words or
  less}, the significance of the quantity $v^2/g$, where $g$ is the
acceleration due to gravity.

~ \vfill ~

\paragraph{Problem~\theproblem:}\refstepcounter{problem}%
Derive, from the law of gravity, the period of the orbit of the Space
Shuttle.  It orbits about $300\,\km$ above the surface of the Earth.
Give your answer in minutes.  State explicitly anything you have to
assume and show your work.

~ \vfill ~

\clearpage

\paragraph{Problem~\theproblem:}\refstepcounter{problem}%
What is the peak force exerted on a basketball when you bounce it,
dropping it from a height of about 2\,m?  Estimate masses and
velocities.  For how long is the ball in contact with the ground?  The
time can be approximated by the length of time it takes a sound wave
to cross a basketball---since the basketball is filled with air, use
the speed of sound in air (about $340\,\mps$).  You will have to
estimate the diameter too.  Show your work.

~ \vfill ~

\paragraph{Problem~\theproblem:}\refstepcounter{problem}%
A typical cheap American car has a mass of $1000\,\kg$, a
cross-sectional area of $3\,\m^2$, and produces $100\,\hp$ ($1\,\hp$
is about $750\,\W$) of mechanical power.  If air resistance is the
limiting factor, what is the top speed at which this car can drive, in
$\miph$?  That is, at what speed does the $100\,\hp$ equal the power
done against air resistance?  Use the air resistance approximation
$F=\rho\,A\,v^2$.  Show your work.

~ \vfill ~

\clearpage

\paragraph{Problem~\theproblem:}\refstepcounter{problem}%
Determine the magnitude $f$ of the frictional force acting on block
$m_1$ in this system.  Use masses $m_1= 30\,\kg$ and $m_2=10\,\kg$ and
coefficient of friction $\mu=0.1$.  Assume that the pulley and strings
are massless and frictionless.  Show your
work.\\ ~\hfill\includegraphics{../mp/pulley_table.eps}\hfill~

~ \vfill ~

\paragraph{Problem~\theproblem:}\refstepcounter{problem}%
A block of mass $1\,\kg$ is moving at $3\,\mps$ in the positive $x$
direction, and a block of mass $4\,\kg$ is moving at $1\,\mps$, also
in the positive $x$ direction.  What are the velocities of the two
blocks relative to the center of mass, or---equivalently---in the
center-of-mass frame?

~ \vfill ~

\clearpage

\paragraph{Problem~\theproblem:}\refstepcounter{problem}%
If a velocity is $v = 10^{-12}\,c$, then what is the corresponding
Lorentz factor $\gamma$?  Or, if you prefer, how much does $\gamma$
differ from 1.0 in this case?

~ \vfill ~

\paragraph{Problem~\theproblem:}\refstepcounter{problem}%
Charged pions are produced in high-energy collisions between protons
and neutrons. They decay in their own rest frame according to the law
\begin{equation}
N(t) = N_0\,e^{-t/T} \quad ,
\end{equation}
where $T=2.6\times10^{-8}\,\s$ is the mean lifetime. A burst of pions
is produced at the target of an accelerator and it is observed that
two-thirds of them survive at a distance of $30\,\m$ from the
target. At what $\gamma$ value are the pions moving?  Show your work.

~ \vfill ~

\clearpage

\paragraph{Problem~\theproblem:}\refstepcounter{problem}%
Immediately after being hit, at $t=0$, a cue ball of mass $M$ and
radius $R$ slides along the felt at speed $v_i$, not rotating at all
($\omega_i = 0$.  As time goes on, the ball slows down (because of
friction) and, at the same time, starts to spin.  Sketch a plot of the
velocity $v(t)$ and the radius-scaled angular velocity $R\,\omega(t)$
vs $t$ on a single plot.  Clearly label the time at which it starts
rolling without slipping (but there is no need to calculate anything).

~ \vfill ~

\paragraph{Problem~\theproblem:}\refstepcounter{problem}%
The pictured gyroscope precesses.  Which way?  Clockwise or
counter-clockwise as viewed from above?  Show your argument.

~ \vfill ~

\clearpage

~ \vfill ~

\begin{center}
\textsl{[This page intentionally left blank.]}
\end{center}

~ \vfill ~

\end{document}
