\documentclass[12pt]{article}
\newcommand{\kg}{\mathrm{kg}}
\newcommand{\m}{\mathrm{m}}
\newcommand{\s}{\mathrm{s}}
\newcommand{\J}{\mathrm{J}}
\newcommand{\mps}{\m\,\s^{-1}}
\newcommand{\kcal}{\mathrm{kcal}}
\begin{document}
\newcounter{problem}
\thispagestyle{empty}

\section*{NYU General Physics 1---Problem set 6}

\paragraph{Problem~\theproblem:}\refstepcounter{problem}%
\textsl{(a)} What combination of mass $m$ and speed $v$ have units of energy?

\textsl{(b)} What combination of pressure $P$ and length $L$ have units of energy?

\textsl{(c)} What combination of acceleration $g$, mass $m$, and length $L$ have units of energy?

\textsl{(c)} What combination of density $\rho$, length $L$, and speed $v$ have units of energy?

\textsl{(d)} What is an eV in J?  This is the typical energies of what kinds of energetic events?

\textsl{(e)} What is a ``ton of TNT equivalent'' in J?  What is it's cultural significance?

\textsl{(f)} What is a ``barrel of oil equivalent'' in J?  What is it's economic significance?

\paragraph{Problem~\theproblem:}\refstepcounter{problem}%
A standard-sized and typically able college-age human  walks into a building for an appointment on the 10th story.  The
elevator is broken.  She walks up the 9 stories.  Roughly speaking:

\textsl{(a)} How much mechanical energy does she need to expend to climb
those 9 stories?

\textsl{(b)} How much time do you think it takes her to climb those
stairs?  Assume she is at least somewhat motivated to get to her
appointment.  If you are having trouble estimating this, hire a friend
to time themselves climbing stairs.  Use that time and the total
energy from part \textsl{(a)} to get a power, and express the power in
horsepower.  Are you surprised?  Why is horsepower defined as it is?

\textsl{(c)} Now convert the energy you got in part \textsl{(a)} to
what dieticians call ``Calories'', which are really SI kcal units.
What fraction of a standard human $2000\,\kcal$ diet was this stair
climbing exercise?  Does this make sense given what you know about
programs of ``exercise''?

\textsl{(d)} How many flights of stairs could our subject climb in a day if
all her body did was convert a $2000\,\kcal$ input of food into
stair-climbing energy?  Why is that not at all a realistic description
of the body?  On what bodily processes is energy spent in forms
\emph{other} than mechanical forms?

\paragraph{Problem~\theproblem:}\refstepcounter{problem}%
A student of mass $m_\mathrm{student}=80\,\kg$ stands at rest next to
a block of ice of mass $m_\mathrm{ice}=320\,\kg$, also at rest, on a
frictionless frozen lake.  The student pushes on the block until the
block is moving away from the student at $1.5\,\mps$ (that is, until
$\left|\vec{v}_\mathrm{ice}-\vec{v}_\mathrm{student}\right|=1.5\,\mps$).
How much work did the student do?  Give your answer in $\J$.  Don't
forget to conserve linear momentum!  \emph{Hint:} All that work went into
kinetic energy.  \emph{Another hint:} One of the hard things about
this problem is that I am giving you the \emph{relative} velocity and
not the absolute velocity.  How are you going to deal with that?
Spend some time visualizing the problem before writing equations.

\end{document}
