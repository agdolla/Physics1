\documentclass[12pt]{article}
\usepackage{amssymb}
\usepackage{amsfonts}
\usepackage{epsfig,latexsym}
\voffset 0cm
\hoffset -1.5cm
\textheight 20cm
\textwidth 16cm
\def\dspace{\baselineskip = .30in}

\def\beq{\begin{equation}}
\def\eeq{\end{equation}}
\def\be{\begin{eqnarray}}
\def\ee{\end{eqnarray}}


\begin{document}
\begin{center}
{\bf\large Homework 3.}
\end{center}

{\bf Problem 1.}

Since force of friction is constant, we have constant deceleration $a$, such that
$v -at=0$, where $t$ is stopping time. For $v=60\;mi/h=26.7\;m/s$ stopping time is about $5\;s$,
thus the acceleration produced by friction force is $a=5.4\;m/s^2$. Friction force is about
$F_f=ma\approx 1500\;kg\times 5.4\;m/s^2\approx 8\times 10^3 \;N$. Friction force applied to
ground is directed forward, correspondingly force acting on car is backwards.
Distance which car will go is $at^2/2\approx 68\; m$.
\\

{\bf Problem 2.}

The generic formula for each step is
$$\sin x_{i+1}=\sin x_i +\cos x_i \Delta x,\;\;\;\;\; \cos x_{i+1}=\cos x_i -\sin x_i \Delta x,$$
where $\Delta x$ is a step, for this problem 0.1 rad (one can think of this as Taylor series
at each point to the first order, or you can make analogy with position, velocity and acceleration).
Thus we get\\
\begin{tabular}{|c|c|c||c|c|c||c|c|c|}
\hline
angle & Sin &Cos& angle& Sin &Cos &angle&Sin&Cos\\
\hline
0&0&1&0.7&0.665&0.79&1.4&1.06&0.19\\
0.1& 0.1&1&0.8& 0.745&0.73&1.5&1.07&0.08\\
0.2& 0.2&0.99&0.9& 0.82&0.65&1.6&1.08&-0.02\\
0.3& 0.299&0.97&1.0& 0.88&0.571&1.7&1.07&-0.13\\
0.4& 0.396&0.94&1.1& 0.94&0.48&1.8&1.06&-0.24\\
0.5& 0.49&0.9&1.2& 0.988&0.39&1.9&1.04&-0.35\\
0.6& 0.58&0.85&1.3& 1.03&0.29&2.0&1.00&-0.45\\
\hline
\end{tabular}\\
We see that $\pi/2$ should be between 1.5 and 1.6 since Cos changes sign,
which is approximately true. To increase accuracy one can take smaller step or
improve algorithm by going to quadratic approximation.
\\

{\bf Problem 3.}

To keep the package on the orbit "geometry" requires that centripetal
acceleration was $a=v^2/R_{E}$. The physical force which produces this acceleration is 
gravity, thus for stationary orbit one needs $a=g$ or velocity should be $v=\sqrt{g R_{E}}$.
Period of the orbit is $T=2\pi R_{E}/v=2\pi \sqrt{R_{E}/g}\approx 5015\;s\approx 1\;hour\;23\;min$ this seems to be pretty close to the actual period of the Space Shuttle.
\\

{\bf Problem 4.}

There is gravitational force, but the Cosmic Ship is in free fall, like elevator with cut
support cable, like banjo jumper in a first few seconds. Nevertheless there is gravitational
force, your scales  (which measure weight) will fall down with the same acceleration
as you and you won't produce any pressure on scales, so they won't react and 
thus your weight will be zero.
\\

{\bf Problem 5.}
 
 First, why airplanes fly? - Because of lifting force perpendicular to wings.
 Second, what we need to make a turn? - We need centripetal acceleration, so
 we need some physical force which will act as centripetal. The only force we have
 is lifting force (since gravity is always directed down it cannot produce centripetal force).
 We should tilt the wings then horizontal projection of the lifting force will act as
 centripetal. Thus, in the air one should tilt.




\end{document}