\documentclass[12pt]{article}
\usepackage{url, graphicx}

% page layout
\setlength{\topmargin}{-0.25in}
\setlength{\textheight}{9.5in}
\setlength{\headheight}{0in}
\setlength{\headsep}{0in}

% problem formatting
\newcommand{\problemname}{Problem}
\newcounter{problem}

% math
\newcommand{\dd}{\mathrm{d}}

% primary units
\newcommand{\rad}{\mathrm{rad}}
\newcommand{\kg}{\mathrm{kg}}
\newcommand{\m}{\mathrm{m}}
\newcommand{\s}{\mathrm{s}}

% secondary units
\renewcommand{\deg}{\mathrm{deg}}
\newcommand{\km}{\mathrm{km}}
\newcommand{\mi}{\mathrm{mi}}
\newcommand{\h}{\mathrm{h}}
\newcommand{\ns}{\mathrm{ns}}
\newcommand{\J}{\mathrm{J}}
\newcommand{\eV}{\mathrm{eV}}
\newcommand{\W}{\mathrm{W}}

% derived units
\newcommand{\mps}{\m\,\s^{-1}}
\newcommand{\mph}{\mi\,\h^{-1}}
\newcommand{\mpss}{\m\,\s^{-2}}

% random stuff
\sloppy\sloppypar\raggedbottom\frenchspacing\thispagestyle{empty}

\begin{document}

\section*{NYU Physics I---Problem Set 6}

Due Thursday 2016 October 20 at the beginning of lecture.

\paragraph{\problemname~\theproblem:}\refstepcounter{problem}\label{elastic}%
Complete the elastic collision problem started in class.

\paragraph{\problemname~\theproblem:}\refstepcounter{problem}%
Some kind of standard work-energy problem? Could be the pulley and table problem revisited.

\paragraph{\problemname~\theproblem:}\refstepcounter{problem}%
Hanging sign problem

\paragraph{Extra \problemname\ (will not be graded for credit):}%
Re-do \problemname~\ref{elastic}, but now for the left-hand block having
mass $M$ and the right-hand block having mass $m\ll M$; that is, solve
the extreme mass-ratio problem. For initial velocities, use $v$ for
the big block and $0$ for the small block. Then draw the before and
after pictures in the lab and center-of-mass frames, just as you did
in \problemname~\ref{elastic}.

\end{document}
