\documentclass[12pt]{article}
\usepackage{graphics}
\begin{document}

\section*{NYU Physics 1---Final Exam}

\vfill

\paragraph{Name:} ~

\paragraph{email:} ~

\paragraph{recitation:} ~

\vfill

This exam consists of six problems.  Write only in this booklet.  Be
sure to show your work.

\vfill

\emph{You must ensure that the proctor checks off your name on the
attendence sheet when you hand in your exam.  Your exam will
\textbf{not} be graded if your name is not checked off.}

\clearpage

\section*{Problem 1}

(a) A block of mass $M$ slides down a plane inclined at an angle
$\theta$ to the horizontal.  There is a coefficient of sliding
friction $\mu$ between block and plane.  What is the magnitude of the
acceleration $a$ of the block down the plane?  Be sure to draw a
free-body diagram.

\vfill

(b) A \emph{different} block of mass $M$ sits on the same plane.  In
this case, the frictional force is so great that the block does not
slide down the plane.  What is the magnitude $f$ and direction of the
frictional force?  Try to give your answer in terms of $M$, $g$ and
$\theta$ only.

\vfill ~

\clearpage

\section*{Problem 2}

A block of mass $M$, initially at rest, hangs from a massless string.
A bullet of mass $m$ travelling at speed $v$ lodges in the block.  The
block (now of mass $M+m$) recoils and swings on the string.  To what
maximum height $h_\mathrm{max}$ does the block swing, relative to its
starting position?

\clearpage

\section*{Problem 3}

A Japanese Shinkansen (bullet train) (of mass $M=2\times
10^5~\mathrm{kg}$), moving at $v_S=250~\mathrm{km\,h^{-1}}$ with
respect to the tracks, hits and collides elastically with a superball
(of mass $m=4~\mathrm{g}$), which is initially at rest.  The front
face of the train is inclined at an angle of $\theta=45~\mathrm{deg}$
to the horizontal, as shown.
\\ \rule{0.1\textwidth}{0pt}
\resizebox{0.8\textwidth}{!}{\includegraphics{../mp/shinkansen.eps}}

(a) Draw a diagram of the ball--train system, just before the
collision, in the center-of-mass rest frame, ie, as observed by an
observer at rest with respect to the center of mass.  Clearly show the
velocities of the ball and train in this frame.

\vfill

(b) What is the final speed and direction of the ball, immediately
after the collision, in the rest frame of the tracks (ie, \emph{not}
in the center-of-mass frame)?

\vfill ~

\clearpage

\section*{Problem 4}

(a) Why does a ball roll down an inclined ramp more slowly than a
frictionless block would slide down the same ramp?  Give a concise,
verbal, physical explanation.

\vfill

(b) Imagine rolling a solid spherical wooden ball and a thin steel
hoop down a plane inclined at an angle of 25~deg to the horizontal.
The wooden ball has mass $m=1~\mathrm{kg}$ and radius
$R=7~\mathrm{cm}$ and the steel hoop has mass $m=5~\mathrm{kg}$ and
radius $R=15~\mathrm{cm}$.  Assume that they both roll without
slipping.  In the absence of air resistance, which will roll down the
plane with greater acceleration?  Explain your answer with a concise,
verbal, physical explanation, backed-up with formulae \emph{only if
necessary.}  (It shouldn't be necessary.)

\vfill ~

\clearpage

\section*{Problem 5}

A ``seconds'' pendulum has a period of exactly 2.0 seconds.

(a) If $g=9.793~\mathrm{m\,s^{-2}}$ in Austin Texas, how long should a
Texan make her or his seconds pendulum?

\vfill

(b) If $g$ is 0.1~percent larger in Paris, France, than in Austin, how
different, in percent, should the length of a Parisian's seconds
pendulum be than a Texan's?  Should it be longer or shorter?
\emph{Note:} You can answer this question without getting part (a)
correct.

\vfill ~

\clearpage

\section*{Problem 6}

Consider the following attractive radial force law, which is
\emph{very different} from Newton's law of gravity:
\begin{equation}
F = \frac{k\,M\,m}{r^3}
\end{equation}
Where $k$ is a constant.  Consider a body of mass $m$ orbiting in a
circular orbit of radius $R$ around another body of mass $M$, with
$m\ll M$, according to the above force law.

(a) What are the dimensions of the constant $k$, in terms of kg, m and
s?

\vfill

(b) What is the period of the orbit $T$ as a function of radius $R$?

\vfill

(c) What is the potential energy of the mass $m$ at radius $R$,
relative to the potential energy at infinity?

\vfill ~

\clearpage

[This page intentionally left blank for calculations or other work.]

\end{document}
