\documentclass[12pt]{article}
\usepackage{amssymb}
\usepackage{amsfonts}
\usepackage{epsfig,latexsym}
\voffset -.7cm
\hoffset -1.5cm
\textheight 22cm
\textwidth 16cm
\def\dspace{\baselineskip = .30in}

\def\beq{\begin{equation}}
\def\eeq{\end{equation}}
\def\be{\begin{eqnarray}}
\def\ee{\end{eqnarray}}

\begin{document}
\begin{center}
{\bf\large Homework 10.}
\end{center}

{\bf Problem 1.}

Total mass of the car $M=50\;g$, mass of each of 4 wheels is  $m=1\;g$, radius of a wheel
is $r=2\;mm$. No slipping means that $r\omega=v$, where $\omega$ is angular speed of
wheels and $v$ is speed of car. Kinetic energy of the car is
$$K=\frac{Mv^2}{2}+4\frac{I\omega^2}{2}=\frac{Mv^2}{2}+4\frac{mr^2/2 (v/r)^2}{2}=\frac{Mv^2}{2}+2\frac{mv^2}{2},$$
where we have used that moment of inertia of a wheel is $I=mr^2/2$ and no slip condition
$\omega=v/r$. The potential energy of the car is $P=-mgx\sin\alpha$, where $x$ is coordinate
along the slope (farther down bigger $x$), and $\alpha =20^o$. Thus, for total energy
we have
$$E=K+P=\frac{Mv^2}{2}+2\frac{mv^2}{2}-mgx\sin\alpha=Const$$
Let's differentiate energy with respect to time
$$\frac{d}{dt}E=Mv\frac{dv}{dt}+2mv\frac{dv}{dt}-mg\sin\alpha\frac{dx}{dt}=0,\;\;\mbox{or }\;\;Mv a+2mv a-mg\sin\alpha v=0,$$ 
here we used the fact that $a\equiv dv/dt$ and $v\equiv dx/dt$ . So we get for acceleration
$a=mg\sin\alpha/(M+2m)$. If we'd skip inertia of the wheels, we'd get $a=mg\sin\alpha/M$.
Finally we can write  $a_{car}/a_{frictionless}=M/(M+2m)$.
\\

{\bf Problem 3.}

Hooke's law is $F=k\Delta x$, thus we can find $k=F/\Delta x=200\;N/m$.
From energy conservation $mgh_{max}=k x^2/2$, where $ x=h_{max}-h_{b}$, $h_b$ is
length of the bungy cord at rest. Solving quadratic equation we get
$$h_{max}=h_b+\frac{mg}{k}\pm\sqrt{\frac{mg}{k}\left(2 h_b+\frac{mg}{k}\right)}$$
The relevant solution is with plus sign.

The first 5m jumper gets in $t_1=\sqrt{2h_b/g}$.
Afterwards we have equation of motion $m\ddot{x}=-kx+mg$, or $\ddot{x}+\omega^2 (x-mg/k)=0$,
where $\omega=\sqrt{k/m}$ is frequency of oscillation. The solution is
$x-mg/k=A\cos(\omega t +\phi)$, $v=dx/dt=-A\omega\sin(\omega t +\phi)$.
Let's use new variable $y=x-mg/k,\;\;dy/dt=dx/dt$, it is zero at the point  of equilibrium $x=mg/k$
or in other words $h_b+mg/k$ below the bridge.
Starting time from equilibrium point we get $0=A\cos(\phi)$ and $v_0=-A\omega\sin(\phi)$, 
from here $\phi=-\pi/2$, $A=v_0/\omega$, where $v_0$ is speed at equilibrium point defined from
conservation of energy $$mg(h_b+mg/k)=k/2\;(mg/k)^2 +mv_0^2/2,\;\rightarrow\;\;v_0=\sqrt{g(2h_b+mg/k)}$$
Thus total time is sum of time $t_1$ from $0$ to $h_b$ plus $t_2$ from $h_b$ to $h_b+mg/k$ plus $t_3=T/4$, where $T$ is  period of oscillation.
At time  $-t_2$ $y=-mg/k=A\sin\omega(- t_2)$, we get $t_2=1/\omega \arcsin mg/(kA)$.
$t_3=T/4=\pi/2 \sqrt{m/k}$. Thus for total time we get $t=\sqrt{2h_b/g}+ \sqrt{m/k} \arcsin mg/(kA) +\pi/2 \sqrt{m/k}$.

{\bf Problem 2.}

\begin{figure}[h]
\begin{center}
\epsffile{../eps/hogg10_gr1.eps}
\end{center}
\end{figure}


\end{document}
