\documentclass[12pt]{article}
\newcounter{problem}
\begin{document}\thispagestyle{empty}

\section*{NYU General Physics 1---Problem set 14}

\paragraph{Problem~\theproblem:}\refstepcounter{problem}%
``Surface tension'' has dimensions of force per unit length or energy
per unit area.  It is to a surface as pressure is to a volume
(pressure has dimensions of force per unit area or energy per unit
volume).  It is to a surface as tension is to a string (tension has
dimensions of force or energy per unit length).

\textsl{(a)} If you have a spherical object of radius $R$ surrounded
by a stretched surface with surface tension $\sigma$, what, by dimensional
analysis, do you expect to be the \emph{pressure} inside the object?

\textsl{(b)} The pressure inside a living cell is set by salinity.  If
the cell is in a saline-rich environment, water leaves the cell (for
the environment) and the cell shrinks.  If the cell is in a
saline-poor environment, water enters the cell.  In a very low-salt
environment, the cell keeps taking on water until the pressure is set
by the surface tension.  Using your result from part \textsl{(a)},
estimate the pressure in a 5-micron-radius cell in a low-salt
environment if the surface tension of its surface is roughly 5
nanonewton per nanometer.  Get your answer in atmospheres.

\textsl{(c)} Look up the surface tension of pure water and estimate
the gauge pressure inside a typical raindrop.  I say ``gauge
pressure'' because I mean the \emph{additional} pressure provided by
the surface over and above atmospheric pressure.  \emph{In both this
  part and the previous, you are computing gauge pressures.}

\paragraph{Problem~\theproblem:}\refstepcounter{problem}%
\textsl{(a)} Re-find the ram-pressure and viscous-drag forces we have
found previously by dimensional analysis.  Reminder: Ram pressure
depends only on the size of the object, its velocity, and the mass
density $\rho$ of the fluid; viscous-drag depends in addition on the
kinetic viscosity $\nu$ (look it up!).

\textsl{(b)} At roughly what speed does ram pressure exceed stokes
drag for a typical automobile driving through air?  You will need to
look up the viscosity of air.  Check the dimensions of the numbers you
are using!  \emph{Hint: You might want to make the ``spherical cow''
  approximation for the car; this is just rough after all!}

\textsl{(c)} What about for a red blood cell moving through plasma?
Look up what you need to know on the internet; that will include the
sizes of the cells and the properties of plasma.

\textsl{(d)} Do red blood cells float in plasma or do they sink?
Either way, find the velocity at which the viscous drag force, the
gravitational force, and the buoyant force sum to zero.  What would
be different about this velocity if the blood cells and plasma were in
a centrifuge spinning very fast?

\paragraph{Problem~\theproblem:}\refstepcounter{problem}%
[\textsl{optional}] Pick up a copy of \textit{The Making of the Atomic
  Bomb} by Richard Rhodes and read it over the break.  Send Prof Hogg
an email with your reactions to it sometime next term.  \emph{Extra
  hard:} Use what you find in the book to calculate (very roughly) what's called
critical mass (the minimum amount of uranium 235 it takes to sustain a
nuclear fission reaction).  Check your answer on Wikipedia.

\end{document}
