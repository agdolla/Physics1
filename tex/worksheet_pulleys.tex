\documentclass[12pt]{article}
\usepackage{url, graphicx}

% page layout
\setlength{\topmargin}{-0.25in}
\setlength{\textheight}{9.5in}
\setlength{\headheight}{0in}
\setlength{\headsep}{0in}

% problem formatting
\newcommand{\problemname}{Problem}
\newcounter{problem}

% math
\newcommand{\dd}{\mathrm{d}}

% primary units
\newcommand{\rad}{\mathrm{rad}}
\newcommand{\kg}{\mathrm{kg}}
\newcommand{\m}{\mathrm{m}}
\newcommand{\s}{\mathrm{s}}

% secondary units
\renewcommand{\deg}{\mathrm{deg}}
\newcommand{\km}{\mathrm{km}}
\newcommand{\mi}{\mathrm{mi}}
\newcommand{\h}{\mathrm{h}}
\newcommand{\ns}{\mathrm{ns}}
\newcommand{\J}{\mathrm{J}}
\newcommand{\eV}{\mathrm{eV}}
\newcommand{\W}{\mathrm{W}}

% derived units
\newcommand{\mps}{\m\,\s^{-1}}
\newcommand{\mph}{\mi\,\h^{-1}}
\newcommand{\mpss}{\m\,\s^{-2}}

% random stuff
\sloppy\sloppypar\raggedbottom\frenchspacing\thispagestyle{empty}

\begin{document}

\section*{NYU Physics I---pulleys and blocks}

The instructor will draw two machines on the board. The first machine
consists of three blocks, of masses 7, 4, and $3\,\kg$, and two
pulleys (and three strings). The second machine consists of two
blocks, of masses 4 and $3\,\kg$, and one pulley (and two strings).

Choose a partner and work in pairs. Each member of each pair should
write down---on paper---their answers to each part of this worksheet.
You don't have to agree with your partner, but you must discuss and
understand one another.

\paragraph{\theproblem}\refstepcounter{problem}%
In the three-block machine, do you expect the $7\,\kg$ block to
accelerate upwards, or downwards, or not accelerate? You might have a
quick answer to this and want to move on, but talk it out. Consider
extreme scenarios (that is, consider changing the masses of the 3 and
$4\,\kg$ blocks). Are you \emph{sure} about your prediction?

\paragraph{\theproblem}\refstepcounter{problem}%
For the two-block machine, draw free-body diagrams for each of the two
blocks, and for the pulley. Make sure you can justify what you have
drawn. Make sure you are happy with your free-body diagram for the
pulley.

\paragraph{\theproblem}\refstepcounter{problem}%
We are going to assume that the strings in this problem are
inextensible. Why are we going to assume that? What does it do for us?

\paragraph{\theproblem}\refstepcounter{problem}%
We are going to assume that the strings and pulleys are massless. What
does that do for us? How does that help us?

\paragraph{\theproblem}\refstepcounter{problem}%
We are going to assume that the pulleys are frictionless. How does
\emph{that} help us?

\paragraph{\theproblem}\refstepcounter{problem}%
What is the relationship between the accelerations of the two blocks?
Set up a coordinate system and describe this relationship with an
equation.

\paragraph{\theproblem}\refstepcounter{problem}%
Put it all together and solve for the accelerations for the two
blocks, and the tensions in the two strings.

\paragraph{\theproblem}\refstepcounter{problem}%
What is the tension in the top string, and how does it compare to the
total mass times the acceleration due to gravity? Does this change
your thinking about the first question on this worksheet?

\end{document}
