\documentclass[12pt]{article}
\newcommand{\N}{\mathrm{N}}
\newcommand{\m}{\mathrm{m}}
\newcommand{\cm}{\mathrm{cm}}
\newcommand{\s}{\mathrm{s}}
\renewcommand{\min}{\mathrm{min}}
\renewcommand{\l}{\ell}
\newcommand{\C}{\mathrm{deg\,C}}
\newcounter{problem}
\begin{document}\thispagestyle{empty}

\section*{NYU General Physics 1---Problem set 13}

\paragraph{Problem~\theproblem:}\refstepcounter{problem}%
In lecture we discussed the pressure gradient in a bus accelerating at
$2\,\m\,\s^{-2}$.

\textsl{(a)} By thinking about the size of the interior of the bus,
the density of air, and so on, compute roughly the \emph{horizontal}
pressure difference inside the bus.  That is, how much larger is the
pressure at the back of the bus than the front of the bus?  (If it is
at all confusing to you that the equal-pressure surfaces are slanted,
consider the pressure difference between a point at the front of the
bus and a point at the back of the bus, where both points are at the
same height above ``sea level''.)  Give your answer in both
$\N\,\m^{-2}$ and atmospheres.

\textsl{(b)} What things did you have to assume to give your answer?
Why did we assume that the windows on the bus are all closed?
What do you think happens if the windows all along the bus
\emph{aren't} closed?

\paragraph{Problem~\theproblem:}\refstepcounter{problem}%
Blood flows through the aorta at a volumetric rate of something like
$5\,\l\,\min^{-1}$.

\textsl{(a)} If the aorta has a diameter of $3.5\,\cm$, at what speed
$v$ does the blood flow?  Did you have to make any assumptions to
answer that?

\textsl{(b)} Imagine that, because of a pathology, over a $20\,\cm$
length, the aorta narrows to $2.5\,\cm$ in diameter.  What is the
velocity change $\Delta v$ from one end of this to the other?

\textsl{(c)} Consider a little cube of blood in the part of the aorta
that is getting narrower.  Give the cube a side length $\Delta x$ and
note that it is \emph{accelerating}.  What---qualitatively---does this
mean about the pressure in the aorta?  Ignore gravity for this
problem; imagine that all blood flow is driven by blood pressure (this
is true if the patient is lying down).

\textsl{(d)} In preparation for part \textsl{(e)}, find a combination
of speed $v$ and mass density (mass per volume) $\rho$ that has
dimensions of pressure.  Look up the density of blood and compute the
pressure corresponding to the velocity $v$ you found in
part \textsl{(a)}.

\textsl{(e)} Compute the pressure change from one end of this
narrowing aorta to the other end, by thinking about the pressure
gradient at each point that produces the necessary local acceleration.
\emph{Note: This is not easy!}  Compare your result to normal mean
human blood pressure.  Discuss with your friends and colleagues.

\paragraph{Problem~\theproblem:}\refstepcounter{problem}%
[\textsl{optional}] What is the most expensive ingredient of a
standard American Thanksgiving dinner, \emph{by weight}?  (That is, by
dollars per pound.)  How does this value compare to the dollars per
pound of gold we computed in Problem~Set~1?  What does all this have
to do with the history of Western Civilization?

\end{document}
