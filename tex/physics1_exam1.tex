\documentclass[12pt]{article}
\usepackage{url, graphicx}

% page layout
\setlength{\topmargin}{-0.25in}
\setlength{\textheight}{9.5in}
\setlength{\headheight}{0in}
\setlength{\headsep}{0in}

% problem formatting
\newcommand{\problemname}{Problem}
\newcounter{problem}

% math
\newcommand{\dd}{\mathrm{d}}

% primary units
\newcommand{\rad}{\mathrm{rad}}
\newcommand{\kg}{\mathrm{kg}}
\newcommand{\m}{\mathrm{m}}
\newcommand{\s}{\mathrm{s}}

% secondary units
\renewcommand{\deg}{\mathrm{deg}}
\newcommand{\km}{\mathrm{km}}
\newcommand{\mi}{\mathrm{mi}}
\newcommand{\h}{\mathrm{h}}
\newcommand{\ns}{\mathrm{ns}}
\newcommand{\J}{\mathrm{J}}
\newcommand{\eV}{\mathrm{eV}}
\newcommand{\W}{\mathrm{W}}

% derived units
\newcommand{\mps}{\m\,\s^{-1}}
\newcommand{\mph}{\mi\,\h^{-1}}
\newcommand{\mpss}{\m\,\s^{-2}}

% random stuff
\sloppy\sloppypar\raggedbottom\frenchspacing\thispagestyle{empty}

\begin{document}

\noindent
Name: \rule[-1ex]{0.55\textwidth}{0.1pt}
NetID: \rule[-1ex]{0.2\textwidth}{0.1pt}

\section*{NYU Physics I---Term Exam 1}

\paragraph{\problemname~\theproblem:}\refstepcounter{problem}%
(From Problem Set 1, Problem~3)
Given a mass $M$, a length $h$, a velocity $v$ and an acceleration
$g$, give two qualitatively different expressions that have units of
energy.

\vfill

\paragraph{\problemname~\theproblem:}\refstepcounter{problem}%
(From Problem Set 2, Problem~3)
Consider a stone thrown (at $t=0$) precisely upwards (in the $y$
direction, for definiteness) at $1.0\,\m\,\s^{-1}$, with an initial
position (launch point) at $y=0$.  Ignore air resistance! Make one
very careful plot of the vertical velocity $v_y$ of the stone as a
function of time $t$ for the duration $0<t<0.3\,\s$. Use
$g=10\,\m\,\s^{-2}$.

\vfill

\paragraph{\problemname~\theproblem:}\refstepcounter{problem}%
(From Lecture, 2017-09-07)
The radius of the Moon is about 1/6 the radius of the Earth, but
the Moon has similar composition to the Earth. What (very roughly)
is the mass of the Moon? Recall that the mass of the Earth was
about $6\times 10^{24}\,\kg$.

\vfill

\clearpage
\paragraph{\problemname~\theproblem:}\refstepcounter{problem}%
(From Lecture, 2017-09-12)
We considered the velocity of a stone at two points on a trajectory.
That trajectory was in gravity with no air resistance.
Name one thing about the velocity that changed with time between
the two points.
Name one thing about the velocity that stayed the same.

\vfill

\paragraph{\problemname~\theproblem:}\refstepcounter{problem}%
(From Lecture, 2017-09-19)
We did a problem of a block sitting on a horizontal floor. We
found that the normal force was the same magnitude as gravity (but
pointed in the opposite direction). Would the normal force on the
block be larger, smaller, or stay the same, if the block was sitting
on the floor of an elevator accelerating upwards? Say why in one sentence.

\vfill

\paragraph{\problemname~\theproblem:}\refstepcounter{problem}%
(From Problem Set 1, Problem 4)
What is the terminal velocity of a cube of rock, one meter on a side,
falling through the air, roughly?

\vfill
~
\end{document}
