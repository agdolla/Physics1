\documentclass[12pt]{article}
\usepackage{url, graphicx}

% page layout
\setlength{\topmargin}{-0.25in}
\setlength{\textheight}{9.5in}
\setlength{\headheight}{0in}
\setlength{\headsep}{0in}

% problem formatting
\newcommand{\problemname}{Problem}
\newcounter{problem}

% math
\newcommand{\dd}{\mathrm{d}}

% primary units
\newcommand{\rad}{\mathrm{rad}}
\newcommand{\kg}{\mathrm{kg}}
\newcommand{\m}{\mathrm{m}}
\newcommand{\s}{\mathrm{s}}

% secondary units
\renewcommand{\deg}{\mathrm{deg}}
\newcommand{\km}{\mathrm{km}}
\newcommand{\mi}{\mathrm{mi}}
\newcommand{\h}{\mathrm{h}}
\newcommand{\ns}{\mathrm{ns}}
\newcommand{\J}{\mathrm{J}}
\newcommand{\eV}{\mathrm{eV}}
\newcommand{\W}{\mathrm{W}}

% derived units
\newcommand{\mps}{\m\,\s^{-1}}
\newcommand{\mph}{\mi\,\h^{-1}}
\newcommand{\mpss}{\m\,\s^{-2}}

% random stuff
\sloppy\sloppypar\raggedbottom\frenchspacing\thispagestyle{empty}

\begin{document}

\section*{NYU Physics I---Problem Set 14}

Due Thursday 2016 December 15 at the beginning of lecture.

\paragraph{\problemname~\theproblem:}\refstepcounter{problem}%
From the notes at \url{http://cosmo.nyu.edu/hogg/sr/},
Problem 4--8.

\paragraph{\problemname~\theproblem:}\refstepcounter{problem}%
From the notes at \url{http://cosmo.nyu.edu/hogg/sr/},
Problem 4--11.

\paragraph{\problemname~\theproblem:}\refstepcounter{problem}%
From the notes at \url{http://cosmo.nyu.edu/hogg/sr/},
Problem 6--10.

\paragraph{Extra Problem (will not be graded for credit):}%
\textsl{(a)}~Forgetting about Special Relativity, and assuming just
Newtonian mechanics, compute how long you would have to accelerate at
acceleration $g=10\,\mpss$ in order to reach the speed of light.

\textsl{(b)}~A relativistically correct contstant-acceleration
trajectory on a spacetime diagram is a hyperbola, where both
asymptotes are 45-degree lines (null trajectories. Find a formula
(position $x$ as a function of time $t$) for this hyperbola,
constrained to have acceleration $g$ at small times.

\textsl{(c)}~Show that this trajectory is unchanged under the Lorentz
transformation. That is, show that if you boost in the $x$ direction,
the trajectory translates onto itself (except, possibly, for a small
shift in the $x$ or $t$ direction).

\end{document}
