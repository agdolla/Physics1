\documentclass[12pt]{article}
\usepackage{graphics}
\newcommand{\s}{\mathrm{s}}
\begin{document}
\newcounter{problem}
\thispagestyle{empty}

\section*{NYU Physics 1---Problem set 2}

Due Tuesday 2009 September 29 at the beginning of lecture.

\paragraph{Problem~\theproblem:}\refstepcounter{problem}%
Chabay \& Sherwood problem 2.P.43.  Once you have completed the book
problem, look up the formula for kinetic energy (yes, we are getting
ahead of ourselves) and answer this: Is there more kinetic energy in
the system of bullet+block immediately before the collision or
immediately after, or is it conserved?  Explain your result,
physically.

\paragraph{Problem~\theproblem:}\refstepcounter{problem}%
What is the mean acceleration $a$ of a dragster (ie, a drag-racing
automobile) that can travel 1/4~mi in 5.5~s, starting from a dead
stop?  Assume that the dragster accelerates with constant acceleration
throughout the 5.5~s (not a terrible assumption, but not a good one
either).  Give your answer in terms of the gravitational acceleration
$g$.  Does your answer seem reasonable?  What do you predict, under
the constant-acceleration assumption, for the final speed
$v_\mathrm{f}$ of the dragster as it crosses the finish line?  Give
your answer in $\mathrm{mi\,h^{-1}}$.  Search the web for the current
world-record 1/4~mi drag-race time and final speed.

\paragraph{Problem~\theproblem:}\refstepcounter{problem}%
In a 200-m race, the winner crosses the halfway mark (ie, 100~m) at
$t_{1/2}=10.12$~s and the finish line at $t_{\rm f}=19.32$~s.  If you
make a kinematic model of the runner's behavior as (i)~constant
acceleration $a$ from rest for the period $0<t<t_a$ followed by
(ii)~constant speed $v=a\,t_a$ during the period $t_a<t<t_{\rm f}$,
what are $a$ and $t_a$?

\emph{Hint: Draw a graph of $v(t)$ before you start writing equations.}

\emph{Hint: Once you have a graph, consider the question: is $t_a$
going to be before or after $t_{1/2}$?  Make a quantitative argument.}

\end{document}

\paragraph{Problem~\theproblem}\refstepcounter{problem}%
In class, we decided that the relevant physical quantities for
assessing the relative importance of air resistance are the mass $m$
of the falling object, the acceleration due to gravity $g$, the
distance to fall $h$, the cross-sectional area $A$ of the falling
object, and the density of air $\rho$.

\textsl{(a)}~Use these quantities to create the simplest possible
dimensionless number (that is, all your units must cancel out).  Make
the dimensionless number that increases as air resistance becomes more
important, as it does when the area gets bigger, the fall is longer,
and the mass is smaller.

\textsl{(b)}~Now plug in numbers for a window-washer's bucket of
water, falling three stories.  Do you think air resistance matters not
at all, a little bit, a lot, or it completely dominant?
