\documentclass[12pt]{article}
\newcommand{\kg}{\mathrm{kg}}
\newcommand{\m}{\mathrm{m}}
\newcommand{\cm}{\mathrm{cm}}
\newcommand{\s}{\mathrm{s}}
\newcommand{\ms}{\mathrm{ms}}
\begin{document}
\newcounter{problem}
\thispagestyle{empty}

\section*{NYU General Physics 1---Problem set 5}

\paragraph{Problem~\theproblem:}\refstepcounter{problem}%
A solid rubber ball of radius $1.5\,\cm$ is dropped from a height of
$1\,\m$ onto a hard surface.  It bounces.  The objective of this
problem is to figure out the magnitude of the contact force on the
ball \emph{during} the bounce.

\textsl{(a)} The contact force pushing the ball upwards will be far
larger in magnitude than the gravitational force pushing the ball
downwards.  Explain why in words.

\textsl{(b)} Estimate the mass $m$ of the ball and also the speed $v$
at which the ball will be traveling just before it hits the floor.
Use the internet or the library for the density of rubber.

\textsl{(c)} If the ball is in contact with the floor for about
one~millisecond, and if it bounces back upwards with about the ``equal
and opposite'' velocity to that it had before the contact, what is the
mean acceleration $\vec{a}$ of the ball during the bounce?

\textsl{(d)} What is the implied mean contact force $\vec{N}$ during
the bounce?

\textsl{(e)} I said the ball will be in contact with the floor for
$\sim 1\,\ms$.  How could you estimate this?  One option is to
consider the time it takes a sound wave to traverse the diameter of
the ball.  Look up the speed of sound in rubber and estimate this
time; is $1\,\ms$ reasonable?  Also, can you think of other ways to
make an estimate?

\paragraph{Problem~\theproblem:}\refstepcounter{problem}%
You are in the passenger seat of a car traveling fast in a straight
line.  You have your seatbelt on.  The driver slams on the brakes, so
you are accelerating with a magnitude of $12\,\m\,\s^{-2}$.

\textsl{(a)} If your mass is $50\,\kg$, calculate and also draw all of
the forces acting on your body during the acceleration (which most
normal people would call ``deceleration'').

\textsl{(b)} If the car plus contents has a mass of $1300\,\kg$, what
is the total force of the car on the road, from all four tires?  Give
direction and magnitude.

\textsl{(c)} If the road is slippery, the car will go into a skid.
What is the critical coefficient of friction $\mu$ below which the car
will slide?

\paragraph{Problem~\theproblem:}\refstepcounter{problem}%
When you are on a roller coaster, you feel heavier when the roller
coaster goes through the bottom of a dip, and you feel lighter when
the roller coaster goes over the crest of a hill.

\textsl{(a)} What force on your body in the roller coaster is larger
at the bottom of a dip and what force is smaller at the top of the
hill?  Hint: It isn't the gravitational force!  Be sure to be able to
explain \emph{why} the force is different in the different cases.

\textsl{(b)} The astronauts in the Space Station feel weightless; why?
Hint: It isn't because the gravitational force on them is small!

\end{document}
