\documentclass[12pt]{article}
\usepackage{url, graphicx}

% page layout
\setlength{\topmargin}{-0.25in}
\setlength{\textheight}{9.5in}
\setlength{\headheight}{0in}
\setlength{\headsep}{0in}

% problem formatting
\newcommand{\problemname}{Problem}
\newcounter{problem}

% math
\newcommand{\dd}{\mathrm{d}}

% primary units
\newcommand{\rad}{\mathrm{rad}}
\newcommand{\kg}{\mathrm{kg}}
\newcommand{\m}{\mathrm{m}}
\newcommand{\s}{\mathrm{s}}

% secondary units
\renewcommand{\deg}{\mathrm{deg}}
\newcommand{\km}{\mathrm{km}}
\newcommand{\mi}{\mathrm{mi}}
\newcommand{\h}{\mathrm{h}}
\newcommand{\ns}{\mathrm{ns}}
\newcommand{\J}{\mathrm{J}}
\newcommand{\eV}{\mathrm{eV}}
\newcommand{\W}{\mathrm{W}}

% derived units
\newcommand{\mps}{\m\,\s^{-1}}
\newcommand{\mph}{\mi\,\h^{-1}}
\newcommand{\mpss}{\m\,\s^{-2}}

% random stuff
\sloppy\sloppypar\raggedbottom\frenchspacing\thispagestyle{empty}

\begin{document}

\section*{NYU Physics I---Problem Set 5}

Due Thursday 2018 October 11 at the beginning of lecture.

\paragraph{Problem~\theproblem:}\refstepcounter{problem}%
Where is the center-of-mass of the Earth--Moon system? Give your
answer in terms of Earth radii relative to the center of the
Earth. Where is the center-of-mass of the Solar System? Assume (not a
terrible approximation!) that the Solar System contains only the Sun
and Jupiter. Give your answer in terms of Solar radii relative to the
center of the Sun.

\paragraph{Problem~\theproblem:}\refstepcounter{problem}%
NYC had a hot summer in 2018, with most buildings running air
conditioning on a thermostat continuously. To save energy, NYU asked
their employees to conserve energy in various ways, some of which we
might take issue with. Here's an uncontroversial one: You should take
the stairs, not the elevator.

\emph{But is that uncontroversial?}  In the parts that follow, make
sure you are explicit about your assumptions, and that your
assumptions are reasonable. If you are concerned with reasonableness,
do some experiments and make some observations.

\textsl{(a)} How much potential energy does a typical NYU employee
generate if he or she climbs 10 stories of stairs? (Make use of things
you learned or computed in last week's problem set.)

\textsl{(b)} For every kcal (look up the calorie and the kcal; what
are called ``calories'' in nutrition are actually kcal) of mechanical
energy generated, a healthy, fit human also generates about 7 times
more in metabolic load; that is, you burn some 8 times your
mechanically computed power in food calories. How many kcal did the
employee burn going up the 10 stories?

\textsl{(c)} Now imagine (correctly) that all that metabolic energy
gets dumped into the building atmosphere and needs to get corrected by
the building air conditioning. The very finest air-conditioning units are
efficient; they produce about a BTU (or kcal or J or kWh) of heat
for every BTU (or kcal or J or kWh) of cooling they do. Under these
assumptions, what load does an employee puts on the AC by climbing this far?

\textsl{(d)} \textbf{[Not for credit!]} For \emph{extremely deep reasons}, an
elevator can't be perfectly efficient either. Can you think of some of
these reasons? Prof Hogg knows at least three unbeatable physical
constraints. The relevant question for us is: Does an elevator burn
more than eight times the mechanical work it is doing, and does it
drop that load inside the building AC system? If you want, do some
research to understand this. The ``nicest'' assumption you can make
about the elevators is that they are always crowded, so you are only
looking at the \emph{marginal} cost of bringing up one more person.

\paragraph{Problem~\theproblem:}\refstepcounter{problem}%
A student of mass $m_\mathrm{student}=80\,\kg$ stands at rest next to
a block of ice of mass $m_\mathrm{ice}=320\,\kg$, also at rest, on a
frictionless frozen lake.  The student pushes on the block until the
block is moving away from the student at $1.5\,\mps$ (that is, until
$\left|\vec{v}_\mathrm{ice}-\vec{v}_\mathrm{student}\right|=1.5\,\mps$).
How much work did the student do?  Give your answer in $\J$.  Don't
forget to conserve linear momentum!  \emph{Hint:} All that work went into
kinetic energy.  \emph{Another hint:} One of the hard things about
this problem is that I am giving you the \emph{relative} velocity and
not the absolute velocity.  How are you going to deal with that?
Spend some time visualizing the problem before writing equations.

\paragraph{Problem~\theproblem:}\refstepcounter{problem}%
A machine at a packaging facility places stationary packages of mass
$m$ onto a horizontal conveyor belt that is moving packages steadily
and horizontally at speed $v$. Once placed on the belt, the packages
start moving at speed $v$; that is, they are rapidly accelerated.

\textsl{(a)} What is the momentum change for each package as it
gets placed on to the belt, and what is the kinetic energy change?

\textsl{(b)} Imagine that the machine places packages steadily onto
the belt, with time intervals $T$ between packages. Compute the
average mechanical power required by the belt by dividing the kinetic
energy per package by the time interval between packages.

\textsl{(c)} Now compute the average force the belt is applying to the
packages by dividing the momentum per package by the time interval
between packages. This is possible, because force is momentum per unit
time (if that is a surprise, do some library or web research).

\textsl{(d)} Power is force times velocity (dot product, really); so
the force answer can be turned into a power answer and compared to the
power you computed above.

\textsl{(e)} Do you have a discrepancy? If so, why? Which answer is
more correct?

\end{document}
