\documentclass[12pt]{article}
\usepackage{url, graphicx}

% page layout
\setlength{\topmargin}{-0.25in}
\setlength{\textheight}{9.5in}
\setlength{\headheight}{0in}
\setlength{\headsep}{0in}

% problem formatting
\newcommand{\problemname}{Problem}
\newcounter{problem}

% math
\newcommand{\dd}{\mathrm{d}}

% primary units
\newcommand{\rad}{\mathrm{rad}}
\newcommand{\kg}{\mathrm{kg}}
\newcommand{\m}{\mathrm{m}}
\newcommand{\s}{\mathrm{s}}

% secondary units
\renewcommand{\deg}{\mathrm{deg}}
\newcommand{\km}{\mathrm{km}}
\newcommand{\mi}{\mathrm{mi}}
\newcommand{\h}{\mathrm{h}}
\newcommand{\ns}{\mathrm{ns}}
\newcommand{\J}{\mathrm{J}}
\newcommand{\eV}{\mathrm{eV}}
\newcommand{\W}{\mathrm{W}}

% derived units
\newcommand{\mps}{\m\,\s^{-1}}
\newcommand{\mph}{\mi\,\h^{-1}}
\newcommand{\mpss}{\m\,\s^{-2}}

% random stuff
\sloppy\sloppypar\raggedbottom\frenchspacing\thispagestyle{empty}

\begin{document}

\section*{NYU Physics I---Problem Set 13}

Due Thursday 2017 December 7 at the beginning of lecture.

\paragraph{\problemname~\theproblem:}\refstepcounter{problem}%
Get a sense of the speed of light by computing two things:

\textsl{(a)} How many times could light go around the equator of the Earth
in a time interval of $1\,\s$?

\textsl{(b)} How long (in ns) does it take light to go $1\,\ft$?

\paragraph{\problemname~\theproblem:}\refstepcounter{problem}%
From the notes at \url{http://cosmo.nyu.edu/hogg/sr/},
Problem 3--4.

\paragraph{\problemname~\theproblem:}\refstepcounter{problem}%
From the notes at \url{http://cosmo.nyu.edu/hogg/sr/},
Problem 2--14.

\paragraph{\problemname~\theproblem:}\refstepcounter{problem}%
\textsl{(a)} What is $\gamma$ to first order in $\beta^2$ for $\beta
<< 1$? That is, construct a Taylor Series for $\gamma$ in terms of
$\beta^2$ and give the zeroth-order term (1) and then the first-order
term.

\textsl{(b)} What are $\beta$ and $\gamma$ for a person walking
(relative to the sidewalk), a driver on the freeway (relative to the
road), a commercial jet (relative to the air), and an astronaut in the
ISS (relative to the center of mass of the Earth)? Use the first-order
expression from part \textsl{(a)} to compute the $\gamma$ values.

\textsl{(c)} Computing the full time dilation effect in gravity is
complicated! However, the pure kinematic part of the time dilation
only depends on $\gamma$. Two twins part. One gets on the ISS for a
year, and one stays on Earth. When they are reunited in a year, how
much younger is the astronaut than the homebody?

\paragraph{Extra Problem (will not be graded for credit):}%
If the total energy (rest mass plus kinetic) of a point particle is
$\gamma\,m\,c^2$, use the result from \problemname~3 above to get an
approximate expression for the kinetic energy at low speeds.

\end{document}
