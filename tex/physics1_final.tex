\documentclass[12pt]{article} \usepackage{url, graphicx}

% page layout
\setlength{\topmargin}{-0.25in}
\setlength{\textheight}{9.5in}
\setlength{\headheight}{0in}
\setlength{\headsep}{0in}

% problem formatting
\newcommand{\problemname}{Problem}
\newcounter{problem}

% math
\newcommand{\dd}{\mathrm{d}}

% primary units
\newcommand{\rad}{\mathrm{rad}}
\newcommand{\kg}{\mathrm{kg}}
\newcommand{\m}{\mathrm{m}}
\newcommand{\s}{\mathrm{s}}

% secondary units
\renewcommand{\deg}{\mathrm{deg}}
\newcommand{\km}{\mathrm{km}}
\newcommand{\mi}{\mathrm{mi}}
\newcommand{\h}{\mathrm{h}}
\newcommand{\ns}{\mathrm{ns}}
\newcommand{\J}{\mathrm{J}}
\newcommand{\eV}{\mathrm{eV}}
\newcommand{\W}{\mathrm{W}}

% derived units
\newcommand{\mps}{\m\,\s^{-1}}
\newcommand{\mph}{\mi\,\h^{-1}}
\newcommand{\mpss}{\m\,\s^{-2}}

% random stuff
\sloppy\sloppypar\raggedbottom\frenchspacing\thispagestyle{empty}

\begin{document}

\noindent
Name: \rule[-1ex]{0.55\textwidth}{0.1pt}
NetID: \rule[-1ex]{0.2\textwidth}{0.1pt}

\section*{NYU Physics I---Final Exam}

\paragraph{\problemname~\theproblem:}\refstepcounter{problem}%
If you have a mass $M$, a length $h$, and a speed $v$, what
combination would have units of force?
%
(from Problem Set~1, \problemname~2)

\vfill

\paragraph{\problemname~\theproblem:}\refstepcounter{problem}%
In a spreadsheet integration of a particle trajectory, at time
$t=0.31\,\s$ the position is $y=0.410\,\m$, the velocity is
$v_y=0.50\,\mps$, and the acceleration is $a_y=-10\,\mps$. What is the
position $y$ and velocity $v_y$ at time $t=0.32\,\s$?
%
(from Problem Set~2, \problemname~1)

\vfill

\paragraph{\problemname~\theproblem:}\refstepcounter{problem}%
Starting at rest, you accelerated at $a_x=1\,\mpss$ for $3\,\s$ and
then traveled at constant speed $v_x$ for the next $10\,s$. How far
did you travel in the $x$-direction in the total $13\,\s$?
%
(from Problem Set~3, \problemname~1)

\vfill
~
\clearpage

\paragraph{\problemname~\theproblem:}\refstepcounter{problem}%
Consider a car driving around a circular banked turn at constant
altitude with the bank set at $15\,\deg$ to the horizontal Draw the
free-body diagram in the case that the car is going around the curve
at exactly the speed such that no lateral friction force is required
at all, and the car neither loses nor gains altitude (it goes in a
perfect, level, circle). That is, draw the free-body diagram for the
frictionless, perfect-driving case.
%
(from Problem Set~4, \problemname~1)

\vfill

\paragraph{\problemname~\theproblem:}\refstepcounter{problem}%
How much potential energy does a typical NYU employee generate if
he or she climbs 10 stories of stairs? State your assumptions.
%
(from Problem Set~5, \problemname~1)

\vfill

\paragraph{\problemname~\theproblem:}\refstepcounter{problem}%
In this hanging sign problem, what is the direction of the force on
the beam from the wall at the pivot? Circle your best option. (From Problem Set~6,
\problemname~3.)\\
(a) straight up\\
(b) up and to the left\\
(c) straight left\\
(d) down and to the left\\
(e) straight down\\
(f) down and to the right\\
(g) straight right\\
(h) up and to the right\\
(i) there is no force\marginpar{\includegraphics[width=1in]{../mp/hanging_sign.pdf}}

\clearpage

\paragraph{\problemname~\theproblem:}\refstepcounter{problem}%
Consider a mass $M$ attached to a spring of \emph{natural}
(equilibrium) length $\ell$, with spring constant $k$, and hanging
from the ceiling. The system is subject to gravity with gravitational
acceleration $g$. What is the new (stretched, longer) equilibrium
length of the string when it is hanging in gravity?
%
(from Problem Set~7, \problemname~2)

\vfill

\paragraph{\problemname~\theproblem:}\refstepcounter{problem}%
A very thin ladder of length $L$ and mass $M$ leans against a vertical
wall, on a horizontal floor, making a \emph{small} angle of $\theta$
with respect to the wall. Imagine that there is a large coefficient of
friction $\mu$ at the floor so that the ladder is in static
equilibrium, but assume that the wall is effectively frictionless.
Draw a free-body diagram for the ladder, showing all forces acting,
and where they act.
%
(from Problem Set~8, \problemname~1)

\vfill

\paragraph{\problemname~\theproblem:}\refstepcounter{problem}%
What is the correction to your weight coming from buoyancy, roughly?
Express it as a fraction of the gravitational force. Is this
correction positive or negative---that is, does it increase or
decrease the weight measured by the scale? State your assumptions.
%
(from Problem Set~9, \problemname~3)

\vfill
~
\clearpage

\paragraph{\problemname~\theproblem:}\refstepcounter{problem}%
For a sphere of mass $m$ moving at speed $v$ with radius $R$ and
angular speed $\omega$, what is the condition (equation, really) known
as ``rolling without slipping''?
%
(from Problem Set~10, \problemname~2)

\vfill

\paragraph{\problemname~\theproblem:}\refstepcounter{problem}%
Find a combination of Newton’s Constant $G$, the speed of light $c$,
and some (arbitrary) mass $M$, that has units of length.
%
(from Problem Set~11, \problemname~3)

\vfill

\paragraph{\problemname~\theproblem:}\refstepcounter{problem}%
Draw a dot representing the Sun, and then sketch around it three
gravitational orbits of the same semi-major axis but eccentricities of
0.0, 0.5, and 0.9.
%
(from Problem Set~12, \problemname~2)

\vfill
~
\clearpage

\paragraph{\problemname~\theproblem:}\refstepcounter{problem}%
A particle lives for $1\times 10^{-9}\,\s$ in its own rest frame. In
the lab it travels $10\,\m$. At what $\gamma$ is the particle
traveling relative to the lab?
%
(from Problem Set~13, \problemname~2)

\vfill

\paragraph{\problemname~\theproblem:}\refstepcounter{problem}%
A particle of mass $M$, at rest, decays into two photons, traveling in
opposite directions. What are the two photon energies?
%
(from Problem Set~14, \problemname~3)

\vfill

\paragraph{\problemname~\theproblem:}\refstepcounter{problem}%
A car is driving at constant speed in a big circle of radius $R$.
There are frictional forces where the car's wheels touch the ground.
What is the direction of the force \emph{on} the car \emph{from} the
ground?
%
(from Lecture 2016-09-15)

\vfill
~
\clearpage

\paragraph{\problemname~\theproblem:}\refstepcounter{problem}%
You have a block of mass $m$ on an inclined
plane, inclined at an angle $\theta=20\,\deg$ to the horizontal. The
coefficient of friction is $\mu=0.9$. What is the magnitude of the
frictional force on the block? The acceleration due to gravity is $g$.
%
(from Recitation on friction)

\vfill

\paragraph{\problemname~\theproblem:}\refstepcounter{problem}%
What is the definition of an ``elastic collision''?
%
(from Lecture 2016-10-04)

\vfill

\paragraph{\problemname~\theproblem:}\refstepcounter{problem}%
A pendulum has mass $M$ and length $\ell$ and hangs in a room where
the gravitational acceleration is $g$. What combination of these
quantities has units of time?
%
(from Lecture 2016-10-18)

\vfill
~
\clearpage

\paragraph{\problemname~\theproblem:}\refstepcounter{problem}%
If a mass $m$ is attached to a spring of spring constant $k$, what is
its resonant (angular) frequency $\omega_0$?
%
(from Lecture 2016-10-27)

\vfill

\paragraph{\problemname~\theproblem:}\refstepcounter{problem}%
Plot the $x$ velocity $v_x$ as a function of time for a mass
harmonically oscillating in the $x$ direction with a period of
$1\,\s$. Also plot the kinetic energy. Make your plots so I can see
the time alignment of the two things, and label your time axis in one-second intervals.
%
(from Recitation on oscillations)

\vfill

\paragraph{\problemname~\theproblem:}\refstepcounter{problem}%
What is the difference in atmospheric pressure between the bottom of
this room and the top of this room, in Pa? The exam room is about
$9\,\m$ tall!
%
(from Lecture 2016-11-03)

\vfill
~
\clearpage

\paragraph{\problemname~\theproblem:}\refstepcounter{problem}%
A thin hoop and a uniform disk have the same radius and same mass.
Which has the larger moment of inertia?
%
(from Lecture 2016-11-10)

\vfill

\paragraph{\problemname~\theproblem:}\refstepcounter{problem}%
In my frame, an event happens at 12:00 at my location and another
event happens at 12:30 (30 minutes later in time) but 24 light-minutes
away from me. What is the proper time between these events? Express
your answer in minutes.
%
(from Lecture 2016-12-06)

\vfill

\paragraph{\problemname~\theproblem:}\refstepcounter{problem}%
A particle moves at speed $v$ in the $x$ direction. What is the
squared magnitude $|\vec{u}|^2$ of its 4-velocity?
%
(from Lecture 2016-12-08)

\vfill
~
\end{document}
