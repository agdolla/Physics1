\documentclass[12pt]{article}
\usepackage{graphics}
\begin{document}

\section*{NYU Physics 1---In-class Exam 3}

\vfill

\paragraph{Name:} ~

\paragraph{email:} ~

\vfill

This exam consists of two problems.  Write only in this booklet.  Be
sure to show your work.

\vfill ~

\clearpage

\section*{Problem 1}

\noindent~\hfill\includegraphics{../mp/rollinghill.eps}\hfill~

A thin hoop of mass $M$ and radius $R$ (with moment of inertia
$I=M\,R^2$) rolls without slipping along the hilly landscape shown.
If it was released from rest at point A, what is its speed $v$ at
point B?  Give your answer in terms of $h$, $g$, $M$, and $R$.  Use
conservation of energy, and recall that the kinetic energy $K$ of a
translating, rotating object is $K=(1/2)\,M\,v^2+(1/2)\,I\,\omega^2$.
Try to give your answer in terms of just $M$, $R$, $h$, and $g$.

\clearpage

\section*{Problem 2}

If you assume that air resistance comes from elastic collisions with
stationary air molecules (not a crazy assumption), then it turns out
that the air resistance force $F_\mathrm{air}$ acting on a moving car
can be computed simply from the density $\rho$ (mass per volume) of
air, the speed $v$ of the car, and the cross-sectional area $A$ of the
car.  Use either physical arguments or dimensional arguments to
estimate the air resistance force (in Newtons) acting on a typical car
driving at $55~\mathrm{mi\,hr^{-1}}$.  You will need to tell me what
you are assuming for the density of air $\rho$ and the cross-sectional
area $A$ of the car.  Reasonable estimates are preferred to
unreasonable ones!

\clearpage

[This page intentionally left blank for calculations or other work.]

\end{document}
