\documentclass[12pt]{article}
\usepackage{graphics}
\begin{document}

\section*{NYU Physics 1---Final Exam}

\vfill

\paragraph{Name:} ~

\paragraph{email:} ~

\paragraph{recitation:} ~

\vfill

This exam consists of six problems.  Write only in this booklet.  Be
sure to show your work.

\vfill

\emph{You must ensure that the proctor checks off your name on the
attendance sheet when you hand in your exam.  Your exam will
\textbf{not} be graded if your name is not checked off.}

\clearpage

\section*{Problem 1}

At time $t=0$, a stone is thrown vertically upward at
$5~\mathrm{m\,s^{-1}}$.  In a coordinate system with the $z$ direction
pointing vertically upward, sketch the position $z$, vertical velocity
$v_z$ and vertical acceleration $a_z$ of the stone as a function of
time, for the time inteval $0<t<(1~\mathrm{s})$ on three graphs below.
Assume that the stone begins at $z=0$ at $t=0$.  Be as quantitative as
you can and mark relevant times, distances, speeds, and accelerations
on your graphs where possible.  If it makes your mathematics simpler,
feel free to use $g= 10~\mathrm{m\,s^{-2}}$.

\clearpage

\section*{Problem 2}

Draw free-body diagrams for all the masses and pulleys in this
mechanism.  Find the accelerations of the two blocks.\\
\rule{0.35\textwidth}{0pt}
\resizebox{0.3\textwidth}{!}{\includegraphics{../mp/tackle_blocks.eps}}
\\ Be careful with your kinematic constraints, and assume that all the
strings and pulleys are massless and frictionless.

\clearpage

\section*{Problem 3}

A physicist standing at rest on a horizontal sheet of perfectly
frictionless ice catches a baseball thrown horizontally at
$120~\mathrm{km\,h^{-1}}$.  What is the final speed of the physicist
after the catch?  Clearly state any assumptions or approximations you
make; estimate numbers and give your answer in $\mathrm{km\,h^{-1}}$.

\clearpage

\section*{Problem 4}

A figure skater spins in place on frictionless ice at angular speed
$\omega_i$ with her hands outstretched.  She has a total moment of
inertia $I_i$.  As the skater draws her hands into her body, her
moment of inertia decreases to $I_f=I_i/2$.  Does her kinetic energy
$K$ increase, decrease, or stay the same?  If it increases, where does
the energy come from?  If it decreases, where does the energy go to?
\emph{Explain all your answers concisely but clearly.}

\clearpage

\section*{Problem 5}

A ``seconds'' pendulum has a period of exactly 2.0 seconds.

(a) If $g=9.793~\mathrm{m\,s^{-2}}$ in Austin Texas, how long should a
Texan make her or his seconds pendulum?

\vfill

(b) If $g$ is 0.1~percent (a factor of 1.001) larger in Paris, France,
than in Austin, how different, in percent, should the length of a
Parisian's seconds pendulum be than a Texan's?  Should it be longer or
shorter?  \emph{Note:} You can answer this question without getting
part (a) correct.

\vfill ~

\clearpage

\section*{Problem 6}

Jupiter and Earth both orbit in nearly circular orbits around the Sun.
The radius of Jupiter's orbit is about 5~times that of Earth's orbit.
The mass of Jupiter is about 1000~times that of Earth.  State clearly
any other assumptions you have to make.

(a) How long is a ``Jovian'' year?

\vfill

(b) Which planet has larger orbital kinetic energy, and why?

\vfill

\clearpage

[This page intentionally left blank for calculations or other work.]

\end{document}
