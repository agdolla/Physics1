\documentclass[12pt]{article} \usepackage{url, graphicx}

% page layout
\setlength{\topmargin}{-0.25in}
\setlength{\textheight}{9.5in}
\setlength{\headheight}{0in}
\setlength{\headsep}{0in}

% problem formatting
\newcommand{\problemname}{Problem}
\newcounter{problem}

% math
\newcommand{\dd}{\mathrm{d}}

% primary units
\newcommand{\rad}{\mathrm{rad}}
\newcommand{\kg}{\mathrm{kg}}
\newcommand{\m}{\mathrm{m}}
\newcommand{\s}{\mathrm{s}}

% secondary units
\renewcommand{\deg}{\mathrm{deg}}
\newcommand{\km}{\mathrm{km}}
\newcommand{\mi}{\mathrm{mi}}
\newcommand{\h}{\mathrm{h}}
\newcommand{\ns}{\mathrm{ns}}
\newcommand{\J}{\mathrm{J}}
\newcommand{\eV}{\mathrm{eV}}
\newcommand{\W}{\mathrm{W}}

% derived units
\newcommand{\mps}{\m\,\s^{-1}}
\newcommand{\mph}{\mi\,\h^{-1}}
\newcommand{\mpss}{\m\,\s^{-2}}

% random stuff
\sloppy\sloppypar\raggedbottom\frenchspacing\thispagestyle{empty}

\begin{document}

\noindent
Name: \rule[-1ex]{0.55\textwidth}{0.1pt}
NetID: \rule[-1ex]{0.2\textwidth}{0.1pt}

\section*{NYU Physics I---Term Exam 4}

\paragraph{\problemname~\theproblem:}\refstepcounter{problem}%
(from lecture 2018-10-30)
What is the pressure difference, roughly, between the top surface
of an ice cube and the bottom, when that ice cube is floating in water?
Imagine that the ice cube is $2\,\cm$ on a side, and make any other
(correct) assumptions you need to make. Give your answer in Pa or $\N\,\m^{-2}$.

\vfill

\paragraph{\problemname~\theproblem:}\refstepcounter{problem}%
(from Problem Set 7)
You calculated that a pendulum with a period of $2\,\s$ has a length
very close to $1\,\m$. What would be the length of a pendulum with
a period of $4\,\s$?

\vfill

\paragraph{\problemname~\theproblem:}\refstepcounter{problem}%
(from Problem Set 8)
A very thin ladder of length $L$ and mass $M$ leans against a
vertical wall, on a horizontal floor, making an angle of $\theta$ with respect to the
wall. Imagine that there is a large coefficient of friction $\mu$ at the floor such that
the ladder is in static equilibrium, but assume that the wall is effectively
frictionless. Draw a free-body diagram for the ladder, showing all forces acting.

\vfill
~
\clearpage

\paragraph{\problemname~\theproblem:}\refstepcounter{problem}%
(from Problem Set 8)
In the equation
$$
\frac{\dd^2 x}{\dd t^2} + Q\,\frac{\dd x}{\dd t} + P\,x = 0 \quad ,
$$ what are the units of $Q$ and $P$?

\vfill

\paragraph{\problemname~\theproblem:}\refstepcounter{problem}%
(from worksheet on potentials)
If you have a potential of the form
$$
U(x) = A\,x^3 - B\,x + C
$$ where $A$ and $B$ and $C$ are positive constants, find a location ($x$ position) at which
the force is zero.

\vfill

\paragraph{\problemname~\theproblem:}\refstepcounter{problem}%
(from worksheet on ideal gas)
You have a number density $n$ (units of particles per unit volume)
and they are moving at speed $v$ (units of length per time)
in the $x$ direction.
What is the rate $\Gamma$ (units of particles per unit time) at which they
will hit a wall of area $A$ (units of length squared)?
The wall is oriented perpendicular to the $x$ direction.

\vfill
~
\end{document}
