\documentclass[12pt]{article}
\usepackage[pdftex]{graphicx}

\begin{document}
\newcounter{problem}
\thispagestyle{empty}

\section*{NYU General Physics 1---Problem set 4}

\paragraph{Problem~\theproblem:}\refstepcounter{problem}%
Look up the definition of the ``coefficient of friction'' and the
difference between ``static friction'' and ``sliding'' or ``kinetic''
friction.  Once you have done that, re-do the block on an inclined
plane problem but with friction and the following parameters:

\textsl{(a)} The block has mass $m$.  The plane is inclined at an
angle $\theta=35\,\deg$ to the horizontal.  The block-plane interface
has a frictional force with coefficient of friction $\mu=0.1$.  Draw
the free-body diagram.  Draw the vector force sum as a vector diagram
in two dimensions (as we did in lecture).  Compute the magnitude $a$
of the acceleration; give your answer in terms of the magnitude $g$ of
the acceleration due to gravity.  Compute the magnitude $f$ of the
frictional force in terms of $m$, $g$, and $\theta$.

\textsl{(b)} The same again, but now with a coefficient of friction
$\mu=1.3$.  Note that this case will be \emph{qualitatively
  differerent} because the friction will be static.

\textsl{(c)} What is the minimum coefficient of friction $\mu$ at
which the system will be static?  Give your answer in terms of
$\theta$.

\paragraph{Problem~\theproblem:}\refstepcounter{problem}%
In the following device, the string can be approximated as massless
and inextensible, the pulley can be approximated as massless and
frictionless, but the horizontal surface has a coefficient of friction
(static and kinetic) of $\mu$.\\[1ex]
\includegraphics{../mp/pulley_table.pdf}

\textsl{(a)} What do you think this system will do, if released from
rest?  Consider the case that the coefficient of friction $\mu$ is
very large, and also the case in which the coefficient of friction is
very small.  What would it do in the limit that $m_2$ is far larger
than $m_1$?  What about in the limit that $m_1$ is far larger than
$m_2$?  Don't calculate here; use your intuition.

\textsl{(b)} Obtain the acceleration of the blocks and the tension in
the string for the case of kinetic friction and for the case of static
friction.  Note that you do different calculations in each case.  Give
your answers in terms of $m_1$, $m_2$, $\vec{g}$, and $\mu$ (or a
subset of these).

\textsl{(c)} Do your answers to \textsl{(b)} mesh
with the intuitions you wrote down in part \textsl{(a)}?

\paragraph{Problem~\theproblem:}\refstepcounter{problem}%
A ball of mass $M$ swings like a pendulum on a light, inextensible
string of length $L$.  Imagine that the pendulum is swinging back and
forth with a significant amplitude (say an amplitude of $L/4$).  Think
about the tension in the string.  At the lowest point in the swing, is
the tension in the string equal to, less than, or greater than $M\,g$?
Explain why in words.

Don't \emph{calculate} the tension unless you really want to.

\end{document}
