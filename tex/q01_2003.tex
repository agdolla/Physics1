\documentclass[12pt]{article}
\begin{document}

\section*{NYU Physics 1---In-class Exam 1}

\vfill

\paragraph{Name:} ~

\paragraph{email:} ~

\vfill

This exam consists of two problems.  Write only in this booklet.  Be
sure to show your work.

\vfill ~

\clearpage

\section*{Problem 1}

Estimate the total mass of the Silver building on Washington Square.
In case you don't remember, the building takes up one full city block,
is made of stone, and is about 10 stories tall.  Clearly state any
assumptions you make.  Do not try to get a precise answer; an order of
magnitude is sufficient.

\clearpage

\section*{Problem 2}

\textsl{(a)}~At time $t=0$, a stone is thrown vertically upward at
$5~\mathrm{m\,s^{-1}}$.  In a coordinate system with the $z$ direction
pointing vertically upward, sketch the position $z$, vertical velocity
$v_z$ and vertical acceleration $a_z$ of the stone as a function of
time, for the time interval $0<t<1~\mathrm{s}$ on three graphs below.
Assume that the stone begins at $z=0$ at $t=0$.  Be as quantitative as
you can and mark relevant times, distances, speeds, and accelerations
on your graphs where possible.  If it makes your mathematics simpler,
feel free to use $g= 10~\mathrm{m\,s^{-2}}$.  Ignore air resistance.

\clearpage

\section*{Problem 2 continued}

\textsl{(b)}~Draw a free body diagram for the stone, ie, draw all of
the forces acting on the stone, at the moment that the stone is at its
maximum $z$ position.

\end{document}
