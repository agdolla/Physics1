\documentclass[12pt]{article}
\usepackage{graphics}
\begin{document}
\newcounter{problem}
\thispagestyle{empty}

\section*{NYU Physics 1---Problem set 12}

Due Tuesday 2009 December 15 at the beginning of lecture.

\paragraph{Problem~\theproblem:}\refstepcounter{problem}%
Compare the angular momentum of the Earth's \emph{spin} to the angular
momentum in the Earth's \emph{orbit} around the Sun.  Which is larger,
and why?  Why is it physically impossible that the Earth spin so fast
that it have as much angular momentum in its spin as in its orbit?  I
want to see an argument that does \emph{not} refer to relativity.
Even in the absence of relativistic effects, why would this be
impossible?  \emph{Hint:}~Think about gravitational forces.

\paragraph{Problem~\theproblem:}\refstepcounter{problem}%
\textsl{(a)}~A figure skater spins in place on frictionless ice at
angular speed $\omega_i$ with her hands outstretched.  She has a total
moment of inertia $I_i$.  As the skater draws her hands into her body,
her moment of inertia decreases to $I_f=I_i/2$.  Does her kinetic
energy $K$ increase, decrease, or stay the same?  If it increases,
where does the energy come from?  If it decreases, where does the
energy go to?  \emph{Explain all your answers concisely but clearly:
What is conserved?}

\textsl{(b)}~Now estimate the moments of inertia: $I_i$ of an ice
skater with her hands outstretched, and $I_f$ of an ice skater with
her hands drawn in.  Is the factor of 2 used in part \textsl{(a)}
reasonable?

\paragraph{Problem~\theproblem:}\refstepcounter{problem}%
\textsl{(a)}~Immediately after being hit, at $t=0$, a cue ball of mass
$M$ and radius $R$ slides along the felt at speed $v_i$, not rotating
at all.  As time goes on, the ball slows down (because of friction)
and, at the same time, starts to spin.  Draw a free-body diagram for
the cue ball.  At what time $t_\mathrm{r}$ does the ball get to the
situation of ``rolling without slipping''?  Assume that there is a
coefficient $\mu$ of sliding friction.

\textsl{(b)}~Plot $v(t)$ and $R\,\omega(t)$ vs $t$ on a single plot.
\emph{Note that the two things I have asked you to plot have the same
dimensions.}  Clearly label $t_\mathrm{r}$ on your diagram.

\textsl{(c)}~The area between the two curves has dimensions of length.
What is the meaning of this length?  It is the distance of what?

\paragraph{Extra Problem (will not be graded for credit):}%
A pool ball of mass $m$ rolls without slipping at speed $v$ towards a
bumper on the pool table.  The bumper contacts the ball above the
midline, that is, at a height $h>R$ above the surface of the table.
When the ball hits the bumper, the impulse reverses the direction of
the ball, \emph{and,} if the bumper is properly designed, the
direction of \emph{rotation} of the ball.

\textsl{(a)}~What is the optimal height $h$ for the bumper to contact
the ball?  Give your answer in terms of the ball radius $R$.
\emph{Hint:} Treat the point of contact as providing a large force
over a short time, in the exactly horizontal direction to reverse the
motion; now work out the angular impulse from that contact.

\textsl{(b)}~Now go measure the height of the contact point of a
bumper and the diameter of a pool ball in your dorm or a nearby pool
hall.  Is it high or low?  Be careful to measure the point of
\emph{contact} of the bumper.  If you get an answer that is high or
low, does the bumper seem to work okay nonetheless?  Also, you might
take the time to shoot a few games and think about last week's problem
set.

\textsl{(c)}~How would your answer to part~(a) change if pool balls
were thin-walled hollow spheres instead of solid spheres?

\textsl{(d)}~What are you assuming when you assume that the force is
purely horizontal?  Warning, it is not trivial.

\end{document}
